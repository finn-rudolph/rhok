\documentclass[a4paper, 10pt, ngerman]{article}

\usepackage[algoruled, nosemicolon]{algorithm2e}
\usepackage{amsmath}
\usepackage{amssymb}
\usepackage[ngerman]{babel}
\usepackage[backend=biber, style=apa]{biblatex}
\usepackage[left=2.5cm, right=2.5cm, top=2.5cm, bottom=2.cm]{geometry}
\usepackage[hidelinks]{hyperref}
\usepackage[onehalfspacing]{setspace}
\usepackage{sectsty}

\allsectionsfont{\sffamily}

\title{\sffamily{\textbf{Parametrisierung von Pollards Rho-Methode}}}
\author{Finn Rudolph}
\date{03.01.2024}

\addbibresource{rhok.bib}

\renewcommand{\thealgocf}{}

\newcommand{\N}{\mathbb{N}}
\newcommand{\Z}{\mathbb{Z}}
\newcommand{\E}{\mathbb{E}}

\newtheorem{definition}{Definition}
\newtheorem{assumption}{Annahme}

\begin{document}

\maketitle

\section*{Projektüberblick}

\tableofcontents

\section{Zusammenfassung}

\section{Motivation und Fragestellung}

Während für Rekordfaktorisierungen mittlerweile ausschließlich das Zahlkörpersieb verwendet wird, bleibt Pollards Rho-Algorithmus einer schnellsten Algorithmen zur Faktorisierung von Zahlen bis ca. $2^{64}$. Da Leistungssteigerungen bei modernen Computern häufig durch verbesserete Nebenläufigkeit (z. B. durch mehr Prozessorkerne) erzielt werden, ist es eine interessante Frage, wie Pollards Rho-Methode am besten parallel ausgeführt werden kann. Bevor die Fragestellung präzise formuliert werden kann, soll jedoch Pollards Rho-Methode erklärt werden.

\subsection{Pollards Rho-Methode}

Sei $n$ die zu faktorisierende Zahl und $f : \Z/n\Z \to \Z/n\Z$ mit $f : x \mapsto x^{2k} + 1$ für einen Parameter $1 \le k \in \N$. Man wähle einen zufälligen Anfangswert $x_0 \in \Z/n\Z$ und betrachte die Folge $(x_n)_{n \in \N}$ definiert durch $x_n = f(x_{n - 1})$ für einen Parameter $1 \le k \in \N$. Da $(x_n)_{n \in \N}$ über der endlichen Menge $\Z/n\Z$ definiert ist, ist die Folge ab einem bestimmten Punkt periodisch. Sei $p$ ein Primfaktor von $n$ und $\pi : \Z/n\Z \to \Z/p\Z$ die natürliche Projektion. Die Idee von Pollards Rho-Algorithmus ist, zwei Folgenglieder $x_i, x_j \in \Z/n\Z$ zu finden, sodass $x_i \ne x_j$ aber $\pi(x_i) = \pi(x_j)$. Dann ist nämlich $\gcd(n, x_i - x_j)$ ein echter Faktor von $n$. Nimmt man nun heuristisch an, dass die Periodenlänge von $(x_n)_{n \in \N}$ in $\Z/n\Z$ deutlich länger als die Periodenlänge in $\Z/p\Z$ ist, reicht es aus, $x_i, x_j$ mit $i \ne j$ zu finden, die kongruent modulo $p$ sind. Diese Annahme begründet sich darin, dass für den kleinsten Primfaktor $p \le \sqrt n$ gilt. Zum Finden solcher $x_i, x_j$ betrachten wir den funktionalen Graphen von $\pi(f)$, wobei $\pi(f)$ die Abbildung $f$ betrachtet in $\Z/p\Z$ ist.

\begin{definition}[Funktionaler Graph]
    Sei $X$ eine endliche Menge und $f: X \to X$ eine Abbildung. Der funktionale Graph von $f$, geschrieben $\gamma(f)$, ist der gerichtete Graph mit Knotenmenge $X$ und Kantenmenge $E$, wobei die Kante $(x, y) \in X \times X$ genau dann in $E$ liegt, wenn $f(x) = y$.
\end{definition}

\noindent Es ist leicht zu zeigen, dass jede Zusammenhangskomponente eines funktionalen Graphen aus einem Zyklus und an den Zyklusknoten gewurzelten Bäumen besteht. In $\gamma(f)$ betrachtet startet $(x_n)_{n \in \N}$ mit $x_0$ in einem Baum und "`läuft"' durch den Graphen, wobei immer die eindeutige von einem Knoten ausgehende Kante entlanggegangen wird. An der Wurzel des Baums von $x_0$ wird der Zyklus in der Zusammenhangskomponente von $x_0$ betreten, und ab genau diesem Punkt ist $(x_n)_{n \in \N}$ periodisch. Um $x_i, x_j$ mit $i \ne j$ und $\pi(x_i) = \pi(x_j)$ zu finden, wird Floyds Algorithmus zum Finden des Zyklus in der Zusammenhangskomponente von $\pi(x_0)$ in $\gamma(\pi(f))$ verwendet. Floyds Algorithmus macht sich zunutze, dass es ein $r \in \N, 1 \le r \ge 1$ mit $\pi(x_r) = \pi(x_{2r})$ geben muss. Für das minimale solcher $r$ gilt außerdem $r \le \mu + \lambda$, wobei $\mu$ die Höhe von $x_0$ in seinem Baum und $\lambda$ die Länge des Zyklus ist (\cite{knu98}, S. 7). Wir nennen $\lambda + \mu$ auch die Rho-Länge von $x_0$. Indem man die Folgen $(x_n)_{n \in \N}$ und $(x_{2n})_{n \in \N}$ gleichzeitig Glied für Glied berechnet, stößt man in maximal $\mu + \lambda$ Schritten auf gewünschte $x_i, x_j$. Das Überprüfen in jedem Schritt, ob $\pi(x_i) = \pi(x_{2i})$ geschieht natürlich nicht explizit, da $p$ unbekannt ist, aber implizit durch Berechnung von $\gcd(n, x_i - x_{2i})$.

\begin{algorithm*}
    $x \gets $ zufällige natürliche Zahl zwischen $0$ und $n - 1$ \;
    $y \gets x$ \;
    \While{\emph{\sc{true}}}
    {
        $x \gets x^{2k} + 1 \mod n$ \;
        $y \gets (y^{2k} + 1)^{2k} + 1 \mod n$ \;
        $g \gets \gcd(n, x - y)$ \;
        \If{$g \ne 1 \text{\emph{\textbf{ and }}} g \ne n$}
        {
            \Return{$g$} \;
        }
    }

    \caption{Pollards Rho-Algorithmus}
\end{algorithm*}

\noindent Für $k = 1$ ist die erwartete Anzahl an Iterationen der while-Schleife unter heuristischen Annahmen $\sqrt{\pi p / 2}$ (\cite{knu98}, S. 8). Da die Berechnung des größten gemeinsamen Teilers $O(\ln n)$ Schritte benötigt, ist die erwartete Laufzeit des Algorithmus $O(\sqrt p \ln n)$. Durch eine einfache Modifikation können die Kosten des $\gcd$ amortisiert werden, sodass sich die Laufzeit auf $O(\sqrt p)$ verringert (\cite{bre80}). Damit ist pro Iteration also nur noch die Zeit zur Berechnung der $2k$-ten Potenzen von $x$ und $y$ relevant, was durchschnittlich in $3/2 \lg 2k$ Schritten möglich ist. Mit $\lg x$ wird der Logarithmus zur Basis 2 bezeichnet. Ebenfalls unter heuristischen Annahmen wird die Anzahl an Iterationen durch Veränderung von $k$ um einen Faktor $1/\sqrt{\gcd(p - 1, 2k) - 1}$ verringert (\cite{bp81}). Die Kosten pro Iteration steigen wegen der Berechnung von $x^{2k}$ aber um einen Faktor $\lg 2k$.

\subsection{Parallelisierung der Rho-Methode}

Sei $M$ die Anzahl verfügbarer Maschinen. Eine "`Maschine"'  meint hier nicht zwingend einen Computer, sondern eine Ressource, auf der ein sequentielles Programm ausgeführt werden kann, was beispielsweise auch ein Prozessorthread sein kann. Die Rho-Methode lässt sich parallelisieren, indem $M$ Anfangswerte unabhängig voneinander zufällig gewählt werden und auf jeder der $M$ Maschinen Pollards Rho-Algorithmus ausgeführt wird, bis eine der Maschinen einen Faktor findet. Nun ergibt sich folgende Frage, die in dieser Arbeit behandelt werden soll: Wie wählt man den Parameter $k$ für jede Maschine optimal, um eine möglichst geringe Laufzeit zu erzielen? Eine Zuordnung von $k$-Werten für $M$ Maschinen schreiben wir als Tupel $K = (k_1, k_2, \dots, k_M), 1 \le k_i \in \N$. Wir bezeichnen mit $E(K)$ die erwartete Laufzeit des parallelen Pollard-Rho-Algorithmus mit $k$-Werten gegeben durch $K$, also
\begin{align*}
    E(K) = \E \bigg ( \min_{1 \le i \le M} X_i \lg 2k_i \bigg )
\end{align*}
wobei $X_i$ die Zufallsvariable ist, die dem Anfangswert der $i$-ten Maschine dessen Rho-Länge zuordnet.

// MAYBE SAGEN DASS MAN HIER WIEDER IRGENDNE RANDOM ANNAHME MACHT

\section{Eine Formel für unabhängige Maschinen}

In diesem Abschnitt wird eine Formel für $E(K)$ hergleitet werden, die im Fall paarweise verschiedener $k$-Werte gilt.

\section{Die erwartete Rho-Länge in zufälligen Funktionen}

\subsection{Der Fall zweier Startpunkte}

\subsection{Mögliche Methoden für den allgemeinen Fall}

\section{Bestimmung optimaler Exponenten für die Rho-Methode}

\subsection{Der Fall einer Maschine}

\subsection{Der Fall zweier Maschinen}

\section{Experimentelle Ergebnisse}

\section{Fazit}

\printbibliography

\end{document}