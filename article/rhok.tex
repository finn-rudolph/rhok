\documentclass[a4paper, 10pt, ngerman]{article}

\usepackage[algoruled, nosemicolon]{algorithm2e}
\usepackage{amsmath}
\usepackage{amssymb}
\usepackage{amsthm}
\usepackage[ngerman]{babel}
\usepackage[backend=biber, style=apa]{biblatex}
\usepackage[left=2.5cm, right=2.5cm, top=2.5cm, bottom=2.cm]{geometry}
\usepackage[hidelinks]{hyperref}
\usepackage{mathtools}
\usepackage[onehalfspacing]{setspace}
\usepackage{sectsty}
\usepackage[inkscapeformat=pdf]{svg}

\allsectionsfont{\sffamily}

\title{\sffamily{\textbf{Parametrisierung von Pollards Rho-Methode}}}
\author{Finn Rudolph}
\date{27.01.2024}

\addbibresource{rhok.bib}

\renewcommand{\thealgocf}{}

\newcommand{\C}{\mathbb{C}}
\newcommand{\E}{\mathbb{E}}
\newcommand{\N}{\mathbb{N}}
\newcommand{\R}{\mathbb{R}}
\newcommand{\Z}{\mathbb{Z}}
\renewcommand{\P}{\mathbb{P}}

\newtheorem{definition}{Definition}
\newtheorem{theorem}{Satz}
\newtheorem{lemma}{Lemma}
\newtheorem{assumption}{Annahme}

\newcommand*\uparagraph{%
        \par
        \nopagebreak
        \vskip3.25ex plus1ex minus.2ex
        \noindent
 }

\begin{document}

\maketitle

\section*{Projektüberblick}

\tableofcontents

\section{Zusammenfassung}

\section{Motivation und Fragestellung}

Während für Rekordfaktorisierungen mittlerweile ausschließlich das Zahlkörpersieb verwendet wird, bleibt Pollards Rho-Algorithmus einer schnellsten Algorithmen zur Faktorisierung von Zahlen bis ca. $2^{64}$. Da Leistungssteigerungen bei modernen Computern häufig durch verbesserete Nebenläufigkeit (z. B. durch mehr Prozessorkerne) erzielt werden, ist es eine interessante Frage, wie Pollards Rho-Methode am besten parallel ausgeführt werden kann. Bevor die Fragestellung präzise formuliert werden kann, soll jedoch Pollards Rho-Methode erklärt werden.

\subsection{Pollards Rho-Methode}

Sei $n$ die zu faktorisierende Zahl und $f : \Z/n\Z \to \Z/n\Z$ mit $f : x \mapsto x^{2k} + 1$ für einen Parameter $1 \le k \in \N$. Man wähle einen zufälligen Anfangswert $x_0 \in \Z/n\Z$ und betrachte die Folge $(x_n)_{n \in \N}$ definiert durch $x_n = f(x_{n - 1})$ für einen Parameter $1 \le k \in \N$. Da $(x_n)_{n \in \N}$ über der endlichen Menge $\Z/n\Z$ definiert ist, ist die Folge ab einem bestimmten Punkt periodisch. Sei $p$ ein Primfaktor von $n$ und $\pi : \Z/n\Z \to \Z/p\Z$ die natürliche Projektion. Die Idee von Pollards Rho-Algorithmus ist, zwei Folgenglieder $x_i, x_j \in \Z/n\Z$ zu finden, sodass $x_i \ne x_j$ aber $\pi(x_i) = \pi(x_j)$. Dann ist nämlich $\gcd(n, x_i - x_j)$ ein echter Faktor von $n$. Nimmt man nun heuristisch an, dass die Periodenlänge von $(x_n)_{n \in \N}$ in $\Z/n\Z$ deutlich länger als die Periodenlänge in $\Z/p\Z$ ist, reicht es aus, $x_i, x_j$ mit $i \ne j$ zu finden, die kongruent modulo $p$ sind. Diese Annahme begründet sich darin, dass für den kleinsten Primfaktor $p \le \sqrt n$ gilt. Zum Finden solcher $x_i, x_j$ betrachten wir den funktionalen Graphen von $\pi(f)$, wobei $\pi(f)$ die Abbildung $f$ betrachtet in $\Z/p\Z$ ist.

\begin{definition}[Funktionaler Graph]
    Sei $X$ eine endliche Menge und $f: X \to X$ eine Abbildung. Der funktionale Graph von $f$, geschrieben $\gamma(f)$, ist der gerichtete Graph mit Knotenmenge $X$ und Kantenmenge $E$, wobei die Kante $(x, y) \in X \times X$ genau dann in $E$ liegt, wenn $f(x) = y$.
\end{definition}

\noindent Es ist leicht zu zeigen, dass jede Zusammenhangskomponente eines funktionalen Graphen aus einem Zyklus und an den Zyklusknoten gewurzelten Bäumen besteht. In $\gamma(f)$ betrachtet startet $(x_n)_{n \in \N}$ mit $x_0$ in einem Baum und "`läuft"' durch den Graphen, wobei immer die eindeutige von einem Knoten ausgehende Kante entlanggegangen wird. An der Wurzel des Baums von $x_0$ wird der Zyklus in der Zusammenhangskomponente von $x_0$ betreten, und ab genau diesem Punkt ist $(x_n)_{n \in \N}$ periodisch. Um $x_i, x_j$ mit $i \ne j$ und $\pi(x_i) = \pi(x_j)$ zu finden, wird Floyds Algorithmus zum Finden des Zyklus in der Zusammenhangskomponente von $\pi(x_0)$ in $\gamma(\pi(f))$ verwendet. Floyds Algorithmus macht sich zunutze, dass es ein $r \in \N, 1 \le r \ge 1$ mit $\pi(x_r) = \pi(x_{2r})$ geben muss. Für das minimale solcher $r$ gilt außerdem $r \le \mu(f, x_0) + \lambda(f, x_0)$, wobei $\mu(f, x_0)$ die Höhe von $x_0$ in seinem Baum und $\lambda(f, x_0)$ die Länge des Zyklus ist (\cite{knu98}, S. 7). Wir nennen $\nu(f, x_0) = \mu(f, x_0) + \lambda(f, x_0)$ die Rho-Länge von $x_0$ in $f$. Indem man die Folgen $(x_n)_{n \in \N}$ und $(x_{2n})_{n \in \N}$ gleichzeitig Glied für Glied berechnet, stößt man in maximal $\nu(f, x_0)$ Schritten auf gewünschte $x_i, x_j$. Das Überprüfen in jedem Schritt, ob $\pi(x_i) = \pi(x_{2i})$ geschieht natürlich nicht explizit, da $p$ unbekannt ist, aber implizit durch Berechnung von $\gcd(n, x_i - x_{2i})$.

\begin{algorithm*}
    $x \gets $ zufällige natürliche Zahl zwischen $0$ und $n - 1$ \;
    $y \gets x$ \;
    \While{\emph{\sc{true}}}
    {
        $x \gets x^{2k} + 1 \mod n$ \;
        $y \gets (y^{2k} + 1)^{2k} + 1 \mod n$ \;
        $g \gets \gcd(n, x - y)$ \;
        \If{$g \ne 1 \text{\emph{\textbf{ and }}} g \ne n$}
        {
            \Return{$g$} \;
        }
    }

    \caption{Pollards Rho-Algorithmus}
\end{algorithm*}

\noindent Die Analyse von Pollards Rho-Algorithmus erweist sich als schwierig, es ist bis dato keine rigorose Laufzeitanalyse bekannt. Stattdessen werden heuristische Annahmen getroffen. Anstatt der mittleren Anzahl an Iterationen von Floyds Algorithmus wird die mittlere Rho-Länge analysiert, mit der Annahme, dass sich diese zwei Werte für große $p$ nur um eine feste Konstante unterscheiden. Um die zentrale Annahme über Pollards Rho-Algorithmus zu formulieren, wird der Begriff einer asymptotischen Näherung benötigt.

\begin{definition}[Asymptotische Näherung]
    Eine Funktion $f : \R \to \R$ heißt genau dann asymptotische Näherung von einer Funktion $g : \R \to \R$, oder asymptotisch zu $g$, wenn
    \begin{align*}
        \lim_{x \to \infty} \frac {f(x)} {g(x)} = 1
    \end{align*}
    In diesem Fall schreiben wir $f \sim g$.
\end{definition}

Sei $A(n)$ die Menge der Abbildungen $\Z/n\Z \to \Z/n\Z$ für $n \in \N$. Über die Verteilung der Rho-Längen wird Folgendes angenommen.

\begin{assumption}
    Sei $f : x \mapsto x^{2k} + 1$ und $d = \gcd(p - 1, 2k)$. Seien $x_0 \in \Z/p\Z$ und $y_0 \in \Z/(p/(d - 1))\Z$ zufällig und $g \in A(p/(d - 1))$ zufällig. Dann gilt $\P(\nu(f, x_0) = m) \sim \P(\nu(g, y_0) = m)$ für $p \to \infty$.
\end{assumption}

\noindent In anderen Worten sagt Annahme 1, dass sich die Verteilung der Rho-Längen von $f : x \mapsto x^{2k} + 1$ wie bei einer zufälligen Funktion aus $A(p/(d - 1))$ verhält. Insbesondere verhält sich $x \mapsto x^2 + 1$ bezüglich der Rho-Längen wie ein zufällige Funktion $\Z/p\Z \to \Z/p\Z$. \cite{bp81} geben eine Begründung für Annahme 1.

Für $k = 1$ ist damit die erwartete Anzahl an Iterationen der while-Schleife asymptotisch zu $\sqrt{\pi p / 2}$ (\cite{knu98}, S. 8). Da die Berechnung des größten gemeinsamen Teilers $O(\ln n)$ Schritte benötigt, ist die erwartete Laufzeit des Algorithmus $O(\sqrt p \ln n)$. Durch eine einfache Modifikation können die Kosten des $\gcd$ amortisiert werden, sodass sich die Laufzeit auf $O(\sqrt p)$ verringert (\cite{bre80}). Damit ist pro Iteration also nur noch die Zeit zur Berechnung der $2k$-ten Potenzen von $x$ und $y$ relevant, was durchschnittlich in $3/2 \lg 2k$ Schritten möglich ist. Mit $\lg x$ wird der Logarithmus zur Basis 2 bezeichnet. Der Faktor $3/2$ in dieser Arbeit unwichtig und wird daher ignoriert.

\subsection{Parallelisierung der Rho-Methode}\label{sec:par}

Sei $M$ die Anzahl verfügbarer Maschinen. Eine "`Maschine"'  meint hier nicht zwingend einen Computer, sondern eine Ressource, auf der ein sequentielles Programm ausgeführt werden kann, was beispielsweise auch ein Prozessorthread sein kann. Die Rho-Methode lässt sich parallelisieren, indem $M$ Anfangswerte unabhängig voneinander zufällig gewählt werden und auf jeder der $M$ Maschinen Pollards Rho-Algorithmus ausgeführt wird, bis eine der Maschinen einen Faktor findet. Nun ergibt sich folgende Frage, die in dieser Arbeit behandelt werden soll: \emph{Wie wählt man den Parameter $k$ für jede Maschine optimal, um eine möglichst geringe Laufzeit zu erzielen?} Eine Zeiteinheit ist hier die Dauer einer Iteration im Fall $k = 1$, d.h. eine Maschine mit Paramter $k$ führt in $t$ Zeiteinheiten $\lfloor t / \lg 2k \rfloor$ Iterationen aus. Da es sich bei den Veränderungen in der Laufzeit durch Veränderung von $k$ um konstante Faktoren handelt, wird für den Vergleich wird die Laufzeit nicht $O$-Notation verwendet, sondern eine asymptotische Näherung für die erwartete Zahl an Zeiteinheiten bestimmt, wenn $p \to \infty$. Im Gegensatz zur $O$-Notation kann zwischen zwei Funktionen, die asymptotisch zueinander sind, für große $n$ kein konstanter Faktor liegen, sodass sich Veränderungen um konstante Faktoren sinnvoll vergleichen lassen.

Eine Zuordnung von $k$-Werten für $M$ Maschinen wird als Tupel $K = (k_1, k_2, \dots, k_M), 1 \le k_i \in \N$ geschrieben.  Wir bezeichnen mit $L_K(p)$ die erwartete Laufzeit des parallelen Pollard-Rho-Algorithmus mit $k$-Werten gegeben durch $K$ und fixiertem $\gcd(p - 1, 2k_i)$ für alle $1 \le i \le M$. Der Erwartungswert über alle Möglichkeiten von $\gcd(p - 1, 2k_i)$ wird erst in Abschnitt \ref{sec:optk} behandelt. Sei $h_i = p/(\gcd(p - 1, 2k_i) - 1)$. Mit Annahme 1 gilt
\begin{align*}
    L_K(p) = \E \bigg ( \min_{i = 1}^M X_i \bigg )
\end{align*}
wobei $X_i$ die gleichverteilte Zufallsvariable über $A(h_i) \times \Z/h_i\Z$ mit $X_i(f, x_0) = \nu(f, x_0)$ ist. Eine Schwierigkeit in der Herleitung einer Formel für $L_K(p)$ ist, dass $X_i$ und $X_j$ nicht unabhängig sind, wenn $k_i = k_j$, da sich die $i$-te und $j$-te Maschine in diesem Fall im gleichen funktionalen Graphen bewegen. Wenn beispielsweise der Startwert der $i$-ten Maschine fixiert wird, ist bereits klar, dass die Rho-Länge der $j$-ten Maschine größer gleich der Rho-Länge der $i$-ten Maschine sein wird, wenn der Startwert ein Vorfahre des Startwerts der $i$-ten Maschine in einem Baum von $\gamma(f)$ ist. Daher werden im Folgenden die Fälle abhängiger und unabhängiger Maschinen unterschieden.


\section{Eine Formel für den Fall unabhängiger Maschinen}

In diesem Abschnitt wird eine Formel für $L_K(p)$ hergleitet, die im Fall paarweise verschiedener $k$-Werte gilt. Aus Annahme 1 folgt, dass die Rho-Längen der $i$-ten und $j$-ten Maschine als Zufallsvariablen betrachtet stochastisch unabhängig sind, wenn $k_i \ne k_j$. Sei $h_i = p/(\gcd(p - 1, 2k_i) - 1)$ und $s_i$ die Anzahl an Iterationen, nach denen bei der $i$-ten Maschine erstmals eine Kollision auftritt. Sei $t_i = s_i \lg 2k_i$ die Zeit, nach der bei Maschine $i$ erstmals eine Kollision auftritt und $t_{\min} = \min_{i = 1}^M t_i$. Das Ziel ist die Bestimmung von $\E(t_{\min})$.


\paragraph{Eine Formel für $\pmb{\P(s_i > s)}$.} \space \space Nach Annahme 1 gilt $\P(s_i = s) = \P(\nu(f, x_0) = s)$ für eine zufällig gewählte Funktion $f \in A(h_i)$ und ein zufälliges $x_0 \in \Z/h_i\Z$. Für eine zufällige Funktion $f \in A(h_i)$ ist die Wahrscheinlichkeit einer Kollision im $i$-ten Schritt $i/h_i$, wenn in den ersten $i-1$ Schritten keine Kollision aufgetreten ist, da jeder der $h_i$ möglichen Werte gleich wahrscheinlich ist und $i$ von ihnen zu einer Kollision führen. Die Wahrscheinlichkeit, dass in den ersten $s$ Schritten keine Kollision auftritt ist also
\begin{align*}
    \P(s_i > s) = \P(\nu(f, x_0) > s) = \prod_{j = 0}^s \bigg (1 - \frac {j} {h_i} \bigg )
\end{align*}
Da die $i$-te Maschine in $t$ Zeiteinheiten $\lfloor t / \lg 2k_i \rfloor$ Schritte ausführt gilt $\P(t_i > t) \sim \P(s_i > t / \lg 2k_i)$ für große $t$. Das Weglassen der Gaußklammern lässt sich damit rechtfertigen, dass für ausreichend großes $p$ nur große $t$ relevant für den Erwartungswert von $t_{\min}$ sind, bei denen es kaum einen Unterschied macht. Nun wird die Restgliedabschätzung $\exp x = 1 + x + O(x^2)$ für $|x| < 1$ mit $x = -j/h_i$ angewandt.
\begin{align*}
    \P(s_i > s) = \prod_{j = 0}^{s} \Bigg ( \exp \bigg ( \frac {-j}{h_i} \bigg ) - O \bigg ( \frac {j^2} {h_i^2} \bigg ) \Bigg )
\end{align*}
$\E(t_{\min})$ lässt sich nun wie folgt ausdrücken, wobei $t_{\max} = p \, \max_{i = 0}^M \lg 2k_i$ ist die maximal mögliche Anzahl an Zeiteinheiten ist.
\begin{align*}
    \E(t_{\min})
     & \sim \sum_{t = 0}^{t_{\max}} t \, \Bigg ( \prod_{i = 1}^M \P(t_i > t - 1) \Bigg ) \ \P_{t_i > t - 1 \, \forall i}(t_i = t \text{ für mindestens ein } i)                                                                                                           \\
     & = \sum_{t = 0}^{t_{\max}} t \, \Bigg ( \prod_{i = 1}^M \prod_{j = 0}^{(t-1) / \lg 2k_i} \Bigg ( \exp \bigg ( \frac {-j}{h_i} \bigg ) - O \bigg ( \frac {j^2} {h_i^2} \bigg ) \Bigg )\Bigg ) \ \P_{t_i > t - 1 \, \forall i}(t_i = t \text{ für mindestens ein } i)
\end{align*}
Der erste Faktor ist die Wahrscheinlichkeit, dass vor Zeitpunkt $t$ keine Kollision aufgetreten ist. Der zweite Faktor ist die Wahrscheinlichkeit, dass bei Zeitpunkt $t$ mindestens eine Kollision auftritt. $\P_{t_i > t - 1}(t_i > t)$ steht für die Wahrscheinlichkeit, dass $t_i > t$, gegeben $t_i > t - 1$. Bevor diese Wahrscheinlichkeit bestimmt wird, soll gezeigt werden, dass der erste Faktor durch Weglassen der $O(j^2/h_i^2)$-Terme vereinfacht werden kann.


\paragraph{Weglassen von $\pmb{O(j^2/h_i^2)}$.} Intuitiv lässt sich das Weglassen von $O(j^2/h_i^2)$ damit rechtfertigen, dass wenn $s$ deutlich kleiner als $h_i$ ist, auch $j$ deutlich kleiner als $h_i$ ist und damit $j^2/h_i^2$ nahe 0 ist. Wenn dagegen $s$ nahe $h_i$ ist, ist $\P(s_i > s)$ sehr klein, sodass der Beitrag zum Erwartungswert vernachlässigbar ist. Präzise lässt sich das wie folgt begründen. Zunächst wird $s \le h_{i}^{3/5}$ angenommen und gezeigt, dass dann $\prod_{j = 0}^{(t-1) / \lg 2k_i} (\exp (-j/h_i) - O (j^2/h_i^2)) \sim \prod_{j = 0}^{(t-1) / \lg 2k_i} \exp (-j/h_i)$. Durch Ausmultiplizieren des Produkts erhält man
\begin{align*}
         & \Bigg \vert \prod_{j = 0}^{s} \Bigg ( \exp \bigg ( \frac {-j}{h_i} \bigg ) - O \bigg ( \frac {j^2} {h_i^2} \bigg ) \Bigg ) - \prod_{j = 0}^{s} \exp \bigg ( \frac {-j}{h_i} \bigg ) \Bigg \vert \\
    = \, & \Bigg \vert - \sum_{a = 0}^{s} O \bigg ( \frac {a^2} {h_i^2} \bigg ) \prod_{j = 0, j \ne a}^{s} \exp \bigg ( \frac {-j}{h_i} \bigg ) +
    \sum_{a = 0}^{s} \sum_{b = a + 1}^{s} O \bigg ( \frac {a^2b^2} {h_i^4} \bigg ) \prod_{j = 0, j \ne a, j \ne b}^{s} \exp \bigg ( \frac {-j}{h_i} \bigg ) - \cdots \Bigg \vert
\end{align*}
Die Terme des ausmultiplizierten Produkts werden hier nach der Anzahl an $-O(j^2/h_i^2)$-Faktoren gruppiert. Nach der Dreiecksungleichung erhält man eine obere Schranke, wenn die Vorzeichen der Summen jeweils weggelassen werden. Die Produke von $\exp(-j/h_i)$ sind alle kleiner als 1 und können für eine obere Schranke auch weggelassen werden. Obiger Term wird also durch Folgenden nach oben beschränkt.
\begin{align*}
           & \Bigg \vert \sum_{a = 0}^{s} O \bigg ( \frac {a^2} {h_i^2} \bigg ) +
    \sum_{a = 0}^{s} \sum_{b = a + 1}^{s} O \bigg ( \frac {a^2b^2} {h_i^4} \bigg ) + \sum_{a = 0}^{s} \sum_{b = a + 1}^{s} \sum_{c = b + 1}^{s} O \bigg ( \frac {a^2b^2c^2} {h_i^6} \bigg ) + \cdots \Bigg \vert \\
    \le \, & \Bigg \vert \, s \ O \bigg ( \frac {s^2} {h_i^2} \bigg ) +
    s^2 \, O \bigg ( \frac {s^4} {h_i^4} \bigg ) + s^3 \,  O \bigg ( \frac {s^6} {h_i^6} \bigg ) + \cdots \Bigg \vert                                                                                            \\
    \le \, & \Bigg \vert \, \ O \bigg ( \frac 1 {h_i^{1/5}} \bigg ) + \, O \bigg ( \frac 1 {h_i^{2/5}} \bigg ) +  O \bigg ( \frac 1 {h_i^{3/5}} \bigg ) + \cdots \Bigg \vert
\end{align*}
Letzteres geht klar gegen 0, wenn $h_i \to \infty$. (Eigentlich lassen wir $p \to \infty$ gehen, aber da $h_i = p/(\gcd(p - 1, 2k_i) - 1)$ gilt $h_i = \Theta(p)$ und damit $h_i \to \infty \Longleftrightarrow p \to \infty$.) Im Fall $s < h_i^{3/5}$ können wir also
\begin{align*}
    \P(s_i > s)
     & \sim \prod_{j = 0}^{s} \exp \bigg ( \frac {-j}{h_i} \bigg ) \\
     & =\exp \sum_{j = 0}^{s} \frac {-j} {h_i}                     \\
     & = \exp \frac {-s (s + 1)} {2h_i}                            \\
     & \sim \exp \frac {-s^2} {2h_i}
\end{align*}
schreiben. Letzte Annäherung gilt für große $s$, für ausreichend große $p$ sind aber fast alle $s$ im Erwartungswert von $t_{\min}$ groß.

Nun wird $s > h_i^{3/5}$ angenommen und gezeigt, dass dann der Summand in $\E(t_{\min})$ für $p \to \infty$ gegen 0 geht. Hier gilt
\begin{align*}
    \P(s_i > s)
     & = \prod_{j = 0}^{s} \exp \bigg ( \frac {-j}{h_i} \bigg )                                     \\
     & \le \prod_{j = 0}^{\big \lfloor h_i^{3/5} \big \rfloor} \exp \bigg ( \frac {-j}{h_i} \bigg ) \\
     & \sim \exp \frac {- \Big \lfloor h_i^{3/5} \Big \rfloor^2 } {2h_i}                            \\
     & \le \exp \frac {- h_i^{1/5}} {2}
\end{align*}
Die anderen Wahrscheinlichkeiten in dem entsprechenden Summanden sind alle durch 1 begrenzt und $t$ ist $O(p)$. Da aber $h_i = \Theta(p)$ und $\lim_{p \to \infty} O(p) e^{-\Theta(p)^{1/5}} = 0$, geht der Summand gegen 0. Insgesamt bedeutet das, dass man durch Weglassen der $O(j^2/h_i^2)$-Terme entweder eine asymptotisch genaue Annäherung erhält, oder der Term, in dem man die Annäherung verwendet, sowieso gegen 0 geht. Als Zwischenergebnis erhalten wir
\begin{align}
    \E_{t_{\min}}
     & \sim \sum_{t = 0}^{t_{\max}} t \, \Bigg ( \prod_{i = 1}^M \prod_{j = 0}^{(t-1) / \lg 2k_i} \exp \bigg ( \frac {-j}{h_i} \bigg ) \Bigg ) \ \P_{t_i > t - 1 \, \forall i}(t_i = t \text{ für mindestens ein } i) \nonumber \\
     & = \sum_{t = 0}^{t_{\max}} t \, \exp \Bigg ( \sum_{i = 1}^M \sum_{j = 0}^{(t-1) / \lg 2k_i}  \frac {-j}{h_i} \Bigg ) \ \P_{t_i > t - 1 \, \forall i}(t_i = t \text{ für mindestens ein } i) \nonumber                     \\
     & \sim \sum_{t = 0}^{t_{\max}} t \, \exp \Bigg ( \sum_{i = 1}^M \frac {-t^2} {2 h_i \lg^2 2k_i} \Bigg ) \ \P_{t_i > t - 1 \, \forall i}(t_i = t \text{ für mindestens ein } i) \label{expectation-tmin-intermed}
\end{align}


\paragraph{Die Wahrscheinlichkeit von mindestens einer Kollision bei Zeit $\pmb{t}$.} Hier soll
\begin{align*}
    \P_{t_i > t - 1 \, \forall i}(t_i = t \text{ für mindestens ein } i)
\end{align*}
aus (\ref{expectation-tmin-intermed}) asymptotisch angenähert werden. Die Wahrscheinlichkeit, dass bei Maschine $i$ eine Kollision nach genau $t$ Zeiteinheiten auftritt ist
\begin{align*}
    \P_{t_i > t - 1}(t_i = t) \sim
    \begin{cases}
        t / h_i \lg 2k_i & \quad t = \lceil m \lg 2k_i \rceil, \ m \in \N   \\
        0                & \quad t \ne \lceil m \lg 2k_i \rceil, \ m \in \N
    \end{cases}
\end{align*}
Der erste Fall tritt ein, wenn Zeiteinheit $t$ das Ende einer Iteration von Maschine $i$ enthält. Da bei Zeitpunkt $t$ bereits $t/\lg 2k_i$ Schritte durchgeführt wurden, trifft man im nächsten Schritt mit Wahrscheinlichkeit $t/h_i \lg 2k_i$ auf einen bereits besuchten Knoten. Im zweiten Fall befindet sich Maschine $i$ bei Zeiteinheit $t$ mitten in einer Iteration, es kann also keine Kollision auftreten. Zur Bestimmung von $\E_{t_{\min}}$ kann das durch
\begin{align*}
    \P_{t_i > t - 1}(t_i = t) \sim \frac {t} {h_i \lg^2 k_i}
\end{align*}
angenähert werden. Die Wahrscheinlichkeit einer Kollision an einem Zeitpunkt wird durch den zusätzlichen Faktor $\lg 2k_i$ auf die umliegenden Zeitpunkte "`verteilt"'. Pro Iteration von Maschine $i$ gibt es statt einem Zeitpunkt mit Kollisionswahrscheinlichkeit $t/h_i \lg 2k_i$ nun $\lg 2k_i$ Zeitpunkte mit Kollisionswahrscheinlichkeit $t/h_i \lg^2 2k_i$. Dass das eine asymptotische Näherung ist, lässt sich damit begründen, dass die Terme in (\ref{expectation-tmin-intermed}) stetig in $t$ sind, eine Veränderung von $t$ um maximal $\lg 2k_i$ macht also asymptotisch keinen Unterschied. Die Wahrscheinlichkeit, dass bei mindestens einer Maschine eine Kollision auftritt, kann asymptotisch durch die Summe der Wahrscheinlichkeiten $\P_{t_i > t - 1}(t_i = t)$ angenähert werden.
\begin{align}
    \P_{t_i > t - 1 \, \forall i}(t_i = t \text{ für mindestens ein } i) \sim \sum_{i = 1}^M \frac t {h_i \lg^2 2k_i} = t \sum_{i = 1}^M \frac 1 {h_i \lg^2 2k_i} \label{prob-at-least-one-coll}
\end{align}
Denn für alle relevanten $t$, d.h. $t \le h_i^{3/5} \, \forall i$ ist die Wahrscheinlichkeit, dass zwei oder mehr Maschinen bei Zeit $t$ kollidieren für $p \to \infty$ verschwindend klein.


\paragraph*{Die finale Formel.} \space \space Für $\E(t_{\min})$ gilt mit (\ref{expectation-tmin-intermed}) und (\ref{prob-at-least-one-coll})
\begin{align*}
    \E(t_{\min})
     & \sim \Bigg ( \sum_{i = 1}^M \frac 1 {h_i \lg^2 2k_i} \Bigg ) \sum_{t = 0}^{t_{\max}} t^2 \, \exp \Bigg ( \frac {-t^2} 2 \sum_{i = 1}^M \frac 1 {h_i \lg^2 2k_i} \Bigg )
\end{align*}
Als letzter Schritt wird die Summe über $t$ durch ein Integral angenähert. Die intuitive Erklärung dafür, dass das am asymptotischen Wert nichts ändert ist, dass das Argument der Summe stetig in $t$ ist, sodass der Wert an einem Punkt ähnlich zu Werten nahe dieses Punktes ist. Nach einem ähnlichen Argument wie dem dafür, dass Terme mit $t/\lg 2k_i > h_i^{3/5}$ für ein $i$ asymptotisch irrelevant sind, kann die obere Integralgrenze bis $\infty$ geöffnet werden. Man erhält also
\begin{align}
    L_{k_1, \dots k_M}(p) = \E(t_{\min})
     & \sim \Bigg ( \sum_{i = 1}^M \frac 1 {h_i \lg^2 2k_i} \Bigg ) \int_{0}^{\infty} t^2 \, \exp \Bigg ( \frac {-t^2} 2 \sum_{i = 1}^M \frac 1 {h_i \lg^2 2k_i} \Bigg ) \; dt \nonumber \\
     & = \Bigg (\sum_{i = 1}^M \frac 1 {h_i \lg^2 2k_i} \Bigg ) \, \sqrt {\pi / 2}  \ \Bigg ( \sum_{i = 1}^M \frac 1 {h_i \lg^2 2k_i} \Bigg )^{-3/2} \nonumber                           \\
     & = \sqrt{\pi p / 2} \; \Bigg ( \sum_{i = 1}^M \frac {\gcd(p - 1, 2k_i) - 1} {\lg^2 2k_i} \Bigg )^{-1/2}
    \label{expectation-tmin}
\end{align}
Zur Auswertung des Integrals wurde die Tabelle in Wikipedia: \cite{gint} verwendet. Die Formel ist verträglich mit bereits bekannten Laufzeitabschätzungen für Pollards Rho-Algorithmus. Setzt man beispielsweise $M = 1$ und $k_1 = 1$, erhält man $\sqrt{\pi p / 2}$, wie in \cite{pol75}.

\section{Die erwartete minimale Rho-Länge bei mehreren Anfangswerten}

Um $L_K(p)$ zu bestimmen, wenn $k_i = k_j$ für $i \ne j$ gilt, muss die Abhängigkeit der Rho-Längen der $i$-ten und $j$-ten Maschinen berücksichtigt werden. Denn setzt man beispielsweise $M = 2$ und $k_1 = k_2 = 1$ in (\ref{expectation-tmin}) ein, erhält man eine erwartete Laufzeit von $\sqrt {\pi p / 4}$. In diesem Abschnitt wird allerdings gezeigt, dass unter Berücksichtigung der Abhängigkeit $25/32 \sqrt{\pi p / 2}$ Schritte benötigt werden, und letzterer Wert wird von Experimenten unterstützt. Der Fall abhängiger Maschinen hat sich als weitaus schwieriger herausgestellt und es wurde keine allgemeine Formel gefunden. Jedoch konnte der Fall $M = 2, k_1 = k_2$ gelöst werden. Für den Fall $k_1 = k_2 = \dots = k_M$, sei $k = k_1$ und $h = \gcd(p - 1, 2k)$. Nach Annahme 1 die Verteilung der Rho-Längen von $f : x \mapsto x^{2k} + 1$ für $p \to \infty$ asymptotisch zur Verteilung der Rho-Längen zufälliger Elemente in $A(h)$. Da sich die $M$ Maschinen im Fall $k_1 = k_2 = \dots = k_M$ alle im gleichen funktionalen Graphen bewegen, reduziert sich das Problem der Bestimmung einer asymptotischen Näherung von $L_{k, k, \dots, k}$ auf folgende Frage: \emph{Für $n \in \N$, gegeben ein zufälliges Element $f$ aus $A(n)$ und $M$ zufällige Elemente $x_{i, 0} \in Z/n\Z \; (1 \le i \le M)$, was ist der Erwartungswert von $\min_{i = 1}^M \nu(f, x_{i, 0})$?}

\subsection{Theoretischer Hintergrund: Erzeugende Funktionen}

Der grundlegende Ansatz zur Beantwortung obiger Frage für $M = 2$ ist, die Summe der minimalen Rho-Längen über alle Elemente von $A(p)$ und Paare an Anfangswerten zu bestimmen. Dafür soll eine erzeugende Funktion $\Psi(x, w)$ hergeleitet werden, in der die Variable $x$ die Größe des Graphen und die Variable $w$ die minimale Rho-Länge markiert. Dann gilt nämlich
\begin{align*}
    L_{1, 1}(n) = \frac {n!}{n^{n + 2}} [x^n] \Bigg (\frac {\partial} {\partial w} \Psi(x, w) \Bigg ) \Bigg \vert_{w = 1}
\end{align*}
wobei $[x^n]$ den $n$-ten Koeffizienten in der Reihenentwicklung des nachstehenden Terms bezeichnet. Da funktionale Graphen beschriftet sind, werden stets erzeugende Funktionen von exponentiellem Typ (EF) verwendet. Folgende Komponenten eines funktionalen Graphen werden als Grundlage verwendet, um einen funktionalen Graphen zu konstruieren (\cite{fo90}, S. 333).
\begin{align*}
    T(x) = x \exp T(x) \quad (\text{Baum}) \qquad\qquad C(x) = \ln \frac {1} {1 - x} \quad (\text{Zyklus})
\end{align*}
Ein funktionaler Graph ist eine Menge an Zyklen von Bäumen, also ist die EF für funktionale Graphen
\begin{align*}
    F(x) = \exp C(T(x)) = \exp \ln \frac 1 {1 - T(x)} = \frac 1 {1 - T(x)}
\end{align*}

\noindent Für die Koeffizienten von $((\partial / \partial w) \Psi(x, w)) |_{w = 1}$ wird später eine asymptotische Näherung bestimmt. Dafür wird folgender Satz von \cite{fo90} verwendet, der hier der Vollständigkeit halber erneut formuliert wird.

\begin{theorem}[\cite{fo90}]
    Sei $f(x)$ eine Funktion, die analytisch in
    \begin{align*}
        D = \{x : |x| \le s_1, \arg(x - s) > \pi/2 - \eta \}
    \end{align*}
    ist, für positive reelle Zahlen $s, s_1, \eta$ mit $s_1 > s$. Man nehme an, dass
    \begin{align*}
        f(x) \sim \sigma \bigg ( \frac 1 {1 - x/s} \bigg )
    \end{align*}
    wenn $x \to s$ in $D$, wobei $\sigma(x) = x^\alpha \ln^\beta x$ und $\alpha \notin \{0, -1, -2, \dots\}$. Dann gilt für die Koeffizienten der Taylorreihe von $f$
    \begin{align*}
        [x^n]f(x) \sim s^{-n} \frac {\sigma(n)}{n\Gamma(\alpha)}
    \end{align*}
\end{theorem}

\subsection{Der Fall zweier Anfangswerte}\label{sec:start2}

// Sagen dass wir erst nur den für a und b relevanten Teil des Graphen konstruieren.

\begin{theorem}
    Sei $A(n)$ die Menge der Abbildungen $\Z/n\Z \to \Z/n\Z$. Dann gilt
    \begin{align*}
        \frac 1 {n^{n + 2}} \sum_{f \in A(n)} \; \sum_{a \in \Z/n\Z} \; \sum_{b \in \Z/n\Z} \min(\nu(f, a), \nu(f, b)) \sim \frac {25} {32} \sqrt{\pi n / 2}
    \end{align*}
\end{theorem}

\begin{proof}
    // Maybe bezug von $\Psi$ zu obiger Summe noch einmal klar machen


    Seien $a$ und $b$ die zwei Startknoten und $\Psi(x, w)$ wie bereits definiert. Zur Bestimmung von $\Psi(x, w)$ unterscheiden wir drei disjunkte Fälle, die in Abbildung 1 dargestellt sind. Die Graphen werden grundsätzlich so konstruiert, dass $\nu(f, a) \le \nu(f, b)$, und wenn $\nu(f, a) < \nu(f, b)$, wird mit einem Faktor 2 für das mögliche Vertauschen von $a$ und $b$ multipliziert.
    \\

    \begin{figure}
        \begin{tabular}{ccc}
            \includesvg[width=150pt]{pics/alpha} & \includesvg[width=150pt]{pics/beta} & \includesvg[width=150pt]{pics/gamma} \\
            ($\alpha$)                           & ($\beta$)                           & ($\gamma$)
        \end{tabular}
        \caption{Die drei Fälle für die Bestimmung von $\Psi(x, w)$. Die Kanten stellen keine einzelne Kante dar, sondern einen beliebig langen (und möglicherweise leeren) Pfad. Beispielsweise steht die Kurve $r$ in ($\alpha$) für den Zyklus in dem Zusammenhangskomponenten von $a$.}
    \end{figure}

    \noindent \textbf{Fall 1.} (\emph{$a$ und $b$ liegen in unterschiedlichen Zusammenhangskomponenten.}) \space \space Dieser Fall wird erneut in die Fälle $\lambda(f, b) \le \nu(f, a)$ und $\lambda(f, b) > \nu(f, a)$ unterteilt. Die erzeugende Funktion für den ersten Fall ist
    \begin{align*}
        \alpha_1(x, w) = \frac {x^2w(1 + x^2w)} {(1 - x^2w)^3} \cdot \Bigg (1 + \frac {2x} {1 - x} \Bigg ) = \frac {x^2w(1 + x)(1 + x^2w)} {(1 - x^2w)^3(1 - x)}
    \end{align*}
    In diesem Fall ist es möglich, zuerst zwei $\rho$-Graphen mit gleicher Größe zu erzeugen und anschließend den Pfad von $b$ zu seinem Zyklus zu verlängern. Ein $\rho$-Graph ist ein Zusammenhangskomponent in Abbildung ($\alpha$), d.h. ein Zyklus mit einem Pfad anhängend. Es gibt genau $n! \cdot n$ $\rho$-Graphen mit $n$ Knoten, da es für jede Permutation der Knoten $n$ Möglichkeiten für die Größe des Zyklus gibt. Folglich gibt es für gerade $n$ genau $n! \cdot n^2/2$ Paare an $\rho$-Graphen, die beide $n/2$ Knoten besitzen. Die erzeugende Funktion von Paaren an $\rho$-Graphen gleicher Größe ist also
    \begin{align*}
        \sum_{n = 0}^\infty x^n \cdot \frac {n^2} 4 \cdot \frac {1 + (-1)^n} 2 = \frac {x^2(1 + x^2)} {(1 - x^2)^3}
    \end{align*}
    Um die halbe Anzahl an Knoten mit $w$ zu markieren, ersetze man $x$ durch $x \sqrt w$ und erhält $x^2w(1 + x^2w) / (1 - x^2w)^3$. Das erklärt den ersten Faktor in $\alpha_1(x, w)$. Nun gibt es zwei Möglichkeiten: Wird der Pfad von $b$ zu seinem Zyklus ($u$ in Abbildung 1 ($\alpha$)) nicht verlängert, gilt $\nu(f, a) = \nu(f, b)$, es ergibt sich durch Vertauschen von $a$ und $b$ also keine neue Möglichkeit. Wird hingegen ein Pfad von Länge $\ge 1$ angehängt, dessen erzeugende Funktion $x/(1 - x)$ ist, ergibt sich eine weitere Möglichkeit durch Vertauschen von $a$ und $b$.

    \noindent Im zweiten Fall ist die erzeugende Funktion
    \begin{align*}
        \alpha_2(x, w) = 2 \cdot \frac {x^2w} {(1 - x^2w)^2} \cdot  \frac x {1 - x} \cdot \frac {1}{1 - x} = \frac {2x^3w} {(1 - x^2w)^2 (1 - x)^2}
    \end{align*}
    Der Faktor $x^2w$ repräsentiert die zwei Knoten, an denen $a$ und $b$ jeweils ihren Zyklus betreten, und der Knoten von $a$ ist mit $w$ markiert. Mit $1 / (1 - x^2w)^2$ erhält man vier Pfade $r, y, s, z$, sodass die Länge von $r$ gleich der Länge von $y$ und die Länge von $s$ gleich der Länge von $z$ ist. Die Summe der Längen von $r$ und $s$ wird von $w$ markiert. $r$ und $s$ werden wie in Abbildung 1 ($\alpha$) für den Zusammenhangskomponenten von $a$ verwendet. Damit ist der Exponent von $w$ genau die Rho-Länge von $a$. Der Zyklus von $b$ besteht aus $y, z$ und einem Pfad von Länge $\ge 1$, sodass $\lambda(f, b) > \nu(f, a)$ gilt. Der Term $1 / (1 - x)$ steht für den Pfad von $b$ zum Zyklus.

    \noindent Die erzeugende Funktion für Fall 1 ist also
    \begin{align*}
        \alpha(x, w)
        = \alpha_1(x, w) + \alpha_2(x, w)
        = \frac {x^2w(1 + 2x - x^2 + x^2w - 2x^3w - x^4w)} {(1 - x^2w)^3(1 - x)^2}
    \end{align*}

    \noindent \textbf{Fall 2.} (\emph{$a$ und $b$ liegen im gleichen Baum und ihr kleinster gemeinsamer Vorfahre ist nicht die Wurzel.}) Anders formuliert: Betrachtet man die Pfade, die $a$ und $b$ durch wiederholtes Anwenden von $f$ in $\gamma(f)$ ablaufen, treffen diese sich nicht erstmals in einem Zyklusknoten. Die erzeugende Funktion lautet
    \begin{align*}
        \beta(x, w)
        = xw \cdot \frac {xw} {1 - xw} \cdot \frac {1} {1 - xw} \cdot \frac {1} {1 - x^2w} \cdot \Bigg (1 + \frac {2x} {1 - x} \Bigg )
        = \frac {x^2w^2(1 + x)} {(1 - xw)^2(1 - x^2w)(1 - x)}
    \end{align*}
    Der Zyklusknoten, an dem der Baum von $a$ und $b$ anhängt, wird durch $xw$ repräsentiert. Der Pfad $s$ in Abbildung 1 ($\beta$) muss mindestens Länge 1 haben, da der kleinste gemeinsame Vorfahre von $a$ und $b$ sonst die Wurzel wäre, was den Faktor $xw/(1 - xw)$ erklärt. Der Faktor $1/(1 - xw)$ steht für den Zyklus $r$. Mit $1/(1 - x^2w)$ werden zwei gleich lange Pfade erzeugt, deren Länge durch $w$ markiert wird. Ein Pfad ist $t$ in Abbildung 1 ($\beta$), und der andere ist ein Teil von $u$. Nun gibt es wie in Fall 1 wieder die Option, $u$ echt zu verlängern und so einen Faktor 2 für die mögliche Vertauschung von $a$ und $b$ zu erhalten, oder ihn zu lassen, wobei es wegen Symmetrie nur eine Möglichkeit gibt.
    \\

    \noindent \textbf{Fall 3.} (\emph{$a$ und $b$ liegen in verschiedenen Bäumen oder die Wurzel ist kleinster gemeinsamer Vorfahre.}) Die erzeugende Funktion ist hier
    \begin{align*}
        \gamma(x, w)
        = xw \cdot \frac {1} {(1 - xw)^2} \cdot \frac {1} {1 - x^2w} \cdot \Bigg (1 + \frac {2x}{1 - x} \Bigg )
        = \frac {xw(1 + x)} {(1 - xw)^2(1 - x^2w)(1 - x)}
    \end{align*}
    $xw$ stellt den Zyklusknoten da, an dem der Baum von $a$ anhängt. Der Term $1/(1 - xw)^2$ repräsentiert die beiden Pfade von der Wurzel von $a$ zur Wurzel von $b$ und zurück ($r$ und $s$ in Abbildung 1 ($\gamma$)). Ähnlich wie in Fall 2 ist der Term $1/(1 - x^2w)$ ein Paar an gleich langen Pfaden, deren Länge von $w$ markiert wird. Einer der Pfade ist $u$ in Abbildung 1 ($\gamma$) und der andere ein Teil von $t$. Auch hier kann man die Länge von $t$ Zyklus unverändert lassen, in diesem Fall gibt es eine Möglichkeit, oder einen Pfad von Länge $\ge 1$ hinzufügen, sodass es zwei Möglichkeiten wegen Vertauschung von $a$ und $b$ gibt.

    Ein funktionaler Graph besteht natürlich nicht nur aus einem Zyklus und den Pfaden von $a$ und $b$ zum Zyklus. Von jedem Knoten kann ein Baum ausgehen und es kann noch weitere Zusammenhangskomponenten geben. Durch Erstetzen von $x$ durch $T(x)$ und Hinzufügen einer beliebigen Menge weiterer Zusammenhangskomponenten erhält man also $\Psi(x, w)$.
    \begin{align*}
        \Psi(x, w)
         & = (\alpha(T(x), w) + \beta(T(x), w) + \gamma(T(x), w)) \cdot \exp C(T(x)) \\
         & = \frac {\alpha(T(x), w) + \beta(T(x), w) + \gamma(T(x), w)} {1 - T(x)}
    \end{align*}
    Damit erhalten wir
    \begin{align*}
        \psi(x) = \Bigg (\frac {\partial} {\partial w} \Psi(x, w) \Bigg ) \Bigg \vert_{w = 1} = \frac {T(x)(1 + 2T(x) + 2(T(x))^2)(1 + 5T(x) + 3(T(x))^2 + (T(x))^3)} {(1 - T(x))^6(1 + T(x))^3}
    \end{align*}
    Nun soll die Methode von \cite{fo90} verwendet werden, um eine asymptotische Abschätzung für die Koeffizienten der Reihe von $\psi(x)$ zu erhalten. Nach \cite{fo90}, S. 334, Proposition 1 ist die betragsmäßig (in $\C$) kleinste Singularität von $T(x)$ bei $x = e^{-1}$ und es gilt
    \begin{align*}
        T(x) = 1 - \sqrt{2}\cdot \sqrt {1 - ex} - O(1 - ez)
    \end{align*}
    für $x \to e^{-1}$. $\psi(x)$ hat keine betragsmäßig kleineren Singularitäten, denn wenn $1 - T(x) = 0$, rechnet man leicht nach, dass $x = e^{-1}$ gilt, und wenn $1 + T(x) = 0$, gilt $x = -e$. Es wird nun Satz 1 mit $s = e^{-1}$ verwendet. Wenn $x \to e^{-1}$, gilt
    \begin{align*}
        \psi(x)
         & \sim \frac {T(e^{-1})(1 + 2T(e^{-1}) + 2(T(e^{-1}))^2)(1 + 5T(e^{-1}) + 3(T(e^{-1}))^2 + (T(e^{-1}))^3)} {(1 - (1 - \sqrt 2 \cdot \sqrt {1 - ez}))^6(1 + T(e^{-1}))^3} \\
         & = \frac {(1 + 2 + 2)(1 + 5 + 3 + 1)} {2^3} \cdot \frac 1 {(\sqrt 2 )^6 (\sqrt{1 - ez})^6}                                                                              \\
         & = \frac {50} {64} \cdot \frac 1 {(1 - ez)^3}
    \end{align*}
    Folglich gilt mit der Notation von Satz 1 $\sigma(x) = x^3$ und $\alpha = 3$. Aus Satz 1 folgt
    \begin{align*}
        [x^n] \psi(x)
         & \sim \frac {50} {64} \cdot (e^{-1})^{-n} \cdot \frac {n^3} {n \Gamma(3)} \\
         & = \frac {25} {32} \cdot \frac {e^n n^2} {2}
    \end{align*}
    und mit Stirlings Näherung $n! \sim \sqrt{2\pi n} (n/e)^n$
    \begin{align*}
        \frac 1 {n^{n + 2}} \sum_{f \in A(n)} \;
         & \sum_{a \in \Z/n\Z} \; \sum_{b \in \Z/n\Z} \min(\nu(f, a), \nu(f, b))    \\
         & \sim \frac {n!}{n^{n + 2}} \cdot \frac {25} {32} \cdot \frac {e^n n^2} 2 \\
         & = n! \bigg (\frac {e} {n} \bigg )^n \cdot \frac {25} {64}                \\
         & \sim \sqrt {2 \pi n} \cdot \frac {25}{64}                                \\
         & = \frac {25} {32} \sqrt{\pi n/2}
    \end{align*}
\end{proof}

\noindent Aus der Definition von $L_K(p)$ folgt sofort $L_{k, k}(p) \sim 25/32 \sqrt{\pi p / (2\gcd(p - 1, k))}$. Man bemerke außerdem, dass Satz 2 unabhängig von der Anwendung auf Pollards Rho-Algorithmus formuliert wurde und nicht auf heuristischen Annahmen basiert.

\section{Bestimmung optimaler Exponenten für die Rho-Methode}\label{sec:optk}

In diesem Abschnitt wird die Frage behandelt, wie der Paramter $k$ bei $M$ Maschinen bestmöglich gewählt werden kann. Mit Formel (\ref{lkp}) und Satz 2 konnten Ergebnisse in den Fällen $M = 1$ und $M = 2$ erzielt werden. Die grundlegende Strategie ist, den Erwartungswert von $L_K(p)$ über alle Möglichkeiten von $\gcd(p - 1, 2k_i)$ für alle $1 \le i \le M$ zu bilden und so einen Wert für die erwartete Laufzeit in Abhängigkeit der $k_i$ zu erhalten. Zur Berechnung der Wahrscheinlichkeit, dass $\gcd(p - 1, 2k_i)$ einen bestimmten Wert annimmt, wird folgendes Lemma benötigt.

\begin{lemma}
    Sei $1 \le n \in \N$ und $m$ eine zufällige natürliche Zahl, sodass $0 \le m \le n - 1$. Dann gilt für $d \in \N$
    \begin{align*}
        \mathbb{P}(\gcd(n, m) = d) = \frac {\varphi(n / d)} n
    \end{align*}
    wobei $\varphi$ die eulersche Phifunktion ist. Für $n/d \notin \N$ wird $\varphi(n/d) = 0$ definiert.
\end{lemma}

\begin{proof}
    Jede natürliche Zahl $m$ mit $\gcd(n, m) = d$ lässt sich als $m = da$ schreiben. Dabei muss $\gcd(a, n / d) = 1$ gelten, sonst wäre $\gcd(n, m) > d$. Auch gilt $0 \le a \le n/d - 1$. Umgekehrt gilt $\gcd(n, da) = d$ für jedes $0 \le a \le n/d - 1$ mit $\gcd(n/d, a) = 1$. Die Anzahl an $m \in \N$ mit $0 \le m \le n - 1$ mit $\gcd(n, m) = d$ ist also genau die Anzahl an $a \in \N$ mit $0 \le a \le  n / d - 1$ und $\gcd(n / d, a) = 1$. Diese Zahl ist aber genau $\varphi(n/d)$.
\end{proof}

\subsection{Der Fall einer Maschine}

\begin{theorem}
    Sei $L_k(p)$ wie in Abschnitt \ref{sec:par} definiert. Es gilt $\E(L_1(p)) < \E(L_k(p))$, wobei $1 < k \in \N$ und der Erwartungswert über alle möglichen $\gcd((p - 1)/2, k)$ genommen wird.
\end{theorem}

\begin{proof}
    Durch Einsetzen von $M = 1$ in (\ref{lkp}) erhalten wir
    \begin{align*}
        L_k(p) \sim \sqrt {\pi p / 2} \lg^2 2k \cdot \frac {1} {\sqrt{\gcd(p - 1, 2k) - 1}}
    \end{align*}
    Der Einfachheit halber wird hier $k$ für $k_1$ geschrieben. Die erwartete Laufzeit im Fall $k = 1$ ist folglich $\sqrt{\pi p/2}$, es wird also gezeigt, dass $\E(L_k(p)) > \sqrt{\pi p / 2}$ für $k > 1$. Der Erwartungswert von $L_k(p)$ wird über jede Möglichkeit von $\gcd(p - 1, 2k)$ gebildet. Da $p - 1$ gerade ist, gilt $\gcd(p - 1, 2k) = 2 \gcd((p - 1)/2, k)$. Unter der Annahme, dass jeder Rest von $(p - 1)/2 \bmod k$ gleich wahrscheinlich ist, was für eine Abschätzung der durchschnittlichen Laufzeit sinnvoll ist, gilt $\mathbb{P}(\gcd((p - 1)/2, k) = d) = \varphi(k/d)/k$. Daraus folgt
    \begin{align}
        \E(L_k(p))
         & = \sqrt{\pi p / 2} \lg^2 2k \cdot \sum_{d | k} \frac {\P(\gcd((p - 1)/2, k) = d)} { \sqrt {2d - 1}} \nonumber \\
         & \ge \sqrt{\pi p / 2} \lg^2 2k \cdot \frac {\varphi(k)} k \label{elk}
    \end{align}
    Sei $k = q_1^{e_1} q_2^{e_2} \cdots q_r^{e_r}$ für Primzahlen $q_i$. Folgende Methode, um $\varphi(k)/k$ nach unten zu beschränken, stammt von \cite{phi}.
    \begin{align*}
        \frac {\varphi(k)} k = \prod_{i = 1}^r \left ( \frac {q_i - 1} {q_i} \right )
        = \frac{\prod_{i = 1}^r \left ( 1 - \frac 1 {q_i^2} \right )}{\prod_{i = 1}^r \left ( 1 + \frac 1 {q_i} \right )}
        > \frac {\prod_{n = 2}^\infty \Big ( 1 - \frac 1 {n^2} \Big )} {\prod_{i = 1}^r \left ( 1 + \frac 1 {q_i} \right )}
        = \frac {\prod_{n = 2}^\infty \Big ( \frac {n - 1} {n} \Big ) \Big ( \frac {n + 1} n \Big )} {\prod_{i = 1}^r \left ( 1 + \frac 1 {q_i} \right )}
    \end{align*}
    Der Zähler ist ein Teleskopprodukt und folglich gleich dem ersten Faktor $1/2$. Der Nenner ist kleiner gleich $\sum_{n = 1}^{k} 1/n$ und kann folglich mit $1 + \ln k \le 1 + \lg k = \lg 2k$ nach oben begrenzt werden. Es gilt also
    \begin{align*}
        \frac {\varphi(k)} k
        > \frac {\prod_{n = 2}^\infty \Big ( \frac {n - 1} {n} \Big ) \Big ( \frac {n + 1} n \Big )} {\prod_{i = 1}^r \left ( 1 + \frac 1 {q_i} \right )}
        \ge \frac {1}{2\lg 2k}
    \end{align*}
    Durch Einsetzen in (\ref{elk}) ergibt sich insgesamt
    \begin{align*}
        \E(L_k) > \sqrt{\pi p / 2} \lg^2 2k \cdot \frac {1} {2 \lg 2k} = \sqrt {\pi p / 2} \cdot \frac {\lg 2k} {2}
    \end{align*}
    was für $k \ge 2$ die gewünschte Ungleichung liefert.
\end{proof}

\subsection{Der Fall zweier Maschinen}

Im Fall $M = 2$ konnte bewiesen werden, dass $k_1 = k_2 = 1$ besser ist als $k_1, k_2$ prim und $k_1 = k_2 > 1$. Es wird vermutet, dass $k_1 = k_2 = 1$ optimal ist, der Beweis scheint aber deutlich schwieriger als für $M = 1$. Während es im Fall $M = 1$ möglich war, nur mit dem Fall $\gcd((p - 1)/2, k) = 1$ eine ausreichend große untere Schranke zu erhalten, funktioniert das im Fall $M = 2$ für allgemeine $k_i$ nicht mehr.

\begin{theorem}
    Sei $L_{k_1, k_2}(p)$ wie in Abschnitt \ref{sec:par} definiert. Im Folgenden wird der Erwartungswert von $L_{k_1, k_2}$ über alle möglichen $\gcd((p - 1)/2, k_i)$ für $i = 1, 2$ genommen.
    \begin{enumerate}
        \item Wenn $k_1, k_2$ unterschiedliche Primzahlen sind, gilt $\E(L_{1, 1}(p)) < \E(L_{k_1, k_2}(p))$.
        \item Wenn $1 < k \in \N$, gilt $\E(L_{1, 1}(p)) < \E(L_{k, k}(p))$.
    \end{enumerate}
\end{theorem}

\begin{proof}
    Nach Abschnitt \ref{sec:start2} gilt $L_{1, 1}(p) \sim 25/32 \sqrt{\pi p /2}$, es gilt also in beiden Fällen zu zeigen, dass der Erwartungswert jeweils größer ist. Zunächst wird Teil 1 gezeigt. Da $q_1 \ne q_2$, können wir Gleichung (\ref{lkp}) mit $M = 2$ verwenden.
    \begin{align*}
        L_{k_1, k_2}(p) \sim \sqrt{\pi / 2} \ \Bigg (\frac 1 {h_1 \lg 2k_1} + \frac 1 {h_2 \lg 2k_2} \Bigg ) \Bigg ( \frac 1 {h_1 \lg^2 2k_1} + \frac 1 {h_2 \lg^2 2k_2} \Bigg )^{-3/2}
    \end{align*}
    Wie im Fall $M = 1$ bilden wir den Erwartungswert über alle möglichen $\gcd((p - 1)/2, k_i)$.
    \begin{align*}
        \E(L_{k_1, k_2})
         & \sim \sqrt{\pi / 2} \sum_{d_1 | k_1} \P(\gcd((p - 1)/2, k_2) = d_1) \sum_{d_2 | k_2} \P(\gcd((p - 1)/2, k_2) = d_2)                                                                                              \\
         & \qquad \Bigg (\frac 1 {h_1 \lg 2k_1} + \frac 1 {h_2 \lg 2k_2} \Bigg ) \Bigg ( \frac 1 {h_1 \lg^2 2k_1} + \frac 1 {h_2 \lg^2 2k_2} \Bigg )^{-3/2}                                                                 \\
         & \ge \sqrt{\pi / 2} \ \frac {\varphi(k_1) \varphi(k_2)} {k_1k_2} \Bigg (\frac 1 {p\lg 2k_1} + \frac 1 {p \lg 2k_2} \Bigg ) \Bigg ( \frac 1 {p \lg^2 2k_1} + \frac 1 {p \lg^2 2k_2} \Bigg )^{-3/2}                 \\
         & = \sqrt {\pi p / 2} \  \frac {(k_1 - 1)(k_2 - 1)} {k_1 k_2} \Bigg ( \frac {\lg 2k_2 + \lg 2k_1} {\lg 2k_1 \, \lg 2k_2} \Bigg ) \Bigg ( \frac {\lg^2 2k_2 + \lg^2 2k_1} {\lg^2 2k_1 \, \lg^2 2k_2} \Bigg )^{-3/2} \\
         & = \sqrt {\pi p / 2} \ \frac {(k_1 - 1)(k_2 - 1)(\lg^2 2k_1) (\lg^2 2k_2)} {k_1 k_2} \, \frac {\lg 2k_1 + \lg 2k_2} {(\lg^2 2k_1 + \lg^2 2k_2)^{3/2}}
    \end{align*}
    Nach der Dreiecksungleichung und $k_i \ge 1$ gilt $\lg 2k_1 + \lg 2k_2 > \sqrt{\lg^2 2k_1 + \lg^2 2k_2}$.
    \begin{align*}
        \E(L_{k_1, k_2}(p)) > \sqrt {\pi p / 2} \ \frac {(k_1 - 1)(k_2 - 1)} {k_1 k_2} \, \, \frac { (\lg^2 2k_1) (\lg^2 2k_2)} {\lg^2 2k_1 + \lg^2 2k_2}
    \end{align*}
    Falls $k_1 = 2, k_2 = 3$ gilt damit $E(L_{2, 3}(p)) > 0.8340 \sqrt{\pi p /2}  > 25 / 32 \sqrt{\pi p / 2}$. Die Terme $(k_i - 1)/k_i$ sind aber streng monoton steigend mit $k_i$ für $i = 1, 2$. Ebenso ist der nachfolgende Bruch streng monoton steigend mit jedem der $k_i$. Um das zu zeigen, sei $x = \lg^2 2k_1, y = \lg^2 2k_2$ und $x' > x$. Dann gilt
    \begin{align*}
        \frac {x'y} {x' + y}
        = \frac {x'y} {x' + y} \, \frac {x + y} {xy} \, \frac {xy} {x + y}
        = \frac {xx'y + x'y^2} {xx'y + xy^2} \, \frac {xy} {x + y} > \frac {xy} {x + y}
    \end{align*}
    da $x, x', y > 0$ und $x < x'$. Da $k_1 = 2, k_2 = 3$ der kleinstmögliche Fall ist, folgt für jedes Paar an verschiedenen Primzahlen $k_1, k_2$, dass $\E(L_{k_1, k_2}(p)) > 25 / 32 \sqrt{\pi p / 2}$.

    Für $L_{k, k}(p)$ gilt nach Abschnitt \ref{sec:start2}
\end{proof}

\section{Experimentelle Ergebnisse}\label{sec:ex}

\subsection{Die minimale Rho-Länge bei mehreren Anfangswerten}

\subsection{Experimentelle Bestimmung optimaler Parameter}

\section{Fazit}

\printbibliography

\end{document}