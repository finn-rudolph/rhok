\documentclass[a4paper, 10pt, ngerman]{article}

\usepackage[algoruled, nosemicolon]{algorithm2e}
\usepackage{amsmath}
\usepackage{amssymb}
\usepackage{amsthm}
\usepackage[ngerman]{babel}
\usepackage[backend=biber, style=apa]{biblatex}
\usepackage[left=2.5cm, right=2.5cm, top=2.5cm, bottom=2.cm]{geometry}
\usepackage[hidelinks]{hyperref}
\usepackage{mathtools}
\usepackage[onehalfspacing]{setspace}
\usepackage{sectsty}
\usepackage[inkscapeformat=pdf]{svg}

\allsectionsfont{\sffamily}

\title{\sffamily{\textbf{Parametrisierung von Pollards Rho-Methode}}}
\author{Finn Rudolph}
\date{27.01.2024}

\addbibresource{rhok.bib}

\renewcommand{\thealgocf}{}

\newcommand{\C}{\mathbb{C}}
\newcommand{\E}{\mathbb{E}}
\newcommand{\N}{\mathbb{N}}
\newcommand{\R}{\mathbb{R}}
\newcommand{\Z}{\mathbb{Z}}

\newtheorem{definition}{Definition}
\newtheorem{theorem}{Satz}
\newtheorem{lemma}{Lemma}

\begin{document}

\maketitle

\section*{Projektüberblick}

\tableofcontents

\section{Zusammenfassung}

\section{Motivation und Fragestellung}

Während für Rekordfaktorisierungen mittlerweile ausschließlich das Zahlkörpersieb verwendet wird, bleibt Pollards Rho-Algorithmus einer schnellsten Algorithmen zur Faktorisierung von Zahlen bis ca. $2^{64}$. Da Leistungssteigerungen bei modernen Computern häufig durch verbesserete Nebenläufigkeit (z. B. durch mehr Prozessorkerne) erzielt werden, ist es eine interessante Frage, wie Pollards Rho-Methode am besten parallel ausgeführt werden kann. Bevor die Fragestellung präzise formuliert werden kann, soll jedoch Pollards Rho-Methode erklärt werden.

\subsection{Pollards Rho-Methode}

Sei $n$ die zu faktorisierende Zahl und $f : \Z/n\Z \to \Z/n\Z$ mit $f : x \mapsto x^{2k} + 1$ für einen Parameter $1 \le k \in \N$. Man wähle einen zufälligen Anfangswert $x_0 \in \Z/n\Z$ und betrachte die Folge $(x_n)_{n \in \N}$ definiert durch $x_n = f(x_{n - 1})$ für einen Parameter $1 \le k \in \N$. Da $(x_n)_{n \in \N}$ über der endlichen Menge $\Z/n\Z$ definiert ist, ist die Folge ab einem bestimmten Punkt periodisch. Sei $p$ ein Primfaktor von $n$ und $\pi : \Z/n\Z \to \Z/p\Z$ die natürliche Projektion. Die Idee von Pollards Rho-Algorithmus ist, zwei Folgenglieder $x_i, x_j \in \Z/n\Z$ zu finden, sodass $x_i \ne x_j$ aber $\pi(x_i) = \pi(x_j)$. Dann ist nämlich $\gcd(n, x_i - x_j)$ ein echter Faktor von $n$. Nimmt man nun heuristisch an, dass die Periodenlänge von $(x_n)_{n \in \N}$ in $\Z/n\Z$ deutlich länger als die Periodenlänge in $\Z/p\Z$ ist, reicht es aus, $x_i, x_j$ mit $i \ne j$ zu finden, die kongruent modulo $p$ sind. Diese Annahme begründet sich darin, dass für den kleinsten Primfaktor $p \le \sqrt n$ gilt. Zum Finden solcher $x_i, x_j$ betrachten wir den funktionalen Graphen von $\pi(f)$, wobei $\pi(f)$ die Abbildung $f$ betrachtet in $\Z/p\Z$ ist.

\begin{definition}[Funktionaler Graph]
    Sei $X$ eine endliche Menge und $f: X \to X$ eine Abbildung. Der funktionale Graph von $f$, geschrieben $\gamma(f)$, ist der gerichtete Graph mit Knotenmenge $X$ und Kantenmenge $E$, wobei die Kante $(x, y) \in X \times X$ genau dann in $E$ liegt, wenn $f(x) = y$.
\end{definition}

\noindent Es ist leicht zu zeigen, dass jede Zusammenhangskomponente eines funktionalen Graphen aus einem Zyklus und an den Zyklusknoten gewurzelten Bäumen besteht. In $\gamma(f)$ betrachtet startet $(x_n)_{n \in \N}$ mit $x_0$ in einem Baum und "`läuft"' durch den Graphen, wobei immer die eindeutige von einem Knoten ausgehende Kante entlanggegangen wird. An der Wurzel des Baums von $x_0$ wird der Zyklus in der Zusammenhangskomponente von $x_0$ betreten, und ab genau diesem Punkt ist $(x_n)_{n \in \N}$ periodisch. Um $x_i, x_j$ mit $i \ne j$ und $\pi(x_i) = \pi(x_j)$ zu finden, wird Floyds Algorithmus zum Finden des Zyklus in der Zusammenhangskomponente von $\pi(x_0)$ in $\gamma(\pi(f))$ verwendet. Floyds Algorithmus macht sich zunutze, dass es ein $r \in \N, 1 \le r \ge 1$ mit $\pi(x_r) = \pi(x_{2r})$ geben muss. Für das minimale solcher $r$ gilt außerdem $r \le \mu(f, x_0) + \lambda(f, x_0)$, wobei $\mu(f, x_0)$ die Höhe von $x_0$ in seinem Baum und $\lambda(f, x_0)$ die Länge des Zyklus ist (\cite{knu98}, S. 7). Wir nennen $\nu(f, x_0) = \mu(f, x_0) + \lambda(f, x_0)$ die Rho-Länge von $x_0$ in $f$. Indem man die Folgen $(x_n)_{n \in \N}$ und $(x_{2n})_{n \in \N}$ gleichzeitig Glied für Glied berechnet, stößt man in maximal $\nu(f, x_0)$ Schritten auf gewünschte $x_i, x_j$. Das Überprüfen in jedem Schritt, ob $\pi(x_i) = \pi(x_{2i})$ geschieht natürlich nicht explizit, da $p$ unbekannt ist, aber implizit durch Berechnung von $\gcd(n, x_i - x_{2i})$.

\begin{algorithm*}
    $x \gets $ zufällige natürliche Zahl zwischen $0$ und $n - 1$ \;
    $y \gets x$ \;
    \While{\emph{\sc{true}}}
    {
        $x \gets x^{2k} + 1 \mod n$ \;
        $y \gets (y^{2k} + 1)^{2k} + 1 \mod n$ \;
        $g \gets \gcd(n, x - y)$ \;
        \If{$g \ne 1 \text{\emph{\textbf{ and }}} g \ne n$}
        {
            \Return{$g$} \;
        }
    }

    \caption{Pollards Rho-Algorithmus}
\end{algorithm*}

\noindent Nun soll Pollards Rho-Algorithmus analysiert werden. Sei $\Z/p\Z^*$ die Gruppe der Einheiten in $\Z/p\Z$ und $g$ = $\gcd(p - 1, 2k)$. Man beobachte, dass $\psi : \Z/p\Z^\times \to \Z/p\Z^\times$ mit $\psi : x \mapsto x^{2k}$ ein Gruppenenhomorphismus mit $[\Z/p\Z : \text{Im} f] = g$ ist. Daher haben genau $(p - 1) / g$ Knoten in $\gamma(f)$ Eingangsgrad $g$ und alle anderen Eingangsgrad 0. Knoten 1 als Einzelfall mit Eingangsgrad 1 ignorieren wir. Sei $A(n)$ die Menge der Abbildungen $\Z/n\Z \to \Z/n\Z$ und $A(n, g)$ die Teilmenge von $A(n)$, in der $(p - 1) / g$ Knoten Eingangsgrad $g$ und alle anderen Eingangsgrad 0 haben. Der Analyse der Rho-Methode liegen folgende Annhamen zugrunde:
\begin{itemize}
    \item[(A1)] Die Abbildung $f : x \mapsto x^{2k} + 1$ verhält sich bezüglich der Verteilung der Rho-Längen wie eine zufällig gewählte Abbildung aus $A(p, g)$.
    \item[(A2)] Die Abbildung $f : x \mapsto x^{2} + 1$ verhält sich bezüglich der Verteilung der Rho-Längen wie eine zufällig gewählte Abbildung aus $A(p)$.
    \item[(A3)] Die durchschnittliche Anzahl an Iterationen der while-Schleife ist $\overline \nu$, bis auf eine feste Konstante, wobei $\overline \nu$ der Mittelwert von $\nu(f, x_0)$ über alle Elemente von $A(p)$ und Startwerte $x_0 \in \Z/p\Z$ ist.
\end{itemize}
Diese Annahmen werden auch von \cite{pol75} und \cite{bp81} getroffen. Eine rigorose Analyse von Pollards Rho-Algorithmus ist nicht bekannt, die Annahmen lassen sich allerdings experimentell bestätigen. Für $k = 1$ ist damit die erwartete Anzahl an Iterationen der while-Schleife $\sqrt{\pi p / 2}$ (\cite{knu98}, S. 8). Da die Berechnung des größten gemeinsamen Teilers $O(\ln n)$ Schritte benötigt, ist die erwartete Laufzeit des Algorithmus $O(\sqrt p \ln n)$. Durch eine einfache Modifikation können die Kosten des $\gcd$ amortisiert werden, sodass sich die Laufzeit auf $O(\sqrt p)$ verringert (\cite{bre80}). Damit ist pro Iteration also nur noch die Zeit zur Berechnung der $2k$-ten Potenzen von $x$ und $y$ relevant, was durchschnittlich in $3/2 \lg 2k$ Schritten möglich ist. Mit $\lg x$ wird der Logarithmus zur Basis 2 bezeichnet. Der Faktor $3/2$ wird im Folgenden ignoriert, da er für den Vergleich verschiedener Werte für $k$ irrelevant ist.

\subsection{Parallelisierung der Rho-Methode}

Sei $M$ die Anzahl verfügbarer Maschinen. Eine "`Maschine"'  meint hier nicht zwingend einen Computer, sondern eine Ressource, auf der ein sequentielles Programm ausgeführt werden kann, was beispielsweise auch ein Prozessorthread sein kann. Die Rho-Methode lässt sich parallelisieren, indem $M$ Anfangswerte unabhängig voneinander zufällig gewählt werden und auf jeder der $M$ Maschinen Pollards Rho-Algorithmus ausgeführt wird, bis eine der Maschinen einen Faktor findet. Nun ergibt sich folgende Frage, die in dieser Arbeit behandelt werden soll: \emph{Wie wählt man den Parameter $k$ für jede Maschine optimal, um eine möglichst geringe Laufzeit zu erzielen?} Eine Zeiteinheit ist hier die Dauer einer Iteration im Fall $k = 1$, d.h. eine Maschine mit Paramter $k$ führt in $t$ Zeiteinheiten $\lfloor t / \lg 2k \rfloor$ Iterationen aus. Da es sich bei den Veränderungen in der Laufzeit durch Veränderung von $k$ um konstante Faktoren handelt, wird für den Vergleich wird die Laufzeit nicht $O$-Notation verwendet, sondern eine asymptotische Näherung für die erwartete Zahl an Zeiteinheiten bestimmt.

\begin{definition}[Asymptotische Näherung]
    Eine Funktion $f : \R \to \R$ heißt genau dann asymptotische Näherung von einer Funktion $g : \R \to \R$, oder asymptotisch zu $g$, wenn
    \begin{align*}
        \lim_{x \to \infty} \frac {f(x)} {g(x)} = 1
    \end{align*}
    In diesem Fall schreiben wir $f \sim g$.
\end{definition}

\noindent Im Gegensatz zur $O$-Notation kann zwischen zwei Funktionen, die asymptotisch zueinander sind, für große $n$ kein konstanter Faktor liegen, sodass sich Veränderungen um konstante Faktoren sinnvoll vergleichen lassen.

Eine Zuordnung von $k$-Werten für $M$ Maschinen wird als Tupel $K = (k_1, k_2, \dots, k_M), 1 \le k_i \in \N$ geschrieben.  Wir bezeichnen mit $L_K(p)$ die erwartete Laufzeit des parallelen Pollard-Rho-Algorithmus mit $k$-Werten gegeben durch $K$. Mit (A1) und (A3) gilt also
\begin{align*}
    L_K(p) = \E \bigg ( \min_{i = 1}^M X_i \bigg )
\end{align*}
wobei $X_i$ die gleichverteilte Zufallsvariable über $A(p, \gcd(p - 1, 2k_i)) \times \Z/p\Z$ mit $X_i(f, x_0) = \nu(f, x_0)$ ist. Eine Schwierigkeit in der Herleitung einer Formel für $L_K(p)$ ist, dass die Rho-Längen der $i$-ten und $j$-ten Maschine nicht unabhängig sind, wenn $k_i = k_j$, da sich die $i$-te und $j$-te Maschine in diesem Fall im gleichen funktionalen Graphen bewegen. Wenn beispielsweise der Startwert der $i$-ten Maschine fixiert wird, ist bereits klar, dass die Rho-Länge der $j$-ten Maschine größer gleich der Rho-Länge der $i$-ten Maschine sein wird, wenn der Startwert ein Vorfahre des Startwerts der $i$-ten Maschine in einem Baum von $\gamma(f)$ ist. Daher werden im Folgenden die zwei Fälle abhängiger und unabhängiger Maschinen unterschieden.

\section{Eine Formel für den Fall unabhängiger Maschinen}

In diesem Abschnitt wird eine Formel für $L_K(p)$ hergleitet, die im Fall paarweise verschiedener $k$-Werte gilt. Für den Fall mehrerer Maschinen wird die Annahme getroffen, dass die Rho-Längen der $i$-ten und $j$-ten Maschine als Zufallsvariablen betrachtet stochastisch unabhängig sind, wenn $k_i \ne k_j$. Denn wenn $k_i \ne k_j$ und sich $x \mapsto x^{2k_i} + 1$ und $x \mapsto x^{2k_j} + 1$ bezüglich der Rho-Länge wie zufällig gewählte Abbildung verhalten, ist es vertretbar, die Unabhängigkeit der Rho-Längen anzunehmen.

Folgendes Argument basiert auf dem von \cite{bp81}, S. 627 -- 628, angepasst für mehrere Maschinen. Es wird die die Wahrscheinlichkeit betracht, dass nach genau $t$ Zeiteinheiten erstmalig eine Kollision bei einer Maschine auftritt. Die Wahrscheinlichkeit, dass im $i$-ten Schritt zu einem Vorgänger eines bereits besuchten Knotens gegangen wird (sodass im nächsten Schritt eine Kollision stattfände), ist $i \cdot (\gcd(p - 1, 2k) - 1)/(p - 1)$. Folglich ist die Wahrscheinlichkeit, dass bei einer Maschine mit Paramter $k$ nach $t$ Zeiteinheiten noch keine Kollision aufgetreten ist
\begin{align*}
    \prod_{j = 0}^{\lfloor t / \lg 2k \rfloor} \bigg (1 - \frac {j} {p / (\gcd(p - 1, 2k) - 1)} \bigg )
\end{align*}
Wir lassen im Folgenden die Gaußklammern weg, da sie asymptotisch nichts verändern und verwenden Taylors Näherung $1 + x \approx \exp x$ für $|x|$ klein. Sei $h_i = p / (\gcd(p - 1, 2k_i) - 1)$. Da in allen praktischen Fällen $h_i$ groß und $t$ deutlich kleiner als $p$ ist, ist die Wahrscheinlichkeit, dass nach genau $t$ Zeiteinheiten die erste Kollision auftritt
\begin{align*}
    \mathbb{P}_t
     & = \prod_{i = 1}^M \prod_{j = 0}^{t / \lg k_i - 1}
    \Bigg (1 - \frac {j} {h_i} \Bigg ) \cdot \Bigg (1 - \prod_{i = 1}^M \Bigg (1 - \frac {t / \lg 2k_i} {h_i} \Bigg ) \Bigg )                                                            \\
     & \approx \prod_{i = 1}^M \prod_{j = 0}^{t / \lg k_i - 1} \exp \Bigg ( \frac {-j}{h_i} \Bigg ) \cdot \Bigg (1 - \prod_{i = 1}^M \exp \Bigg (\frac {-t}{h_i \lg 2k_i} \Bigg )\Bigg ) \\
     & = \exp \Bigg ({- \sum_{i = 1}^M \sum_{j = 0}^{t/\lg 2k_i - 1} \frac j {h_i}} \Bigg ) \cdot \Bigg (1 - \exp \Bigg ( -\sum_{i = 1}^M \frac t {h_i \lg 2k_i} \Bigg ) \Bigg )         \\
     & \approx \exp \Bigg (- \sum_{i = 1}^M \frac {(t/\lg 2k_i)(t/\lg 2k_i - 1)} {2h_i} \Bigg ) \cdot t \cdot \Bigg (\sum_{i = 1}^M \frac 1 {h_i \lg 2k_i} \Bigg )                       \\
     & \approx \exp \Bigg (- \frac {t^2} 2 \sum_{i = 1}^M \frac 1 {h_i \lg^2 2k_i} \Bigg ) \cdot t \cdot \Bigg (\sum_{i = 1}^M \frac 1 {h_i \lg 2k_i} \Bigg )
\end{align*}
Um nun eine Formel für den Erwartungswert zu erhalten, wird $\sum_{a = 1}^p t \mathbb{P}_t$ durch $\int_{0}^\infty t \mathbb{P}_t \; dt$ angenähert, was wegen der schnellen Konvergenz von $e^{-t^2} \to 0$ wenn $t \to \infty$ vertetbar ist.
\begin{align}
    L_K(p)
     & = \Bigg ( \sum_{i = 1}^M \frac 1 {h_i \lg 2k_i} \Bigg ) \int_{0}^{\infty} t^2 \exp \Bigg (- \frac {t^2} 2 \sum_{i = 1}^M \frac 1 {h_i \lg^2 2k_i} \Bigg ) \nonumber \\
     & = \Bigg ( \sum_{i = 1}^M \frac 1 {h_i \lg 2k_i} \Bigg ) \sqrt {\pi/2} \Bigg (\sum_{i = 1}^M \frac 1 {h_i \lg^2 2k_i} \Bigg )^{-3/2}
\end{align}
Zur Auswertung des Integrals wurde die Tabelle in Wikipedia: \cite{gint} verwendet. Auch wenn die Herleitung aufgrund der vielen Näherungen krude scheint, erklärt sie gut das Laufzeitverhalten im Fall unabhängiger Maschinen. Setzt man beispielsweise $M = 1$ und $k_1 = 1$, erhält man $\sqrt{\pi p / 2}$, wie in \cite{pol75}. Für $M = 1$ und beliebiges $k_1$ erhält man $\sqrt{\pi p / 2} \cdot \lg^2 2k_1 / \sqrt{\gcd(p-1, 2k_1) - 1}$. Daraus folgt, dass die Anzahl an Iterationen im Fall einer Maschine um einen Faktor $1/\sqrt{\gcd(p - 1, 2k) - 1}$ im Vergleich zu $k = 1$ reduziert wird. Dieses Ergebnis erhielten auch \cite{bp81}.

\section{Die erwartete minimale Rho-Länge bei mehreren Anfangswerten}

Um $L_K(p)$ zu bestimmen, wenn $k_i = k_j$ für $i \ne j$ gilt, muss die Abhängigkeit der Rho-Längen der $i$-ten und $j$-ten Maschinen berücksichtigt werden. Denn setzt man beispielsweise $M = 2$ und $k_1 = k_2 = 1$ in (1) ein, erhält man eine erwartete Laufzeit von $\sqrt {\pi p / 4}$. In diesem Abschnitt wird allerdings gezeigt, dass unter Berücksichtigung der Abhängigkeit $25/32 \cdot \sqrt{\pi p / 2}$ Schritte benötigt werden, und letzterer Wert wird von Experimenten unterstützt. Der Fall abhängiger Maschinen hat sich als weitaus schwieriger herausgestellt und es wurde keine allgemeine Formel gefunden. Jedoch konnte der Fall $M = 2, k_1 = k_2$ gelöst werden und einige Ideen für den Fall von beliebigem $M$ und $k_1 = k_2 = \dots = k_M$ entwickelt werden. Für letzteren Fall genügt es, den Fall $k_i = 1 \; (1 \le i \le M)$ zu betrachten. Denn wie bereits angemerkt wird für eine Maschine die durchschnittliche Rho-Länge durch Veränderung von $k$ um einen Faktor $1 / \sqrt {\gcd(p - 1, 2k) - 1}$ verringert, es gilt also $L_{k, k, \dots, k}(p) = L_{1, 1, \dots, 1}(p) / \sqrt{\gcd(p - 1, 2k)-1}$. Mit $k_i = 1 \; (1 \le i \le M)$ reduziert sich das Problem nach (A2) auf folgende Frage: \emph{Gegeben ein zufälliges Element $f$ aus $A(p)$ und $M$ zufällige Elemente $x_{i, 0} \in Z/p\Z \; (1 \le i \le M)$, was ist der Erwartungswert von $\min_{i = 1}^M \nu(f, x_{i, 0})$?}

\subsection{Theoretischer Hintergrund: Erzeugende Funktionen}

Der grundlegende Ansatz zur Beantwortung obiger Frage für $M = 2$ ist, die Summe der minimalen Rho-Längen über alle Elemente von $A(p)$ und Paare an Anfangswerten zu bestimmen. Dafür soll eine erzeugende Funktion $\Psi(x, w)$ hergeleitet werden, in der die Variable $x$ die Größe des Graphen und die Variable $w$ die minimale Rho-Länge markiert. Dann gilt nämlich
\begin{align*}
    L_{1, 1}(n) = \frac {n!}{n^{n + 2}} [x^n] \Bigg (\frac {\partial} {\partial w} \Psi(x, w) \Bigg ) \Bigg \vert_{w = 1}
\end{align*}
wobei $[x^n]$ den $n$-ten Koeffizienten in der Reihenentwicklung des nachstehenden Terms bezeichnet.

Da funktionale Graphen beschriftet sind, werden stets erzeugende Funktionen von exponentiellem Typ (EF) verwendet. Folgende Komponenten eines funktionalen Graphen werden als Grundlage verwendet, um einen funktionalen Graphen zu konstruieren (\cite{fo90}, S. 333).
\begin{align*}
    T(x) = x \exp T(x) \quad (\text{Baum}) \qquad\qquad C(x) = \ln \frac {1} {1 - x} \quad (\text{Zyklus})
\end{align*}
Ein funktionaler Graph ist eine Menge an Zyklen von Bäumen, also ist die EF für funktionale Graphen
\begin{align*}
    F(x) = \exp C(T(x)) = \exp \ln \frac 1 {1 - T(x)} = \frac 1 {1 - T(x)}
\end{align*}

Für die Koeffizienten von $((\partial / \partial w) \Psi(x, w)) |_{w = 1}$ wird später eine asymptotische Näherung bestimmt. Dafür wird folgender Satz von \cite{fo90} verwendet, der hier der Vollständigkeit halber erneut formuliert wird.

\begin{theorem}[\cite{fo90}]
    Sei $f(x)$ eine Funktion, die analytisch in
    \begin{align*}
        D = \{x : |x| \le s_1, \arg(x - s) > \pi/2 - \eta \}
    \end{align*}
    ist, für positive reelle Zahlen $s, s_1, \eta$ mit $s_1 > s$. Man nehme an, dass
    \begin{align*}
        f(x) \sim \sigma \bigg ( \frac 1 {1 - x/s} \bigg )
    \end{align*}
    wenn $x \to s$ in $D$, wobei $\sigma(x) = x^\alpha \ln^\beta x$ und $\alpha \notin \{0, -1, -2, \dots\}$. Dann gilt für die Koeffizienten der Taylorreihe von $f$
    \begin{align*}
        [x^n]f(x) \sim s^{-n} \frac {\sigma(n)}{n\Gamma(\alpha)}
    \end{align*}
\end{theorem}

\subsection{Der Fall zweier Anfangswerte}

\begin{theorem}
    Sei $A(n)$ die Menge der Abbildungen $\Z/n\Z \to \Z/n\Z$. Dann gilt
    \begin{align*}
        \frac 1 {n^{n + 2}} \sum_{f \in A(n)} \; \sum_{a \in \Z/n\Z} \; \sum_{b \in \Z/n\Z} \min(\nu(f, a), \nu(f, b)) \sim \frac {25} {32} \sqrt{\pi n / 2}
    \end{align*}
\end{theorem}

\begin{proof}
    Seien $a$ und $b$ die zwei Startknoten und $\Psi(x, w)$ wie bereits definiert. Zur Bestimmung von $\Psi(x, w)$ unterscheiden wir drei disjunkte Fälle, die in Abbildung 1 dargestellt sind. Die Graphen werden grundsätzlich so konstruiert, dass $\nu(f, a) \le \nu(f, b)$, und wenn $\nu(f, a) < \nu(f, b)$, wird mit einem Faktor 2 für das mögliche Vertauschen von $a$ und $b$ multipliziert.

    \begin{figure}
        \begin{tabular}{ccc}
            \includesvg[width=150pt]{pics/alpha} & \includesvg[width=150pt]{pics/beta} & \includesvg[width=150pt]{pics/gamma} \\
            ($\alpha$)                           & ($\beta$)                           & ($\gamma$)
        \end{tabular}
        \caption{Die drei Fälle für die Bestimmung von $\Psi(x, w)$. Die Kanten stellen keine einzelne Kante dar, sondern einen beliebig langen (und möglicherweise leeren) Pfad. Beispielsweise steht die Kurve $r$ in ($\alpha$) für den Zyklus in dem Zusammenhangskomponenten von $a$.}
    \end{figure}

    \textbf{Fall 1.} (\emph{$a$ und $b$ liegen in unterschiedlichen Zusammenhangskomponenten.}) Dieser Fall wird erneut in die Fälle $\lambda(f, b) \le \nu(f, a)$ und $\lambda(f, b) > \nu(f, a)$ unterteilt. Die erzeugende Funktion für den ersten Fall ist
    \begin{align*}
        \alpha_1(x, w) = \frac {x^2w(1 + x^2w)} {(1 - x^2w)^3} \cdot \Bigg (1 + \frac {2x} {1 - x} \Bigg ) = \frac {x^2w(1 + x)(1 + x^2w)} {(1 - x^2w)^3(1 - x)}
    \end{align*}
    In diesem Fall ist es möglich, zuerst zwei $\rho$-Graphen mit gleicher Größe zu erzeugen und anschließend den Pfad von $b$ zu seinem Zyklus zu verlängern. Ein $\rho$-Graph ist ein Zusammenhangskomponent in Abbildung ($\alpha$), d.h. ein Zyklus mit einem Pfad anhängend. Es gibt genau $n! \cdot n$ $\rho$-Graphen mit $n$ Knoten, da es für jede Permutation der Knoten $n$ Möglichkeiten für die Größe des Zyklus gibt. Folglich gibt es für gerade $n$ genau $n! \cdot n^2/2$ Paare an $\rho$-Graphen, die beide $n/2$ Knoten besitzen. Die erzeugende Funktion von Paaren an $\rho$-Graphen gleicher Größe ist also
    \begin{align*}
        \sum_{n = 0}^\infty x^n \cdot \frac {n^2} 4 \cdot \frac {1 + (-1)^n} 2 = \frac {x^2(1 + x^2)} {(1 - x^2)^3}
    \end{align*}
    Um die halbe Anzahl an Knoten mit $w$ zu markieren, ersetze man $x$ durch $x \sqrt w$ und erhält $x^2w(1 + x^2w) / (1 - x^2w)^3$. Das erklärt den ersten Faktor in $\alpha_1(x, w)$. Nun gibt es zwei Möglichkeiten: Wird der Pfad von $b$ zu seinem Zyklus ($u$ in Abbildung 1 ($\alpha$)) nicht verlängert, gilt $\nu(f, a) = \nu(f, b)$, es ergibt sich durch Vertauschen von $a$ und $b$ also keine neue Möglichkeit. Wird hingegen ein Pfad von Länge $\ge 1$ angehängt, dessen erzeugende Funktion $x/(1 - x)$ ist, ergibt sich eine weitere Möglichkeit durch Vertauschen von $a$ und $b$.

    Im zweiten Fall ist die erzeugende Funktion
    \begin{align*}
        \alpha_2(x, w) = 2 \cdot \frac {x^2w} {(1 - x^2w)^2} \cdot  \frac x {1 - x} \cdot \frac {1}{1 - x} = \frac {2x^3w} {(1 - x^2w)^2 (1 - x)^2}
    \end{align*}
    Der Faktor $x^2w$ repräsentiert die zwei Knoten, an denen $a$ und $b$ jeweils ihren Zyklus betreten, und der Knoten von $a$ ist mit $w$ markiert. Mit $1 / (1 - x^2w)^2$ erhält man vier Pfade $r, y, s, z$, sodass die Länge von $r$ gleich der Länge von $y$ und die Länge von $s$ gleich der Länge von $z$ ist. Die Summe der Längen von $r$ und $s$ wird von $w$ markiert. $r$ und $s$ werden wie in Abbildung 1 ($\alpha$) für den Zusammenhangskomponenten von $a$ verwendet. Damit ist der Exponent von $w$ genau die Rho-Länge von $a$. Der Zyklus von $b$ besteht aus $y, z$ und einem Pfad von Länge $\ge 1$, sodass $\lambda(f, b) > \nu(f, a)$ gilt. Der Term $1 / (1 - x)$ steht für den Pfad von $b$ zum Zyklus.

    Die erzeugende Funktion für Fall 1 ist also
    \begin{align*}
        \alpha(x, w)
        = \alpha_1(x, w) + \alpha_2(x, w)
        = \frac {x^2w(1 + 2x - x^2 + x^2w - 2x^3w - x^4w)} {(1 - x^2w)^3(1 - x)^2}
    \end{align*}

    \textbf{Fall 2.} (\emph{$a$ und $b$ liegen im gleichen Baum und ihr kleinster gemeinsamer Vorfahre ist nicht die Wurzel.}) Anders formuliert: Betrachtet man die Pfade, die $a$ und $b$ durch wiederholtes Anwenden von $f$ in $\gamma(f)$ ablaufen, treffen diese sich nicht erstmals in einem Zyklusknoten. Die erzeugende Funktion lautet
    \begin{align*}
        \beta(x, w)
        = xw \cdot \frac {xw} {1 - xw} \cdot \frac {1} {1 - xw} \cdot \frac {1} {1 - x^2w} \cdot \Bigg (1 + \frac {2x} {1 - x} \Bigg )
        = \frac {x^2w^2(1 + x)} {(1 - xw)^2(1 - x^2w)(1 - x)}
    \end{align*}
    Der Zyklusknoten, an dem der Baum von $a$ und $b$ anhängt, wird durch $xw$ repräsentiert. Der Pfad $s$ in Abbildung 1 ($\beta$) muss mindestens Länge 1 haben, da der kleinste gemeinsame Vorfahre von $a$ und $b$ sonst die Wurzel wäre, was den Faktor $xw/(1 - xw)$ erklärt. Der Faktor $1/(1 - xw)$ steht für den Zyklus $r$. Mit $1/(1 - x^2w)$ werden zwei gleich lange Pfade erzeugt, deren Länge durch $w$ markiert wird. Ein Pfad ist $t$ in Abbildung 1 ($\beta$), und der andere ist ein Teil von $u$. Nun gibt es wie in Fall 1 wieder die Option, $u$ echt zu verlängern und so einen Faktor 2 für die mögliche Vertauschung von $a$ und $b$ zu erhalten, oder ihn zu lassen, wobei es wegen Symmetrie nur eine Möglichkeit gibt.

    \textbf{Fall 3.} (\emph{$a$ und $b$ liegen in verschiedenen Bäumen oder die Wurzel ist kleinster gemeinsamer Vorfahre.}) Die erzeugende Funktion ist hier
    \begin{align*}
        \gamma(x, w)
        = xw \cdot \frac {1} {(1 - xw)^2} \cdot \frac {1} {1 - x^2w} \cdot \Bigg (1 + \frac {2x}{1 - x} \Bigg )
        = \frac {xw(1 + x)} {(1 - xw)^2(1 - x^2w)(1 - x)}
    \end{align*}
    $xw$ stellt den Zyklusknoten da, an dem der Baum von $a$ anhängt. Der Term $1/(1 - xw)^2$ repräsentiert die beiden Pfade von der Wurzel von $a$ zur Wurzel von $b$ und zurück ($r$ und $s$ in Abbildung 1 ($\gamma$)). Ähnlich wie in Fall 2 ist der Term $1/(1 - x^2w)$ ein Paar an gleich langen Pfaden, deren Länge von $w$ markiert wird. Einer der Pfade ist $u$ in Abbildung 1 ($\gamma$) und der andere ein Teil von $t$. Auch hier kann man die Länge von $t$ Zyklus unverändert lassen, in diesem Fall gibt es eine Möglichkeit, oder einen Pfad von Länge $\ge 1$ hinzufügen, sodass es zwei Möglichkeiten wegen Vertauschung von $a$ und $b$ gibt.

    Ein funktionaler Graph besteht natürlich nicht nur aus einem Zyklus und den Pfaden von $a$ und $b$ zum Zyklus. Von jedem Knoten kann ein Baum ausgehen und es kann noch weitere Zusammenhangskomponenten geben. Durch Erstetzen von $x$ durch $T(x)$ und Hinzufügen einer beliebigen Menge weiterer Zusammenhangskomponenten erhält man also $\Psi(x, w)$.
    \begin{align*}
        \Psi(x, w)
         & = (\alpha(T(x), w) + \beta(T(x), w) + \gamma(T(x), w)) \cdot \exp C(T(x)) \\
         & = \frac {\alpha(T(x), w) + \beta(T(x), w) + \gamma(T(x), w)} {1 - T(x)}
    \end{align*}
    Damit erhalten wir
    \begin{align*}
        \psi(x) = \Bigg (\frac {\partial} {\partial w} \Psi(x, w) \Bigg ) \Bigg \vert_{w = 1} = \frac {T(x)(1 + 2T(x) + 2(T(x))^2)(1 + 5T(x) + 3(T(x))^2 + (T(x))^3)} {(1 - T(x))^6(1 + T(x))^3}
    \end{align*}
    Nun soll die Methode von \cite{fo90} verwendet werden, um eine asymptotische Abschätzung für die Koeffizienten der Reihe von $\psi(x)$ zu erhalten. Nach \cite{fo90}, S. 334, Proposition 1 ist die betragsmäßig (in $\C$) kleinste Singularität von $T(x)$ bei $x = e^{-1}$ und es gilt
    \begin{align*}
        T(x) = 1 - \sqrt{2}\cdot \sqrt {1 - ex} - O(1 - ez)
    \end{align*}
    für $x \to e^{-1}$. $\psi(x)$ hat keine betragsmäßig kleineren Singularitäten, denn wenn $1 - T(x) = 0$, rechnet man leicht nach, dass $x = e^{-1}$ gilt, und wenn $1 + T(x) = 0$, gilt $x = -e$. Es wird nun Satz 1 mit $s = e^{-1}$ verwendet. Wenn $x \to e^{-1}$, gilt
    \begin{align*}
        \psi(x)
         & \sim \frac {T(e^{-1})(1 + 2T(e^{-1}) + 2(T(e^{-1}))^2)(1 + 5T(e^{-1}) + 3(T(e^{-1}))^2 + (T(e^{-1}))^3)} {(1 - (1 - \sqrt 2 \cdot \sqrt {1 - ez}))^6(1 + T(e^{-1}))^3} \\
         & = \frac {(1 + 2 + 2)(1 + 5 + 3 + 1)} {2^3} \cdot \frac 1 {(\sqrt 2 )^6 (\sqrt{1 - ez})^6}                                                                              \\
         & = \frac {50} {64} \cdot \frac 1 {(1 - ez)^3}
    \end{align*}
    Folglich gilt mit der Notation von Satz 1 $\sigma(x) = x^3$ und $\alpha = 3$. Aus Satz 1 folgt
    \begin{align*}
        [x^n] \psi(x)
         & \sim \frac {50} {64} \cdot (e^{-1})^{-n} \cdot \frac {n^3} {n \Gamma(3)} \\
         & = \frac {25} {32} \cdot \frac {e^n n^2} {2}
    \end{align*}
    und mit Stirlings Näherung $n! \sim \sqrt{2\pi n} (n/e)^n$
    \begin{align*}
        \frac 1 {n^{n + 2}} \sum_{f \in A(n)} \;
         & \sum_{a \in \Z/n\Z} \; \sum_{b \in \Z/n\Z} \min(\nu(f, a), \nu(f, b))    \\
         & \sim \frac {n!}{n^{n + 2}} \cdot \frac {25} {32} \cdot \frac {e^n n^2} 2 \\
         & = n! \bigg (\frac {e} {n} \bigg )^n \cdot \frac {25} {64}                \\
         & \sim \sqrt {2 \pi n} \cdot \frac {25}{64}                                \\
         & = \frac {25} {32} \sqrt{\pi n/2}
    \end{align*}
\end{proof}

\noindent Mit obigen Annahmen über $L_K(p)$ folgt aus Satz 2 sofort $L_{1, 1}(p) \sim 25/32 \sqrt{2 \pi / 2}$. Man bemerke außerdem, dass Satz 2 unabhängig von der Anwendung auf Pollards Rho-Algorithmus formuliert wurde und nicht auf heuristischen Annahmen basiert.

\subsection{Mögliche Methoden für den allgemeinen Fall}

\section{Bestimmung optimaler Exponenten für die Rho-Methode}

In diesem Abschnitt wird die Frage behandelt, wie der Paramter $k$ bei $M$ Maschinen bestmöglich gewählt wird. Mit Formel (1) und Satz 2 konnten Ergebnisse in den Fällen $M = 1$ und $M = 2$ erzielt werden. Es wird noch folgendes Lemma benöigt.

\begin{lemma}
    Sei $1 \le n \in \N$ und $m$ eine zufällige natürliche Zahl, sodass $0 \le m \le n - 1$. Dann gilt für $d \in \N$
    \begin{align*}
        \mathbb{P}(\gcd(n, m) = d) = \frac {\varphi(n / d)} n
    \end{align*}
    wobei $\varphi$ die eulersche Phifunktion ist. Für $n/d \notin \N$ wird $\varphi(n/d) = 0$ definiert.
\end{lemma}

\begin{proof}
    Jede natürliche Zahl $m$ mit $\gcd(n, m) = d$ lässt sich als $m = da$ schreiben. Dabei muss $\gcd(a, n / d) = 1$ gelten, sonst wäre $\gcd(n, m) > d$. Auch gilt $0 \le a \le n/d - 1$. Umgekehrt gilt $\gcd(n, da) = d$ für jedes $0 \le a \le n/d - 1$ mit $\gcd(n/d, a) = 1$. Die Anzahl an $m \in \N$ mit $0 \le m \le n - 1$ mit $\gcd(n, m) = d$ ist also genau die Anzahl an $a \in \N$ mit $0 \le a \le  n / d - 1$ und $\gcd(n / d, a) = 1$. Diese Zahl ist aber genau $\varphi(n/d)$.
\end{proof}

\subsection{Der Fall einer Maschine}

\begin{theorem}
    Für Pollards Rho-Algorithmus auf einer Maschine ist $k = 1$ optimal.
\end{theorem}

\begin{proof}
    Durch Einsetzen von $M = 1$ in (1) erhalten wir
    \begin{align*}
        L_k(p) = \sqrt {\pi p / 2} \lg^2 2k \cdot \frac {1} {\sqrt{\gcd(p - 1, 2k) - 1}}
    \end{align*}
    Der Einfachheit halber wird hier $k$ für $k_1$ geschrieben. Die erwartete Laufzeit im Fall $k = 1$ ist folglich $\sqrt{\pi p/2}$, es wird also gezeigt, dass $\E(L_k(p)) > \sqrt{\pi p / 2}$ für $k > 1$. Der Erwartungswert von $L_k(p)$ wird über jede Möglichkeit von $\gcd(p - 1, 2k)$ gebildet. Da $p - 1$ gerade ist, gilt $\gcd(p - 1, 2k) = 2 \gcd((p - 1)/2, k)$. Sei $Y$ die Zufallsvariable von $\gcd((p - 1)/2, k)$, also $\mathbb{P}(Y = d) = \mathbb{P}(\gcd((p - 1)/2, k) = d)$. Unter der Annahme, dass jeder Rest von $(p - 1)/2 \bmod k$ gleich wahrscheinlich ist, was für eine Abschätzung der durchschnittlichen Laufzeit sinnvoll ist, ist die Verteilung von $Y$ durch Lemma 1 mit $n = k$ gegeben, d.h. $\mathbb{P}(Y = d) = \varphi(k/d)/k$. Nach dem Fundamentalsatz der Arithmetik ist es möglich
    \begin{align*}
        k = q_1^{e_1} \cdot q_2^{e_2} \cdots q_r^{e_r}
    \end{align*}
    für Primzahlen $q_1, q_2, \dots, q_r$ zu schreiben. Man definiere Zufallsvariablen $Y_1, Y_2, \dots, Y_r$ mit
    \begin{align*}
        \mathbb{P}(Y_i = d) = \mathbb{P}(\gcd(q_i^{e_i}, m) = d) \qquad 1 \le i \le r
    \end{align*}
    für eine zufällige natürliche Zahl $0 \le m \le q_i^{e_i} - 1$. Aus dem chinesischen Restsatz folgt
    \begin{align*}
        Y = \prod_{i = 1}^r Y_i
    \end{align*}
    Da die $q_i^{e_i}$ teilerfremd sind, sind $Y_1, Y_2, \dots, Y_r$ stochastisch unabhängig und folglich gilt
    \begin{align*}
        \E(L_k)
         & = \E \left ( \sqrt{\pi p / 2} \lg^2 2k \cdot \frac {1} {\sqrt{2Y - 1}} \right )                     \\
         & = \sqrt{\pi p / 2} \lg^2 2k \cdot \E \left ( \frac 1 {\sqrt{2Y - 1}} \right )                       \\
         & \ge \sqrt{\pi p / 4} \lg^2 2k \cdot \E \left ( \frac 1 {\sqrt{Y}} \right )                          \\
         & = \sqrt{\pi p / 4} \lg^2 2k \cdot \prod_{i = 1}^r \E \left ( \frac 1 {\sqrt {Y_i}} \right ) \tag{2}
    \end{align*}
    Wir wollen nun $\prod_{i = 1}^r \E(1/\sqrt{Y_i})$ bestimmen. Es gilt $\mathbb{P}(Y_i = d) = 0$, wenn $d$ keine Potenz von $q_i$ ist. Andernfalls gilt nach Lemma 1
    \begin{equation*}
        \mathbb{P}(Y_i = q_i^{h})
        = \frac {\varphi(q_i^{e_i} / q_i^h)}{q_i^{e_i}}
        = \frac{\varphi(q_i^{e_i - h})}{q_i^{e_i}}
        = \begin{dcases*}
            \frac {q_i - 1} {q_i^{h + 1}} & \quad $h < e_i$ \vspace{2mm} \\
            \frac {1} {q_i^{e_i}}         & \quad $h = e_i$              \\
        \end{dcases*}
    \end{equation*}
    Daraus folgt
    \begin{align*}
        \prod_{i = 1}^r \E \left ( \frac 1 {\sqrt {Y_i}} \right )
         & = \prod_{i = 1}^r \sum_{j = 0}^{e_i} \mathbb{P}(Y_i = q_i^j) \frac {1} {\sqrt{q_i^j}}                                                                                                                                                                                      \\
         & = \prod_{i = 1}^r \left ( \frac {q_i - 1} {q_i} \cdot \frac {1} {\sqrt{1}} + \frac{q_i - 1} {q_i^2} \cdot \frac 1 {\sqrt{q_i}} + \dots + \frac {q_i - 1} {q_i^{e_i}} \cdot \frac 1 {\sqrt{q_i^{e_i- 1}}} + \frac {1} {q_i^{e_i}} \cdot \frac 1 {\sqrt{q_i^{e_i}}} \right ) \\
         & \ge \prod_{i = 1}^r \left ( \frac {q_i - 1} {q_i} \right )
    \end{align*}
    Folgende Methode, um das Produkt auszuwerten, stammt von \cite{phi}.
    \begin{align*}
        \prod_{i = 1}^r \left ( \frac {q_i - 1} {q_i} \right )
        = \frac{\prod_{i = 1}^r \left ( 1 - \frac 1 {q_i^2} \right )}{\prod_{i = 1}^r \left ( 1 + \frac 1 {q_i} \right )}
        > \frac {\prod_{n = 2}^\infty \Big ( 1 - \frac 1 {n^2} \Big )} {\prod_{i = 1}^r \left ( 1 + \frac 1 {q_i} \right )}
        = \frac {\prod_{n = 2}^\infty \Big ( \frac {n - 1} {n} \Big ) \Big ( \frac {n + 1} n \Big )} {\prod_{i = 1}^r \left ( 1 + \frac 1 {q_i} \right )}
    \end{align*}
    Der Zähler ist ein Teleskopprodukt und folglich gleich dem ersten Faktor $1/2$. Sei $k' = \prod_{i = 1}^r q_i$. Der Nenner ist kleiner gleich $\sum_{n = 1}^{k'} 1/n$ und kann folglich mit $1 + \ln k' \le 1 + \ln k \le 1 + \lg k = \lg 2k$ nach oben begrenzt werden. Es gilt also
    \begin{align*}
        \frac {\prod_{n = 2}^\infty \Big ( \frac {n - 1} {n} \Big ) \Big ( \frac {n + 1} n \Big )} {\prod_{i = 1}^r \left ( 1 + \frac 1 {q_i} \right )} \ge \frac {1}{2\lg 2k}
    \end{align*}
    Durch Einsetzen in (2) ergibt sich insgesamt
    \begin{align*}
        \E(L_k) > \sqrt{\pi p / 4} \lg^2 2k \frac {1} {2 \lg 2k} = \sqrt {\pi p / 2} \cdot \frac {\lg 2k} {2 \sqrt 2}
    \end{align*}
    was für $k \ge 4$ die gewünschte Ungleichung liefert.
\end{proof}

\subsection{Der Fall zweier Maschinen}

// + unter der Wurzel rauswerfen (den Summanden von kleinerem $k$ droppen)

\section{Experimentelle Ergebnisse}\label{sec:ex}

\subsection{Die minimale Rho-Länge bei mehreren Anfangswerten}

\subsection{Vergleich verschiedener Werte für den Parameter K}

\section{Fazit}

// TODO: Gleichungsnummer rausnehmen, wenn keine weiteren

\printbibliography

\end{document}