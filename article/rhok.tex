\documentclass[a4paper, 10pt, ngerman]{article}

\usepackage[algoruled, nosemicolon]{algorithm2e}
\usepackage{amsmath}
\usepackage{amssymb}
\usepackage{amsthm}
\usepackage[ngerman]{babel}
\usepackage[backend=biber, style=apa]{biblatex}
\usepackage[font=small]{caption}
\usepackage[left=3cm, right=3cm, top=2.5cm, bottom=2.5cm]{geometry}
\usepackage[hidelinks]{hyperref}
\usepackage{mathtools}
\usepackage{pgfplots}
\usepackage[onehalfspacing]{setspace}
\usepackage{sectsty}
\usepackage[inkscapeformat=pdf]{svg}
\usepackage{xcolor}

\title{Parametrisierung von Pollards Rho-Methode}
\author{Finn Rudolph}
\date{27.01.2024}

\addbibresource{rhok.bib}

\renewcommand{\thealgocf}{}

\newcommand{\C}{\mathbb{C}}
\newcommand{\E}{\mathbb{E}}
\newcommand{\N}{\mathbb{N}}
\newcommand{\R}{\mathbb{R}}
\newcommand{\Z}{\mathbb{Z}}
\renewcommand{\P}{\mathbb{P}}

\newtheorem{definition}{Definition}
\newtheorem{theorem}{Satz}
\newtheorem{lemma}{Lemma}
\newtheorem*{assumption*}{Annahme}

\newcommand*\uparagraph{%
        \par
        \nopagebreak
        \vskip3.25ex plus1ex minus.2ex
        \noindent
}

\sectionfont{\large}
\subsectionfont{\normalsize}

\begin{document}

\noindent\rule{\textwidth}{0.4pt}

\makeatletter
\begin{flushleft}
    \textbf{\LARGE{\@title}} \\
    \vspace{1.5em}
    Finn Rudolph \\
    \vspace{1em}
    Erarbeitungsort: Ferdinandstraße 42, 53127 Bonn \\
    Fachgebiet: Mathmatik / Informatik \\
    Wettbewerbssparte: Jugend forscht \\
    Bundesland: Bayern \\
    Wettbewerbsjahr: 2024
\end{flushleft}

\vspace{0.5em}

\section*{Projektüberblick}

\tableofcontents

\section{Zusammenfassung}

\section{Motivation und Fragestellung}

Während für Rekordfaktorisierungen mittlerweile ausschließlich das Zahlkörpersieb verwendet wird, bleibt Pollards Rho-Algorithmus einer schnellsten Algorithmen zur Faktorisierung von kleineren Zahlen. Da Leistungssteigerungen bei modernen Computern häufig durch verbesserete Nebenläufigkeit (z. B. durch mehr Prozessorkerne) erzielt werden, ist es eine interessante Frage, wie Pollards Rho-Methode am besten parallel ausgeführt werden kann. Bevor die Fragestellung präzise formuliert werden kann, soll jedoch Pollards Rho-Methode erklärt werden.

\subsection{Pollards Rho-Methode}

Sei $n$ die zu faktorisierende Zahl. Es wird angenommen, dass $n$ ungerade und keine Potenz einer natürlichen Zahl ist, da sonst einfach ein Faktor gefunden werden kann. Sei $f : \Z/n\Z \to \Z/n\Z$ mit $f : x \mapsto x^{2k} + 1$ für einen Parameter $1 \le k \in \N$. Man wähle einen zufälligen Anfangswert $x_0 \in \Z/n\Z$ und betrachte die Folge $(x_n)_{n \in \N}$ definiert durch $x_n = f(x_{n - 1})$ für einen Parameter $1 \le k \in \N$. Da $(x_n)_{n \in \N}$ über der endlichen Menge $\Z/n\Z$ definiert ist, ist die Folge ab einem bestimmten Punkt periodisch. Sei $p$ ein Primfaktor von $n$ und $\pi : \Z/n\Z \to \Z/p\Z$ die natürliche Projektion. Die Idee von Pollards Rho-Algorithmus ist, zwei Folgenglieder $x_i, x_j \in \Z/n\Z$ zu finden, sodass $x_i \ne x_j$ aber $\pi(x_i) = \pi(x_j)$. Dann ist nämlich $\gcd(n, x_i - x_j)$ ein echter Faktor von $n$. Das Ereignis, dass $\pi(x_i) = \pi(x_j)$, wird \emph{Kollision} genannt. Nimmt man an, dass die Periodenlänge von $(x_n)_{n \in \N}$ in $\Z/n\Z$ deutlich länger als die Periodenlänge in $\Z/p\Z$ ist, reicht es aus, $x_i, x_j$ mit $i \ne j$ zu finden, die kongruent modulo $p$ sind. Die Annahme ist plausibel, weil für den kleinsten Primfaktor $p \le \sqrt n$ gilt, es also deutlich weniger mögliche Werte für $\pi(x_i)$ als $x_i$ gibt. Im Folgenden ist $p$ immer der kleinster Primfaktor von $n$. Außerdem wird angenommen, dass die anderen Primfaktoren von $n$ so viel größer als $p$ sind, dass die Wahrscheinlichkeit einer Kollision modulo eines anderen Primfaktors vernachlässigbar gering ist. Zum Finden solcher $x_i, x_j$ ist es hilfreich, den funktionalen Graphen von $f$ zu betrachten.

\begin{definition}[Funktionaler Graph]
    Sei $X$ eine endliche Menge und $f: X \to X$ eine Abbildung. Der funktionale Graph von $f$, geschrieben $\gamma(f)$, ist der gerichtete Graph mit Knotenmenge $X$ und Kantenmenge $E$, wobei die Kante $(x, y) \in X \times X$ genau dann in $E$ liegt, wenn $f(x) = y$.
\end{definition}

\noindent Es ist leicht zu zeigen, dass jede Zusammenhangskomponente eines funktionalen Graphen aus einem Zyklus und an den Zyklusknoten gewurzelten Bäumen besteht. In $\gamma(f)$ betrachtet startet $(x_n)_{n \in \N}$ mit $x_0$ in einem Baum und "`läuft"' durch den Graphen, wobei immer die eindeutige von einem Knoten ausgehende Kante entlanggegangen wird. An der Wurzel des Baums von $x_0$ wird der Zyklus in der Zusammenhangskomponente von $x_0$ betreten, und ab genau diesem Punkt ist $(x_n)_{n \in \N}$ periodisch. Sei $\pi(f)$ die Abbildung $f$, betrachtet in $\Z/p\Z$. Um $x_i, x_j$ mit $i \ne j$ und $\pi(x_i) = \pi(x_j)$ zu finden, wird Floyds Algorithmus zum Finden des Zyklus in der Zusammenhangskomponente von $\pi(x_0)$ in $\gamma(\pi(f))$ verwendet. Floyds Algorithmus macht sich zunutze, dass es ein $r \in \N, 1 \le r \ge 1$ mit $\pi(x_r) = \pi(x_{2r})$ geben muss. Für das minimale solcher $r$ gilt außerdem $r \le \mu(f, x_0) + \lambda(f, x_0)$, wobei $\mu(f, x_0)$ die Höhe von $x_0$ in seinem Baum und $\lambda(f, x_0)$ die Länge des Zyklus ist (\cite{knu98}, S. 7). Wir nennen $\nu(f, x_0) = \mu(f, x_0) + \lambda(f, x_0)$ die Rho-Länge von $x_0$ in $f$. Indem man die Folgen $(x_n)_{n \in \N}$ und $(x_{2n})_{n \in \N}$ gleichzeitig Glied für Glied berechnet, stößt man in maximal $\nu(f, x_0)$ Schritten auf gewünschte $x_i, x_j$. Das Überprüfen in jedem Schritt, ob $\pi(x_i) = \pi(x_{2i})$ geschieht natürlich nicht explizit, da $p$ unbekannt ist, aber implizit durch Berechnung von $\gcd(n, x_i - x_{2i})$.

\begin{algorithm*}
    $x \gets $ zufällige natürliche Zahl zwischen $0$ und $n - 1$ \;
    $y \gets x$ \;
    \While{\emph{\textsc{true}}}
    {
        $x \gets x^{2k} + 1 \mod n$ \;
        $y \gets (y^{2k} + 1)^{2k} + 1 \mod n$ \;
        $g \gets \gcd(n, x - y)$ \;
        \If{$g \ne 1 \text{\emph{\textbf{ and }}} g \ne n$}
        {
            \Return{$g$} \;
        }
    }

    \caption{Pollards Rho-Algorithmus}
\end{algorithm*}

\noindent Die Analyse von Pollards Rho-Algorithmus erweist sich als schwierig, es ist bis dato keine rigorose Laufzeitanalyse bekannt. Unter heuristischen Annahmen lässt sich die Laufzeit allerdings sehr gut abschätzen. Anstatt der mittleren Anzahl an Iterationen von Floyds Algorithmus wird die mittlere Rho-Länge analysiert, mit der Annahme, dass sich diese zwei Werte für große $p$ nur um eine feste Konstante unterscheiden. Eine zentrale Annahme dreht sich um die Verteilung der Rho-Längen in $\gamma(f)$, für ihre Formulierung wird der Begriff einer asymptotischen Näherung benötigt.

\begin{definition}[Asymptotische Näherung]
    Eine Funktion $f : \R \to \R$ heißt genau dann asymptotische Näherung von einer Funktion $g : \R \to \R$, oder asymptotisch zu $g$, wenn
    \begin{align*}
        \lim_{x \to \infty} \frac {f(x)} {g(x)} = 1
    \end{align*}
    In diesem Fall schreiben wir $f \sim g$.
\end{definition}

\noindent Sei $A(n)$ die Menge der Abbildungen $\Z/n\Z \to \Z/n\Z$ für $n \in \N$. Über die Verteilung der Rho-Längen wird folgende Annahme getroffen, die auch \emph{Random Mapping Assumption} (RMA) genannt wird.

\begin{assumption*}[Random Mapping Assumption]
    Sei $f : x \mapsto x^{2k} + 1$ und $d = \gcd(p - 1, 2k)$. Seien $x_0 \in \Z/p\Z$ und $y_0 \in \Z/((p - 1)/d)\Z$ zufällig und $g \in A((p - 1)/d)$ zufällig. Dann gilt $\P(\nu(f, x_0) = m) \sim \P(\nu(g, y_0) = m)$ für $p \to \infty$.
\end{assumption*}

\noindent In anderen Worten sagt die Random Mapping Assumption, dass sich die Verteilung der Rho-Längen von $f : x \mapsto x^{2k} + 1$ wie bei einer zufälligen Funktion aus $A((p - 1)/d)$ verhält. Insbesondere verhält sich $x \mapsto x^2 + 1$ bezüglich der Rho-Längen wie ein zufällige Funktion $\Z/(p - 1)\Z \to \Z/(p - 1)\Z$. \cite{bp81} geben eine Begründung für RMA. Im Folgenden wird statt $(p - 1)/d$ einfach $p/d$ verwendet, da $p \sim p - 1$ für $p \to \infty$.

Für $k = 1$ ist damit die erwartete Anzahl an Iterationen der while-Schleife asymptotisch zu $\sqrt{\pi p / 2}$ (\cite{knu98}, S. 8). Da die Berechnung des größten gemeinsamen Teilers $O(\ln n)$ Schritte benötigt, ist die erwartete Laufzeit des Algorithmus $O(\sqrt p \ln n)$. Durch eine einfache Modifikation können die Kosten des $\gcd$ amortisiert werden, sodass sich die Laufzeit auf $O(\sqrt p)$ verringert (\cite{bre80}). Damit ist pro Iteration also nur noch die Zeit zur Berechnung der $2k$-ten Potenzen von $x$ und $y$ relevant, was durchschnittlich in $c \lg 2k$ Schritten möglich ist, wobei $c$ eine hier unwichtige Konstante ist. Mit $\lg x$ wird der Logarithmus zur Basis 2 bezeichnet.

\subsection{Parallelisierung der Rho-Methode}

Sei $M$ die Anzahl verfügbarer Maschinen. Eine "`Maschine"'  meint hier nicht zwingend einen Computer, sondern eine Ressource, auf der ein sequentielles Programm ausgeführt werden kann, was beispielsweise auch ein Prozessorthread sein kann. Die Rho-Methode lässt sich parallelisieren, indem $M$ Anfangswerte unabhängig voneinander zufällig gewählt werden und auf jeder der $M$ Maschinen Pollards Rho-Algorithmus ausgeführt wird, bis eine der Maschinen einen Faktor findet. Nun ergibt sich folgende Frage, die in dieser Arbeit behandelt werden soll: \emph{Wie wählt man den Parameter $k$ für jede Maschine optimal, um eine möglichst geringe Laufzeit zu erzielen?} Das ist nicht sofort klar, da durch ein größeres $k$ möglicherweise $\gcd(p - 1, 2k)$ groß ist, sodass die Zahl an Iterationen um einen Faktor $\sqrt{\gcd(p - 1, 2k) -1}$ sinkt. Allerdings steigt die Dauer einer Iteration um einen Faktor $\lg 2k$. Da es sich bei den Veränderungen in der Laufzeit durch Veränderung von $k$ um konstante Faktoren handelt, wird für den Vergleich der Laufzeit nicht $O$-Notation verwendet, sondern eine asymptotische Näherung für die erwartete Zahl an Zeiteinheiten bestimmt, wenn $p \to \infty$. Wir definieren eine \emph{Zeiteinheit} als die Dauer einer Iteration für $k = 1$, bei einer Maschine mit Parameter $k$ dauert eine Iteration also $\lg 2k$ Zeiteinheiten. Im Gegensatz zur $O$-Notation kann zwischen zwei Funktionen, die asymptotisch zueinander sind, für große $n$ kein konstanter Faktor liegen, sodass sich Veränderungen um konstante Faktoren sinnvoll vergleichen lassen.

Sei $k_1, k_2, \dots, k_M, 1 \le k_i \in \N$ eine Zuordnung von $k$-Werten für $M$ Maschinen. Mit $L_{k_1, \dots, k_M}(p)$ wird die erwartete Laufzeit des parallelen Pollard-Rho-Algorithmus mit entsprechenden $k$-Werten bezeichnet. Dabei ist $\gcd(p - 1, 2k_i)$ für alle $1 \le i \le M$ noch fixiert, der Erwartungswert über alle Möglichkeiten von $\gcd(p - 1, 2k_i)$ wird erst in Abschnitt \ref{sec:optimal-k} behandelt. Sei $h_i = p/(\gcd(p - 1, 2k_i) - 1)$. Mit RMA gilt
\begin{align}
    L_{k_1, \dots, k_M}(p) = \E \bigg ( \min_{i = 1}^M X_i \bigg )
    \label{lk-definition}
\end{align}
wobei $X_i$ die gleichverteilte Zufallsvariable über $A(h_i) \times \Z/h_i\Z$ ist, mit $X_i(f, x_0) = \nu(f, x_0)$ für $f \in A(h_i), x_0 \in \Z/h_i\Z$. Eine Schwierigkeit in der Herleitung einer Formel für $L_{k_1, \dots, k_M}(p)$ ist, dass $X_i$ und $X_j$ nicht unabhängig sind, wenn $k_i = k_j$. Grund dafür ist, dass $f$ nicht unabhängig gewählt wird, denn die $i$-te und $j$-te Maschine bewegen sich in diesem Fall im gleichen funktionalen Graphen.  Daher werden im Folgenden die Fälle abhängiger und unabhängiger Maschinen unterschieden.

\section{Eine Formel für den Fall unabhängiger Maschinen}

In diesem Abschnitt wird eine Formel für $L_{k_1, \dots, k_M}(p)$ hergleitet, die im Fall paarweise verschiedener $k$-Werte gilt. Die Rho-Längen verschiedener Maschinen sind hier stochastisch unabhängig, was aus RMA folgt. Sei $h_i = p/(\gcd(p - 1, 2k_i) - 1)$ und $s_i$ die Anzahl an Iterationen, nach denen bei der $i$-ten Maschine erstmals eine Kollision auftritt. Sei $t_i = s_i \lg 2k_i$ die Zeit, nach der bei Maschine $i$ erstmals eine Kollision auftritt und $t_{\min} = \min_{i = 1}^M t_i$. Das Ziel ist die Bestimmung von $\E(t_{\min})$.

\paragraph{Eine Formel für $\pmb{\P(s_i > s)}$.} Nach RMA gilt $\P(s_i = s) = \P(\nu(f, x_0) = s)$ für eine zufällig gewählte Funktion $f \in A(h_i)$ und ein zufälliges $x_0 \in \Z/h_i\Z$. Für eine zufällige Funktion $f \in A(h_i)$ ist die Wahrscheinlichkeit einer Kollision im $i$-ten Schritt $i/h_i$, wenn in den ersten $i-1$ Schritten keine Kollision aufgetreten ist, da jeder der $h_i$ möglichen Werte gleich wahrscheinlich ist und $i$ von ihnen zu einer Kollision führen. Die Wahrscheinlichkeit, dass in den ersten $s$ Schritten keine Kollision auftritt ist also
\begin{align*}
    \P(s_i > s)
    = \P(\nu(f, x_0) > s)
    = \prod_{j = 0}^s \bigg (1 - \frac {j} {h_i} \bigg )
\end{align*}
Da die $i$-te Maschine in $t$ Zeiteinheiten $\lfloor t / \lg 2k_i \rfloor$ Schritte ausführt, gilt $\P(t_i > t) \sim \P(s_i > t / \lg 2k_i)$ für große $t$. Das Weglassen der Gaußklammern lässt sich damit rechtfertigen, dass für ausreichend großes $p$ nur große $t$ relevant für den Erwartungswert von $t_{\min}$ sind, bei denen es kaum einen Unterschied macht. Zur Herleitung der Formel werden noch weitere solcher Vereinfachungen nötig sein, diese zu beweisen ist aber meist uninteressant und aufwändig. Im Abschnitt \emph{Weglassen von $O(j^2/h_i^2)$} wird für eine von ihnen genauer begründet, um zu zeigen, wie ein solches Argument aussehen kann. Nun wird die Restgliedabschätzung $\exp x = 1 + x + O(x^2)$ für $|x| < 1$ mit $x = -j/h_i$ angewandt.
\begin{align*}
    \P(s_i > s)
    = \prod_{j = 0}^{s} \Bigg ( \exp \bigg ( \frac {-j}{h_i} \bigg )
    - O \bigg ( \frac {j^2} {h_i^2} \bigg ) \Bigg )
\end{align*}
$\E(t_{\min})$ lässt sich nun wie folgt ausdrücken, wobei $t_{\max} = p \, \max_{i = 0}^M \lg 2k_i$ ist die maximal mögliche Anzahl an Zeiteinheiten ist.
\begin{align*}
    \E(t_{\min})
     & \sim \sum_{t = 0}^{t_{\max}} t \,
    \Bigg ( \prod_{i = 1}^M \P(t_i > t - 1) \Bigg ) \
    \P_{t_i > t - 1 \, \forall i}(t_i = t \text{ für mindestens ein } i) \\
     & = \sum_{t = 0}^{t_{\max}} t \,
    \Bigg ( \prod_{i = 1}^M \prod_{j = 0}^{(t-1) / \lg 2k_i}
    \Bigg ( \exp \bigg ( \frac {-j}{h_i} \bigg )
        - O \bigg ( \frac {j^2} {h_i^2} \bigg ) \Bigg )\Bigg ) \
    \P_{t_i > t - 1 \, \forall i}(t_i = t \text{ für mindestens ein } i)
\end{align*}
Der erste Faktor ist die Wahrscheinlichkeit, dass vor Zeitpunkt $t$ keine Kollision aufgetreten ist. Der zweite Faktor ist die Wahrscheinlichkeit, dass bei Zeitpunkt $t$ mindestens eine Kollision auftritt. $\P_{t_i > t - 1 \, \forall i} (\,\cdots)$ ist die Wahrscheinlichkeite dass $t_i > t$ für mindestens ein $i$, gegeben $t_i > t - 1$ für alle $i$. Bevor diese Wahrscheinlichkeit bestimmt wird, soll gezeigt werden, dass der erste Faktor durch Weglassen der $O(j^2/h_i^2)$-Terme vereinfacht werden kann.

\paragraph{Weglassen von $\boldsymbol{O(j^2/h_i^2)}$.} Intuitiv lässt sich das Weglassen von $O(j^2/h_i^2)$ damit rechtfertigen, dass wenn $s$ deutlich kleiner als $h_i$ ist, auch $j$ deutlich kleiner als $h_i$ ist und damit $j^2/h_i^2$ nahe 0 ist. Wenn dagegen $s$ nahe $h_i$ ist, ist $\P(s_i > s)$ sehr klein, sodass der Beitrag zum Erwartungswert vernachlässigbar ist. Präzise lässt sich das wie folgt begründen. Zunächst wird $s \le h_{i}^{3/5}$ angenommen und gezeigt, dass dann $\prod_{j = 0}^{(t-1) / \lg 2k_i} (\exp (-j/h_i) - O (j^2/h_i^2)) \sim \prod_{j = 0}^{(t-1) / \lg 2k_i} \exp (-j/h_i)$. Durch Ausmultiplizieren des Produkts erhält man
\begin{align*}
        & \ \Bigg \vert
    \prod_{j = 0}^{s} \Bigg ( \exp \bigg ( \frac {-j}{h_i} \bigg )
    - O \bigg ( \frac {j^2} {h_i^2} \bigg ) \Bigg )
    - \prod_{j = 0}^{s} \exp \bigg ( \frac {-j}{h_i} \bigg )
    \Bigg \vert          \\
    =   & \ \Bigg \vert
    - \sum_{0 \le a \le s} O \bigg ( \frac {a^2} {h_i^2} \bigg )
    \prod_{j = 0, j \notin \{a\}}^{s} \exp \bigg ( \frac {-j}{h_i} \bigg )
    + \sum_{0 \le a < b \le s} O \bigg ( \frac {a^2b^2} {h_i^4} \bigg )
    \prod_{j = 0, j \notin \{a, b\}}^{s}
    \exp \bigg ( \frac {-j}{h_i} \bigg ) - \cdots
    \Bigg \vert          \\
    \le & \ \Bigg \vert
    \sum_{0 \le a \le s} O \bigg ( \frac {a^2} {h_i^2} \bigg )
    \Bigg \vert + \Bigg \vert
    \sum_{0 \le a < b \le s} O \bigg ( \frac {a^2b^2} {h_i^4} \bigg )
    \Bigg \vert + \Bigg \vert
    \sum_{0 \le a < b < c \le s} O \bigg ( \frac {a^2b^2c^2} {h_i^6} \bigg )
    \Bigg \vert + \cdots \\
    \le & \
    s \, O \bigg ( \frac {s^2} {h_i^2} \bigg )
    +s^2 \, O \bigg ( \frac {s^4} {h_i^4} \bigg )
    +     s^3 \,  O \bigg ( \frac {s^6} {h_i^6} \bigg )
    + \cdots             \\
    \le & \
    O \bigg ( \frac 1 {h_i^{1/5}} \bigg )
    + O \bigg ( \frac 1 {h_i^{2/5}} \bigg )
    + O \bigg ( \frac 1 {h_i^{3/5}} \bigg )
    + \cdots
\end{align*}
Die Terme des ausmultiplizierten Produkts werden in der zweiten Zeile nach der Anzahl an $-O(j^2/h_i^2)$-Faktoren gruppiert. Von der zweiten zur dritten Zeile wird die Dreiecksungleichung verwendet und dass die Produke von $\exp(-j/h_i)$ kleiner gleich 1 sind. Anschließend ist alles positiv und die Beträge können weggelassen werden. Die letzte Zeile ist Teil einer geometrischen Reihe und geht daher für $h_i \to \infty$ gegen 0. (Eigentlich lassen wir $p \to \infty$ gehen, aber da $h_i = p/(\gcd(p - 1, 2k_i) - 1)$ gilt $h_i = \Theta(p)$ und damit $h_i \to \infty \Longleftrightarrow p \to \infty$.) Im Fall $s < h_i^{3/5}$ können wir also
\begin{align*}
    \P(s_i > s)
    \sim \prod_{j = 0}^{s} \exp \bigg ( \frac {-j}{h_i} \bigg )
    =\exp \sum_{j = 0}^{s} \frac {-j} {h_i}
    = \exp \frac {-s (s + 1)} {2h_i}
    \sim \exp \frac {-s^2} {2h_i}
\end{align*}
schreiben. Letzte Annäherung gilt für große $s$, für ausreichend große $p$ sind aber fast alle $s$ im Erwartungswert von $t_{\min}$ groß.

Nun wird $s > h_i^{3/5}$ angenommen und gezeigt, dass dann der Summand in $\E(t_{\min})$ für $p \to \infty$ gegen 0 geht. Hier gilt
\begin{align*}
    \P(s_i > s)
    = \prod_{j = 0}^{s} \exp \bigg ( \frac {-j}{h_i} \bigg )
    \le \prod_{j = 0}^{\big \lfloor h_i^{3/5} \big \rfloor}
    \exp \bigg ( \frac {-j}{h_i} \bigg )
    \sim \exp \frac {- \big \lfloor h_i^{3/5} \big \rfloor^2 } {2h_i}
    \le \exp \frac {- h_i^{1/5}} {2}
\end{align*}
Die anderen Wahrscheinlichkeiten in dem entsprechenden Summanden sind alle durch 1 begrenzt und $t$ ist $O(p)$. Da aber $h_i = \Theta(p)$ und $\lim_{p \to \infty} O(p) e^{-\Theta(p)^{1/5}} = 0$, geht der Summand gegen 0. Insgesamt bedeutet das, dass man durch Weglassen der $O(j^2/h_i^2)$-Terme entweder eine asymptotisch genaue Annäherung erhält, oder der Term, in dem man die Annäherung verwendet, sowieso gegen 0 geht. Als Zwischenergebnis erhalten wir
\begin{align}
    \E_{t_{\min}}
     & \sim \sum_{t = 0}^{t_{\max}} t \,
    \Bigg ( \prod_{i = 1}^M \prod_{j = 0}^{(t-1) / \lg 2k_i}
    \exp \bigg ( \frac {-j}{h_i} \bigg ) \Bigg ) \
    \P_{t_i > t - 1 \, \forall i}(t_i = t \text{ für mindestens ein } i)
    \nonumber                            \\
     & = \sum_{t = 0}^{t_{\max}} t \,
    \exp \Bigg ( \sum_{i = 1}^M \sum_{j = 0}^{(t-1) / \lg 2k_i}
    \frac {-j}{h_i} \Bigg ) \
    \P_{t_i > t - 1 \, \forall i}(t_i = t \text{ für mindestens ein } i)
    \nonumber                            \\
     & \sim \sum_{t = 0}^{t_{\max}} t \,
    \exp \Bigg ( \sum_{i = 1}^M \frac {-t^2} {2 h_i \lg^2 2k_i} \Bigg ) \
    \P_{t_i > t - 1 \, \forall i}(t_i = t \text{ für mindestens ein } i)
    \label{expectation-tmin-intermed}
\end{align}

\paragraph{$\boldsymbol{\pmb{\P}_{t_i > t - 1 \, \forall i}(t_i = t}$  für mindestens ein $\boldsymbol{i}$).} Die Wahrscheinlichkeit, dass bei Maschine $i$ eine Kollision nach genau $t$ Zeiteinheiten auftritt ist
    \begin{align*}
        \P_{t_i > t - 1}(t_i = t) \sim
        \begin{cases}
            t / h_i \lg 2k_i & \quad
            t = \lceil m \lg 2k_i \rceil \text{ für ein } m \in \N \\
            0                & \quad
            t \ne \lceil m \lg 2k_i \rceil \text{ für jedes } m \in \N
        \end{cases}
    \end{align*}
    Der erste Fall tritt ein, wenn Zeiteinheit $t$ das Ende einer Iteration von Maschine $i$ enthält. Da bei Zeitpunkt $t$ bereits $t/\lg 2k_i$ Schritte durchgeführt wurden, trifft Maschine $i$ im nächsten Schritt mit Wahrscheinlichkeit $t/h_i \lg 2k_i$ auf einen bereits besuchten Knoten. Im zweiten Fall befindet sich Maschine $i$ bei Zeiteinheit $t$ mitten in einer Iteration, es kann also keine Kollision auftreten. Zur Bestimmung von $\E_{t_{\min}}$ kann das durch
    \begin{align*}
        \P_{t_i > t - 1}(t_i = t) \sim \frac {t} {h_i \lg^2 k_i}
    \end{align*}
    angenähert werden. Die Wahrscheinlichkeit einer Kollision an einem Zeitpunkt wird durch den zusätzlichen Faktor $1/\lg 2k_i$ auf die umliegenden Zeitpunkte "`verteilt"'. Pro Iteration von Maschine $i$ gibt es statt einem Zeitpunkt mit Kollisionswahrscheinlichkeit $t/h_i \lg 2k_i$ nun $\lg 2k_i$ Zeitpunkte mit Kollisionswahrscheinlichkeit $t/h_i \lg^2 2k_i$. Dass das eine asymptotische Näherung ist, lässt sich damit begründen, dass der übrige Teil von (\ref{expectation-tmin-intermed}) stetig in $t$ ist und nicht schnell oszilliert oder Ähnliches. Ob die erste Kollision bei Maschine $i$ also bei Zeitpunkt $t$ oder $t + x$ für $|x| < \lg 2k_i$ auftritt macht daher asymptotisch keinen Unterschied. Die Wahrscheinlichkeit, dass bei mindestens einer Maschine eine Kollision auftritt, kann asymptotisch durch die Summe der Wahrscheinlichkeiten $\P_{t_i > t - 1}(t_i = t)$ angenähert werden. Denn für alle relevanten $t$ (d.h. $t \le h_i^{3/5} \, \forall i$) ist die Wahrscheinlichkeit, dass zwei oder mehr Maschinen gleichzeitig bei Zeit $t$ kollidieren für $p \to \infty$ verschwindend gering.
    \begin{align}
        \P_{t_i > t - 1 \, \forall i}(t_i = t \text{ für mindestens ein } i)
        \sim \sum_{i = 1}^M \frac t {h_i \lg^2 2k_i}
        = t \sum_{i = 1}^M \frac 1 {h_i \lg^2 2k_i}
        \label{prob-at-least-one-coll}
    \end{align}


    \paragraph*{Die finale Formel.} Für $\E(t_{\min})$ gilt mit (\ref{expectation-tmin-intermed}) und (\ref{prob-at-least-one-coll})
    \begin{align*}
        \E(t_{\min})
         & \sim \Bigg ( \sum_{i = 1}^M \frac 1 {h_i \lg^2 2k_i} \Bigg )
        \sum_{t = 0}^{t_{\max}} t^2 \, \exp \Bigg ( \frac {-t^2} 2
        \sum_{i = 1}^M \frac 1 {h_i \lg^2 2k_i} \Bigg )
    \end{align*}
    Als letzter Schritt wird die Summe über $t$ durch ein Integral angenähert. Die intuitive Erklärung dafür, dass das am asymptotischen Wert nichts ändert, ist ähnlich wie eben. (\ref{expectation-tmin-intermed}) und (\ref{prob-at-least-one-coll}) sind stetig in $t$ und schwanken nicht stark bei kleinen Veränderungen von $t$. Die Werte an einzelnen Punkten, wie sie in der Summe vorkommen, sind also nahezu gleich den Werten um diese Punkte herum, die zusätzlich im Integral vorkommen. Nach einem ähnlichen Argument wie dem dafür, dass Terme mit $t/\lg 2k_i > h_i^{3/5}$ für ein $i$ asymptotisch irrelevant sind, kann die obere Integralgrenze bis $\infty$ geöffnet werden. Man erhält also
    \begin{align}
        L_{k_1, \dots k_M}(p) = \E(t_{\min})
         & \sim \Bigg ( \sum_{i = 1}^M \frac 1 {h_i \lg^2 2k_i} \Bigg )
        \int_{0}^{\infty} t^2 \, \exp \Bigg ( \frac {-t^2} 2
        \sum_{i = 1}^M \frac 1 {h_i \lg^2 2k_i} \Bigg ) \; dt
        \nonumber                                                       \\
         & = \Bigg (\sum_{i = 1}^M \frac 1 {h_i \lg^2 2k_i} \Bigg ) \,
        \sqrt {\pi / 2} \
        \Bigg ( \sum_{i = 1}^M \frac 1 {h_i \lg^2 2k_i} \Bigg )^{-3/2}
        \nonumber                                                       \\
         & = \sqrt{\pi p / 2} \; \Bigg (
        \sum_{i = 1}^M \frac {\gcd(p - 1, 2k_i) - 1} {\lg^2 2k_i} \Bigg )^{-1/2}
        \label{expectation-tmin}
    \end{align}
    Zur Auswertung des Integrals wurde die Tabelle in Wikipedia: \cite{gint} verwendet. Man kann sich auf mehereren Wegen davon überzeugen, dass die Formel trotz der vielen Näherungen stimmt. Beispielsweise ist sie verträglich mit der bekannten Laufzeitabschätzung für Pollards Rho-Algorithmus im Fall $M = k_1 = 1$. Setzt man dies ein, erhält man $\sqrt{\pi p / 2}$, wie in \cite{pol75}. Lässt man in obigem Integral einen Faktor $t$ weg, erhält man statt des Erwartungswerts das Integral aller Wahrscheinlichkeiten. Dieses sollte natürlich 1 sein, und das ist es auch.

    \section{Die erwartete minimale Rho-Länge bei mehreren Anfangswerten}

    Um $L_{k_1, \dots, k_M}(p)$ zu bestimmen, wenn $k_i = k_j$ für $i \ne j$ gilt, muss die Abhängigkeit der Rho-Längen der $i$-ten und $j$-ten Maschinen berücksichtigt werden. Denn setzt man beispielsweise $M = 2$ und $k_1 = k_2 = 1$ in (\ref{expectation-tmin}) ein, erhält man eine erwartete Laufzeit von $\sqrt {\pi p / 4}$. In diesem Abschnitt wird allerdings gezeigt, dass unter Berücksichtigung der Abhängigkeit $25/32 \sqrt{\pi p / 2}$ Schritte benötigt werden. Der Fall abhängiger Maschinen hat sich als weitaus schwieriger herausgestellt und es wurde keine allgemeine Formel gefunden. Jedoch konnte der Fall $M = 2, \, k_1 = k_2$ gelöst werden. In diesem Fall lässt sich das Problem folgendermaßen angehen. Sei $k = k_1 = k_2, \; h = p/(\gcd(p - 1, 2k) - 1)$ und $f : x \mapsto x^{2k} + 1$. Da die zwei Maschinen beide $f$ verwenden, bewegen sie sich beide im funktionalen Graphen $\gamma(f)$. Nach der Random Mapping Assumption ist die Verteilung der Rho-Längen von $f$ für $p \to \infty$ asymptotisch zur Verteilung der Rho-Längen eines zufälligen Elements aus $A(h)$. Das Problem reduziert sich daher auf folgende Frage: \emph{Für $n \in \N$, gegeben ein zufälliges Element $g$ aus $A(n)$ und zwei zufällige Elemente $a, b \in Z/n\Z$, was ist der Erwartungswert von $\min\{\nu(g, a), \nu(g, b)\}$?} Die Frage wird von folgendem Satz beantwortet, dessen Beweis das Ziel der nächsten zwei Abschnitte ist.

    \begin{theorem}
        \label{theorem:min-rho-len-m2}
        Sei $A(n)$ die Menge der Abbildungen $\Z/n\Z \to \Z/n\Z$. Wir bezeichnen mit
        \begin{align*}
            \tau_n =  \frac 1 {n^{n + 2}}
            \sum_{g \in A(n)} \; \sum_{a, b \in \Z/n\Z}
            \min\{\nu(g, a), \nu(g, b)\}
        \end{align*}
        die erwartete minimale Rho-Länge zweier zufälliger Knoten in einem funktionalen Graphen von Größe $n$. Es gilt
        \begin{align*}
            \tau_n \sim \frac {25} {32} \sqrt{\pi n / 2}
        \end{align*}
    \end{theorem}

    \noindent Der Vorfaktor in der Definition von $\tau_n$ ist $1/n^{n + 2}$, da es $n^n$ Abbildungen $\Z/n\Z \to \Z/n\Z$ gibt und für jede von diesen $n^2$ Paare an Anfangswerten.

    \subsection{Theoretischer Hintergrund: Erzeugende Funktionen}

    Der Ansatz zum Beweis von Satz \ref{theorem:min-rho-len-m2} ist, die Summe in der Definition von $\tau_n$ mithilfe einer erzeugenden Funktion $\Psi(x, w)$ zu bestimmen. In $\Psi(x, w)$ markiert die Variable $x$ die Größe des Graphen und die Variable $w$ die minimale Rho-Länge der zwei Anfangsknoten. Da funktionale Graphen beschriftet sind, werden stets erzeugende Funktionen von exponentiellem Typ (EF) verwendet. Also gilt
    \begin{align}
        \Psi(x, w)
        = \sum_{n = 0}^\infty \frac {x^n}{n!} \sum_{g \in A(n)}
        \sum_{a, b \in \Z/n\Z} w^{\min\{\nu(g, a), \nu(g, b)\}}
        \label{psi-definition}
    \end{align}
    Aus den Koeffizienten der Reihe von $\Psi$ kann dann wie folgt $\tau_n$ bestimmt werden.
    \begin{align*}
        \tau_n = \frac {n!}{n^{n + 2}} [x^n] \Bigg (\frac {\partial} {\partial w} \Psi(x, w) \Bigg ) \Bigg \vert_{w = 1}
    \end{align*}
    wobei $[x^n]$ den $n$-ten Koeffizienten in der Reihenentwicklung des nachstehenden Terms bezeichnet.

    Folgende Komponenten werden als Grundlage verwendet, um $\Psi(x, w)$ zu konstruieren.
    \begin{align*}
        S(x) & = \frac 1 {1 - x}    & \quad (\text{Folge / Pfad})       \\
        T(x) & = x \exp T(x)        & \quad (\text{Baum})               \\
        F(x) & = \frac 1 {1 - T(x)} & \quad (\text{Funktionaler Graph})
    \end{align*}
    Die letzten beiden Funktionen stammen von \cite{fo90}. Übersetzt bedeutet die EF für einen Baum: "`Ein Baum ist ein Knoten mit einer Menge an Bäumen"'. Die Erklärung für $F(x)$ ist etwas komplizierter, hier aber nicht wichtig. Ein Beweis findet sich in \cite{fs09}, S. 129.

    \subsection{Der Fall zweier Anfangswerte}

    \begin{proof}[Beweis von Satz \ref{theorem:min-rho-len-m2}]
        Seien $a$ und $b$ die zwei Startknoten und $\Psi(x, w)$ wie in (\ref{psi-definition}) definiert. Der Plan ist, zunächst einen geschlossenen Ausdruck für $\Psi(x, w)$ zu bestimmen, um dann die Ableitung nach $w$ zu bilden. Dafür werden drei Fälle unterschieden, die in Abbildung \ref{fig:psi-construction} dargestellt sind. In jedem der Fälle wird zunächst eine EF für funktionale Graphen bestimmt, die nur die für $a$ und $b$ relevanten Teile enthalten. Der "`relevante Teil"' besteht aus all den Knoten, die beim Ablaufen des Graphen besucht werden, also dem Zyklus und dem Pfad zum Zyklus. Der Rest des funktionalen Graphen wird später ergänzt. Grundsätzlich werden die Graphen so konstruiert, dass $\nu(f, a) \le \nu(f, b)$. Wenn dann $\nu(f, a) < \nu(f, b)$ gilt, wird mit einem Faktor 2 für das mögliche Vertauschen von $a$ und $b$ multipliziert. Im Fall $\nu(f, a) = \nu(f, b)$ darf man das nicht tun, weil der Graph sonst doppelt gezählt wird.

        \begin{figure}
            \begin{tabular}{ccc}
                \includesvg[width=145pt]{pics/alpha} & \includesvg[width=145pt]{pics/beta} & \includesvg[width=145pt]{pics/gamma} \\
                ($\alpha$)                           & ($\beta$)                           & ($\gamma$)
            \end{tabular}
            \caption{Die drei Fälle für die Bestimmung von $\Psi(x, w)$. Die Kanten stellen keine einzelne Kante dar, sondern einen beliebig langen (und möglicherweise leeren) Pfad. Beispielsweise steht die Kurve $r$ in ($\alpha$) für den Zyklus in dem Zusammenhangskomponenten von $a$.}
            \label{fig:psi-construction}
        \end{figure}

        \paragraph{Fall 1.} $a$ und $b$ liegen in unterschiedlichen Zusammenhangskomponenten. Dieser Fall wird erneut in die Fälle $\lambda(f, b) \le \nu(f, a)$ und $\lambda(f, b) > \nu(f, a)$ unterteilt. Die erzeugende Funktion für den ersten Fall ist
        \begin{align*}
            \alpha_1(x, w) = \frac {x^2w(1 + x^2w)} {(1 - x^2w)^3} \cdot \Bigg (1 + \frac {2x} {1 - x} \Bigg ) = \frac {x^2w(1 + x)(1 + x^2w)} {(1 - x^2w)^3(1 - x)}
        \end{align*}
        In diesem Fall ist es möglich, zuerst zwei $\rho$-Graphen mit gleicher Größe zu erzeugen und anschließend den Pfad von $b$ zu seinem Zyklus zu verlängern. Ein $\rho$-Graph ist ein Zusammenhangskomponent in Abbildung \ref{fig:psi-construction} ($\alpha$), d.h. ein Zyklus mit einem Pfad anhängend. Es gibt genau $n! \cdot n$ $\rho$-Graphen mit $n$ Knoten, da es für jede Permutation der Knoten $n$ Möglichkeiten für die Größe des Zyklus gibt. Folglich gibt es für gerade $n$ genau $n! \cdot n^2/2$ Paare an $\rho$-Graphen, die beide $n/2$ Knoten besitzen. Die erzeugende Funktion von Paaren an $\rho$-Graphen gleicher Größe ist also
        \begin{align*}
            \sum_{n = 0}^\infty x^n \; \frac {n^2} 4 \, \frac {1 + (-1)^n} 2
            = \frac {x^2(1 + x^2)} {(1 - x^2)^3}
        \end{align*}
        Da $x$ die Zahl an Knoten markiert, wird von $w$ die halbe Zahl an Knoten markiert, wenn $x$ durch $x \sqrt w$ ersetzt wird. So markiert $w$ die Rho-Länge von $a$ und man erhält die EF $x^2w(1 + x^2w) / (1 - x^2w)^3$. Das erklärt den ersten Faktor in $\alpha_1(x, w)$. Nun gibt es zwei Möglichkeiten: Wird der Pfad von $b$ zu seinem Zyklus ($u$ in Abbildung \ref{fig:psi-construction} ($\alpha$)) nicht verlängert, gilt $\nu(f, a) = \nu(f, b)$. Wird hingegen ein Pfad von Länge $\ge 1$ angehängt, dessen erzeugende Funktion $x/(1 - x)$ ist, ergeben sich zwei Möglichkeiten durch Vertauschen von $a$ und $b$.

        Im zweiten Fall ist die erzeugende Funktion
        \begin{align*}
            \alpha_2(x, w) =
            2 \; \frac {x^2w} {(1 - x^2w)^2} \, \frac x {1 - x} \,
            \frac {1}{1 - x}
            = \frac {2x^3w} {(1 - x^2w)^2 (1 - x)^2}
        \end{align*}
        Der Faktor $x^2w$ repräsentiert die zwei Knoten, an denen $a$ und $b$ jeweils ihren Zyklus betreten, und der Knoten von $a$ ist mit $w$ markiert. Mit $1 / (1 - x^2w)^2$ erhält man vier Pfade $r, y, s, z$, sodass die Länge von $r$ gleich der Länge von $y$ und die Länge von $s$ gleich der Länge von $z$ ist. Die Summe der Längen von $r$ und $s$ wird von $w$ markiert. $r$ und $s$ werden wie in Abbildung \ref{fig:psi-construction} ($\alpha$) für den Zusammenhangskomponenten von $a$ verwendet. Damit ist der Exponent von $w$ genau die Rho-Länge von $a$. Der Zyklus von $b$ besteht aus $y, z$ und einem Pfad von Länge $\ge 1$, sodass $\lambda(f, b) > \nu(f, a)$ gilt. Der Term $1 / (1 - x)$ steht für den Pfad von $b$ zum Zyklus.

        \noindent Die erzeugende Funktion für Fall 1 ist also
        \begin{align*}
            \alpha(x, w)
            = \alpha_1(x, w) + \alpha_2(x, w)
            = \frac {x^2w(1 + 2x - x^2 + x^2w - 2x^3w - x^4w)}
            {(1 - x^2w)^3(1 - x)^2}
        \end{align*}

        \paragraph{Fall 2.}  In diesem Fall liegen $a$ und $b$ im gleichen Baum und ihr kleinster gemeinsamer Vorfahre ist nicht die Wurzel. Anders formuliert: Betrachtet man die Pfade, die $a$ und $b$ durch wiederholtes Anwenden von $f$ in $\gamma(f)$ ablaufen, treffen diese sich nicht erstmals in einem Zyklusknoten. Die erzeugende Funktion lautet
        \begin{align*}
            \beta(x, w)
            = xw \; \frac {xw} {1 - xw} \, \frac {1} {1 - xw} \,
            \frac {1} {1 - x^2w} \, \Bigg (1 + \frac {2x} {1 - x} \Bigg )
            = \frac {x^2w^2(1 + x)} {(1 - xw)^2(1 - x^2w)(1 - x)}
        \end{align*}
        Der Zyklusknoten, an dem der Baum von $a$ und $b$ anhängt, wird durch $xw$ repräsentiert. Der Pfad $s$ in Abbildung 1 ($\beta$) muss mindestens Länge 1 haben, da der kleinste gemeinsame Vorfahre von $a$ und $b$ sonst die Wurzel wäre, was den Faktor $xw/(1 - xw)$ erklärt. Der Faktor $1/(1 - xw)$ steht für den Zyklus $r$. Mit $1/(1 - x^2w)$ werden zwei gleich lange Pfade erzeugt, deren einfache Länge durch $w$ markiert wird. Ein Pfad ist $t$ in Abbildung \ref{fig:psi-construction} ($\beta$), und der andere ist ein Teil von $u$. Nun gibt es wie in Fall 1 wieder die Option, $u$ echt zu verlängern und so einen Faktor 2 für die mögliche Vertauschung von $a$ und $b$ zu erhalten, oder ihn zu lassen, wobei es wegen Symmetrie nur eine Möglichkeit gibt.


        \paragraph{Fall 3.} Zuletzt bleibt der Fall, wenn $a$ und $b$ in verschiedenen Bäumen liegen oder die Wurzel ihr kleinster gemeinsamer Vorfahre ist. Eine andere Sichtweise ist, dass sich die Pfade von $a$ und $b$ bei wiederholtem Anwenden von $f$ erstmals bei einem Zyklusknoten treffen. Die erzeugende Funktion ist hier
        \begin{align*}
            \gamma(x, w)
            = xw \, \frac {1} {(1 - xw)^2} \, \frac {1} {1 - x^2w} \,
            \Bigg (1 + \frac {2x}{1 - x} \Bigg )
            = \frac {xw(1 + x)} {(1 - xw)^2(1 - x^2w)(1 - x)}
        \end{align*}
        $xw$ stellt den Zyklusknoten da, an dem der Baum von $a$ anhängt. Der Term $1/(1 - xw)^2$ repräsentiert die beiden Pfade von der Wurzel von $a$ zur Wurzel von $b$ und zurück ($r$ und $s$ in Abbildung 1 ($\gamma$)). Ähnlich wie in Fall 2 ist der Term $1/(1 - x^2w)$ ein Paar an gleich langen Pfaden, deren Länge von $w$ markiert wird. Einer der Pfade ist $u$ in Abbildung 1 ($\gamma$) und der andere ein Teil von $t$. Auch hier kann man die Länge von $t$ Zyklus unverändert lassen, in diesem Fall gibt es eine Möglichkeit, oder einen Pfad von Länge $\ge 1$ hinzufügen, sodass es zwei Möglichkeiten wegen Vertauschung von $a$ und $b$ gibt.

        Ein funktionaler Graph besteht natürlich nicht nur aus einem Zyklus und den Pfaden von $a$ und $b$ zum Zyklus. Von jedem Knoten kann ein Baum ausgehen und es kann noch weitere Zusammenhangskomponenten geben. Indem $x$ durch $T(x)$ ersetzt wird, kann von jedem Knoten ein Baum ausgehen. Weitere Zusammenhangskomponenten bilden einen funktionalen Graphen, um diese zuzulassen, wird die gesamte EF also mit $F(x)$ multipliziert. Da die drei Fälle disjunkt sind und zusammen alle Möglichkeiten abdecken, gilt
        \begin{align*}
            \Psi(x, w)
            = (\alpha(T(x), w) + \beta(T(x), w) + \gamma(T(x), w)) \,F(x)
        \end{align*}
        Damit erhalten wir
        \begin{align*}
            \psi(x) = \Bigg (\frac {\partial} {\partial w}
            \Psi(x, w) \Bigg ) \Bigg \vert_{w = 1}
            = \frac
            {T(x)(1 + 2T(x) + 2(T(x))^2)(1 + 5T(x) + 3(T(x))^2 + (T(x))^3)}
            {(1 - T(x))^6(1 + T(x))^3}
        \end{align*}
        Nun soll die Methode von \cite{fo90} verwendet werden, um eine asymptotische Abschätzung für die Koeffizienten der Taylorreihe von $\psi(x)$ zu erhalten. Nach \cite{fo90}, S. 334, Proposition 1 ist die betragsmäßig (in $\C$) kleinste Singularität von $T(x)$ bei $x = e^{-1}$ und es gilt
        \begin{align*}
            T(x) = 1 - \sqrt{2} \; \sqrt {1 - ex} - O(1 - ez)
        \end{align*}
        für $x \to e^{-1}$. $\psi(x)$ hat keine betragsmäßig kleineren Singularitäten, denn wenn $1 - T(x) = 0$, rechnet man leicht nach, dass $x = e^{-1}$ gilt, und wenn $1 + T(x) = 0$, gilt $x = -e$. Es wird nun Theorem 1 aus \cite{fo90}, S. 333 mit $s = e^{-1}$ verwendet. Wenn $x \to e^{-1}$, gilt
        \begin{align*}
            \psi(x)
             & \sim \frac {T(e^{-1})(1 + 2T(e^{-1})
                + 2(T(e^{-1}))^2)(1 + 5T(e^{-1}) + 3(T(e^{-1}))^2 + (T(e^{-1}))^3)}
            {(1 - (1 - \sqrt 2 \cdot \sqrt {1 - ez}))^6(1 + T(e^{-1}))^3} \\
             & = \frac {(1 + 2 + 2)(1 + 5 + 3 + 1)} {2^3} \,
            \frac 1 {(\sqrt 2 )^6 (\sqrt{1 - ez})^6}                      \\
             & = \frac {50} {64} \, \frac 1 {(1 - ez)^3}
        \end{align*}
        Folglich gilt mit der Notation in \cite{fo90} $\sigma(x) = x^3$ und $\alpha = 3$. Aus Theorem 1 folgt
        \begin{align*}
            [x^n] \psi(x)
            \sim \frac {50} {64} \; (e^{-1})^{-n} \;
            \frac {n^3} {n \Gamma(3)}
            = \frac {25} {32} \; \frac {e^n n^2} {2}
        \end{align*}
        und mit Stirlings Näherung $n! \sim \sqrt{2\pi n} (n/e)^n$
        \begin{align*}
            \tau_n
            \sim \frac {n!}{n^{n + 2}} \, \frac {25} {32} \,\frac {e^n n^2} 2
            = n! \bigg (\frac {e} {n} \bigg )^n \, \frac {25} {64}
            \sim \sqrt {2 \pi n} \; \frac {25}{64}
            = \frac {25} {32} \sqrt{\pi n/2}
        \end{align*}
    \end{proof}

    \noindent Aus Satz \ref{theorem:min-rho-len-m2} folgt sofort, dass $L_{1, 1}(p) \sim 25/32 \sqrt{\pi p / 2}$. Daraus lässt sich auch auf $L_{k, k}(p)$ schließen. Für $1 \le k \in \N$ wird nach RMA $p$ durch $p/(\gcd(p - 1, 2k) - 1)$ ersetzt. Die Laufzeit pro Iteration steigt bei jeder Maschine aber um einen Faktor $\lg 2k$, sodass sie auch insgesamt um einen Faktor $\lg 2k$ steigt. Es gilt also
    \begin{align}
        L_{k, k}(p) \sim 25 / 32 \sqrt{\pi p /2} \
        \frac {\lg 2k} {\sqrt{\gcd(p - 1, 2k) - 1}}
        \label{lkkp}
    \end{align}
    Man bemerke, dass Satz \ref{theorem:min-rho-len-m2} unabhängig von der Anwendung auf Pollards Rho-Algorithmus formuliert wurde und nicht auf der Random Mapping Assumption basiert.

    \section{Bestimmung optimaler Exponenten für die Rho-Methode}
    \label{sec:optimal-k}

    In diesem Abschnitt wird die Frage behandelt, wie der Paramter $k$ bei $M$ Maschinen bestmöglich gewählt werden kann. Mit Formel (\ref{expectation-tmin}) und Satz \ref{theorem:min-rho-len-m2} konnten Ergebnisse in den Fällen $M = 1$ und $M = 2$ erzielt werden. Die grundlegende Strategie ist, den Erwartungswert von $L_{k_1, \dots, k_M}(p)$ über alle Möglichkeiten von $\gcd(p - 1, 2k_i)$ für alle $1 \le i \le M$ zu bilden und so einen Wert für die erwartete Laufzeit in Abhängigkeit der $k_i$ zu erhalten. Da $p - 1$ gerade ist, gilt $\gcd(p-1, 2k_i) = 2\gcd((p - 1)/2, k_i)$. Weitere Kongruenzen von $p - 1$ sind im Allgemeinen nicht bekannt, weshalb angenommen wird, dass jeder Rest $(p- 1)/2$ modulo $k_i$ gleich wahrscheinlich ist.

    \subsection{Der Fall einer Maschine}

    \begin{theorem}
        \label{theorem:optimal-k-m1}
        Sei $L_k(p)$ wie in (\ref{lk-definition}) definiert. Es gilt $\E(L_1(p)) < \E(L_k(p))$, wobei $1 < k \in \N$ und der Erwartungswert über alle möglichen $\gcd((p - 1)/2, k)$ genommen wird.
    \end{theorem}

    \begin{proof}
        Durch Einsetzen von $M = 1$ in (\ref{expectation-tmin}) erhalten wir
        \begin{align*}
            L_k(p) \sim \sqrt {\pi p / 2} \;
            \frac {\lg 2k} {\sqrt{\gcd(p - 1, 2k) - 1}}
        \end{align*}
        Die erwartete Laufzeit im Fall $k = 1$ ist folglich $\sqrt{\pi p/2}$, es wird also gezeigt, dass $\E(L_k(p)) > \sqrt{\pi p / 2}$ für $k > 1$. Mit $\varphi$ wird die eulersche Phifunktion bezeichnet. Dann gilt
        \begin{align*}
            \E(L_k(p))
             & \sim \sqrt{\pi p / 2} \, \lg (2k) \,
            \sum_{d | k} \frac {\P(\gcd((p - 1)/2, k) = d)}
            { \sqrt {2d - 1}} \nonumber                                 \\
             & \ge \sqrt{\pi p / 2} \, \lg (2k) \, \frac {\varphi(k)} k
        \end{align*}
        Die Ungleichung lässt sich zeigen, indem statt der Summe über alle Teiler nur $d = 1$ betrachtet wird. Da es $\varphi(k)$ teilerfremde Zahlen $< k$ gibt, ist $\P(\gcd((p - 1)/2, k) = 1) = \varphi(k)/k$. Nach \cite{rs62}, Theorem 15 gilt
        \begin{align*}
            \frac {\varphi(k)} k
             & > \frac 1 {e^\gamma \ln (\ln (k)) + 2.51 / \ln (\ln (k))}
        \end{align*}
        für $k \ge 3$, wobei $\gamma \approx 0.5772$ die Euler-Mascheroni-Konstante ist. Für $k \ge e^e$ folgt daraus $\varphi(k) / k > 1/(e^\gamma \ln(\ln(k)) + 2.51)$. Indem nun gezeigt wird, dass $\lg (2k) / (e^\gamma \ln (\ln (k)) + 2.51) > 1$ für $k \ge 16$ wird der Satz im Fall $k \ge 16$ bewiesen. Sei $f(x) = \lg (2x) / (e^\gamma \ln (\ln (x)) + 2.51)$. Es gilt $f(16) \approx 1.1557$ und
        \begin{align*}
            f'(x)
            = \frac {\ln (x)(\ln (\ln (x)) + 1.51) - \ln 2}
            {x \ln (2) \ln (x)(\ln(\ln(x)) + 2.51)^2}
        \end{align*}
        Für $x \ge 16$ ist der Nenner von $f'$ positiv, denn $x > 0$ und $\ln x > 0$, und weil $\ln \ln x > 1$ für $x \ge 16$ ist der Term unter dem Quadrat positiv. Der Zähler ist ebenfalls positiv, da $\ln x \ge 1$ und $\ln \ln x \ge 1$, woraus $\ln(x)(\ln(\ln(x)) + 1.51) \ge 1 \cdot (1 + 1.51) = 2.51 > \ln 2$ folgt. Also ist $f$ streng monoton steigend für $x \ge 16$, und da bereits $f(16) > 1$ gezeigt wurde, folgt $f(x) > 1$ für $x \ge 16$. Der Fall $k < 16$ wurde durch Ausrechnen von (\ref{expectation-tmin}) für $1 \le k \le 15$ überprüft.
    \end{proof}

    \subsection{Der Fall zweier Maschinen}

    \begin{theorem}
        \label{theorem:optimal-k-m2}
        Sei $L_{k_1, k_2}(p)$ wie in Abschnitt (\ref{lk-definition}) definiert. Im Folgenden wird der Erwartungswert von $L_{k_1, k_2}$ über alle möglichen $\gcd((p - 1)/2, k_i)$ für $i = 1, 2$ genommen.
        \begin{enumerate}
            \item Wenn $k_1, k_2$ unterschiedliche Primzahlen sind, gilt $\E(L_{1, 1}(p)) < \E(L_{k_1, k_2}(p))$.
            \item Wenn $1 < k \in \N$, gilt $\E(L_{1, 1}(p)) < \E(L_{k, k}(p))$.
        \end{enumerate}
    \end{theorem}

    \begin{proof}
        Nach Satz 1 gilt $L_{1, 1}(p) \sim 25/32 \sqrt{\pi p /2}$, es gilt also in beiden Teilen zu zeigen, dass der Erwartungswert größer ist. Zunächst wird Teil 1 bewiesen. Da $k_1 \ne k_2$, sind die zwei Maschinen unabhängig und (\ref{expectation-tmin}) kann verwendet werden.
        \begin{align*}
            L_{k_1, k_2}(p)
             & \sim \sqrt{\pi p / 2} \
            \Bigg ( \frac {\gcd(p - 1, 2k_1) - 1} {\lg^2 2k_1} +
            \frac {\gcd(p -1, 2k_2) - 1} {\lg^2 2k_2} \Bigg )^{-1/2}
        \end{align*}
        Durch Bilden des Erwartungswerts über alle möglichen $\gcd((p - 1)/2, k_i)$ für $i = 1, 2$ erhält man
        \begin{align}
            \E(L_{k_1, k_2})
             & \sim \sqrt{\pi p / 2}
            \sum_{d_1 | k_1} \P(\gcd((p - 1)/2, k_1) = d_1)
            \nonumber                                                 \\
             & \qquad \sum_{d_2 | k_2} \P(\gcd((p - 1)/2, k_2) = d_2)
            \Bigg ( \frac {2d_1 - 1} {\lg^2 2k_1}
            + \frac {2d_2 - 1} {\lg^2 2k_2} \Bigg )^{-1/2}
            \nonumber                                                 \\
             & \ge \sqrt{\pi p / 2} \
            \frac {\varphi(k_1) \varphi(k_2)} {k_1k_2}
            \Bigg (\frac 1 {\lg^2 2k_1} + \frac 1 {\lg^2 2k_2} \Bigg )^{-1/2}
            \nonumber                                                 \\
             & = \sqrt{\pi p / 2} \
            \frac {(k_1 - 1) (k_2- 1)} {k_1k_2}
            \sqrt{\frac{\lg^2(2k_1) \lg^2(2k_2)}{\lg^2(2k_1) + \lg^2(2k_2)}}
            \label{m2-expectation-lb}
        \end{align}
        Wie bei $M = 1$ wurden die Summen über alle Teiler von $k_1, k_2$ durch den Wert für $d_1 = d_2 = 1$ nach unten begrenzt. Die Terme $(k_i - 1)/k_i$ sind streng monoton steigend mit $k_i$ für $i = 1, 2$. Ebenso ist das Argument der nachfolgenden Wurzel in (\ref{m2-expectation-lb}) streng monoton steigend mit jedem der $k_i$. Um das zu zeigen, sei $x = \lg^2 2k_1, y = \lg^2 2k_2$ und $x' > x$. Dann gilt
        \begin{align*}
            \frac {x'y} {x' + y}
            = \frac {x'y} {x' + y} \, \frac {x + y} {xy} \, \frac {xy} {x + y}
            = \frac {xx'y + x'y^2} {xx'y + xy^2} \, \frac {xy} {x + y}
            > \frac {xy} {x + y}
        \end{align*}
        da $x, x', y > 0$ und $x < x'$. $x$ und $y$ sind symmetrisch in dem Term, daher ist er auch streng monoton steigend bzgl. $y$. Da die Wurzelfunktion streng monoton steigt und die Verkettung streng monoton steigender Funktionen streng monoton steigt, folgt, dass die gesamte Wurzel auf der rechten Seite von (\ref{m2-expectation-lb}) streng monoton steigt. Weil nun jeder einzelne Faktor in (\ref{m2-expectation-lb}) streng monoton steigt und positiv ist, ist ganz (\ref{m2-expectation-lb}) streng monoton steigend in den $k_i$. Indem man $k_1 = 3,\; k_2 = 5$ in (\ref{m2-expectation-lb}) einsetzt, sieht man, dass $E(L_{3, 5}(p)) \ge 1.0881 \sqrt{\pi p /2}  > 25 / 32 \sqrt{\pi p / 2}$. Weil (\ref{m2-expectation-lb}) symmetrisch in $k_1, k_2$ ist, kann $k_1 < k_2$ angenommen werden. Dann folgt aus der Monotonie von (\ref{m2-expectation-lb}), dass $E(L_{k_1, k_2}(p)) > \E(L_{1, 1}(p))$, wenn $k_1 \ge 3$. Es bleibt also lediglich der Fall $k_1 = 2$. Durch Einsetzen von $k_1 = 2, \; k_2 = 11$ in (\ref{m2-expectation-lb}) gilt $\E(L_{2, 11}(p)) \ge 0.8294 \sqrt{\pi p / 2} > 25 / 32 \sqrt{\pi p / 2}$. Aus der Monotonie von (\ref{m2-expectation-lb}) folgt $\E(L_{k_1, k_2}(p)) > \E(L_{1, 1}(p))$ für $k_1 = 2$ und $k_2 \ge 11$. Die übrigen Fälle $k_1 = 2$ und $k_2 = 3, 5, 7$ wurden nachgerechnet.

        Nun zum Beweis von Teil 2. Aus (\ref{lkkp}) folgt $L_{k, k}(p) = 25 / 32 \cdot L_{k}(p)$ für $1 \le k \in \N$. Damit folgt Teil 2 des Satzes aus Satz 3.
    \end{proof}

    \section{Experimentelle Ergebnisse}

    Um zu demonstrieren, dass Formel (\ref{expectation-tmin}), Satz \ref{theorem:optimal-k-m1} und Satz \ref{theorem:optimal-k-m2} das Laufzeitverhalten von Pollards Rho-Algorithmus gut beschreiben, wurden Laufzeitmessungen für $M = 1$ und $M = 2$ durchgeführt. Als Testzahlen wurden 62-Bit Semiprimzahlen verwendet. Eine Semiprimzahl ist das Produkt zweier gleich großer Primzahlen, hier also zweier 31-Bit Primzahlen. Zahlen dieser Art sind ein \emph{worst-case-Input} für den Rho-Algorithmus, da der kleinste Primfaktor maximal groß im Verhältnis zur Zahl ist.

    \paragraph{$\pmb{M = 1}$.} Es wurden $2^{20}$ zufällige Semiprimzahlen gewählt und die durchschnittliche Laufzeit berechnet. Die Messergebnisse für $1 \le k \le 30$ und Werte von (\ref{expectation-tmin}) zum Vergleich sind in Abbildung \ref{fig:measurements-m1} dargestellt. Die Laufzeit wird gut von (\ref{expectation-tmin}) angenähert, denn die Werte befinden sich in der gleichen Größenordnung und auch Charakteristika spezieller Zahlen, wie beispielsweise hohe Werte bei Primzahlen, werden von beiden reflektiert. Die dennoch vorhandenen Abweichungen sind möglicherweise durch die Näherung $\lg 2k$ für die Anzahl an Multiplikationen zur Berechnung von $x^{2k}$ zu erklären. Die Anzahl nötiger Multiplikationen von \emph{Square-And-Multiply} (der Standardalgorithmus zum Berechnen natürlicher Potenzen) hängt nämlich von der Zahl an Einsen in der Binärdarstellung von $2k$ ab. Das könnte die unerwartet geringe Laufzeit bei $k = 8$ und $k = 16$ erklären.

    \begin{figure}
        \centering
        \begin{tikzpicture}
            \begin{axis}[
                    xlabel = {$k$},
                    xmin = 1,
                    xmax = 30,
                    xtick = {1, 5, 10, 15, 20, 25, 30},
                    width = 0.8 * \textwidth,
                    legend pos = north west,
                    legend style = {draw = none},
                    legend cell align = left,
                ]

                \addplot[color=magenta,mark=square]
                coordinates{
                        (1, 1)
                        (2, 1.5773502691896257)
                        (3, 2.1086517918741277)
                        (4, 2.21648605664914)
                        (5, 2.879004348902381)
                        (6, 2.332272327437964)
                        (7, 3.4142999705039783)
                        (8, 2.89543195056782)
                        (9, 3.3067332978837425)
                        (10, 2.9700938779475328)
                        (11, 4.142494904167607)
                        (12, 2.794948220810239)
                        (13, 4.41118188933241)
                        (14, 3.411127980556713)
                        (15, 3.481018974295859)
                        (16, 3.5947294420471545)
                        (17, 4.840295239186588)
                        (18, 3.2665648302249277)
                        (19, 5.017128905823505)
                        (20, 3.4266483423121223)
                        (21, 3.952865384732848)
                        (22, 4.006284568144385)
                        (23, 5.319207262349245)
                        (24, 3.3342189545431062)
                        (25, 4.848341222073144)
                        (26, 4.224515025049568)
                        (27, 4.541184581320172)
                        (28, 3.8606214340939586)
                        (29, 5.682737117278755)
                        (30, 3.3580839014528183)
                    };

                \addplot[color=blue,mark=square]
                coordinates {
                        (1, 1.0)
                        (2, 1.4766911107604923)
                        (3, 2.01813906820873)
                        (4, 2.0577528189326095)
                        (5, 2.876949988974653)
                        (6, 2.5864702092411953)
                        (7, 2.924906535721844)
                        (8, 2.4350207543216054)
                        (9, 3.0568723980684362)
                        (10, 3.0925643309892856)
                        (11, 3.557947444055799)
                        (12, 3.1332943729993348)
                        (13, 3.581382137088513)
                        (14, 3.280716746059921)
                        (15, 3.391594767327995)
                        (16, 3.033453402260712)
                        (17, 4.5440597169333605)
                        (18, 3.485172746889333)
                        (19, 4.6929212661048485)
                        (20, 4.009533408253645)
                        (21, 3.7648251844766474)
                        (22, 3.8652159577151077)
                        (23, 4.586690778599297)
                        (24, 4.580433492235934)
                        (25, 3.8496288544604487)
                        (26, 3.7527223578150233)
                        (27, 4.071579066762359)
                        (28, 3.8359914462794684)
                        (29, 4.61516867164132)
                        (30, 4.495103547214446)};

                \legend{Formel (\ref{expectation-tmin}), Gemessene Laufzeit}
            \end{axis}
        \end{tikzpicture}
        \caption{Die durchschnittliche gemessene Laufzeit des Rho-Algorithmus für $M = 1$ und Werte von Formel (\ref{expectation-tmin}) für $1 \le k \le 30$. Beide wurden normalisiert, sodass bei $k = 1$ der Wert 1 ist.}
        \label{fig:measurements-m1}
    \end{figure}

    \paragraph{$\pmb{M = 2}$.} Hier wurden $2^{20}$ zufällige Semiprimzahlen für jedes Paar $1 \le k_1 \le k_2 \le 8$ gewählt. Die Messergebnisse und berechnete Werte zum Vergleich sind in Abbildung \ref{fig:measurements-m2} dargestellt. Auch hier wird das grundsätzliche Verhalten der Laufzeit gut durch die Formeln (\ref{expectation-tmin}) und (\ref{lkkp}) beschreiben. Allerdings ist die tatsächliche Laufzeit immer etwas geringer als von den Formeln vorhergesagt. Insbesondere ergibt sich für $k_1 = 1, \, k_2 = 2$ eine leicht geringere Laufzeit als für $k_1 = k_2 = 1$. Eine Erklärung dafür könnte die Annahme sein, dass nur ein kleinster Primfaktor relevant für die Laufzeit des Algorithmus ist. Oder anders gesagt, dass alle anderen Primfaktoren so groß sind, dass eine Kollision in ihrem funktionalen Graphen vernachlässigbar klein ist. Semiprimzahlen und auch viele andere Zahlen haben aber mehrere Primfaktoren kleinster Größenordnung, sodass sich in mehreren funktionalen Graphen (modulo jedem Primfaktor) die Möglichkeit einer Kollision ergibt. Werte von $k > 1$ könnten für solche Zahlen besser als erwartet sein, da sich für jeden der Primfaktoren die Möglichkeit ergibt, dass $\gcd(p - 1, 2k) > 1$.

\begin{figure}
    \centering
    \small
    \def\colwidth{1.1em}
    \def\spc{0.5em}
    \renewcommand{\arraystretch}{0.1}

    \begin{tabular}{c|p{\colwidth}p{\colwidth}p{\colwidth}p{\colwidth}p{\colwidth}p{\colwidth}p{\colwidth}p{\colwidth}c|p{\colwidth}p{\colwidth}p{\colwidth}p{\colwidth}p{\colwidth}p{\colwidth}p{\colwidth}p{\colwidth}}
          & 1                                  & 2                                  & 3                                  & 4                                  & 5                                  & 6                                  & 7                                  & 8                                  & \space & 1                                  & 2                                  & 3                                  & 4                                  & 5                                  & 6                                  & 7                                  & 8                                  \\
          & \vspace{\spc}                      & \vspace{\spc}                      & \vspace{\spc}                      & \vspace{\spc}                      & \vspace{\spc}                      & \vspace{\spc}                      & \vspace{\spc}                      & \vspace{\spc}                      & \space & \vspace{\spc}                      & \vspace{\spc}                      & \vspace{\spc}                      & \vspace{\spc}                      & \vspace{\spc}                      & \vspace{\spc}                      & \vspace{\spc}                      & \vspace{\spc}                      \\
        \cline{1-18}
          & \vspace{\spc}                      & \vspace{\spc}                      & \vspace{\spc}                      & \vspace{\spc}                      & \vspace{\spc}                      & \vspace{\spc}                      & \vspace{\spc}                      & \vspace{\spc}                      & \space & \vspace{\spc}                      & \vspace{\spc}                      & \vspace{\spc}                      & \vspace{\spc}                      & \vspace{\spc}                      & \vspace{\spc}                      & \vspace{\spc}                      & \vspace{\spc}                      \\
        1 & \textcolor[HTML]{ 0420fa }{ 1.00 } & \textcolor[HTML]{ 0020ff }{ 0.97 } & \textcolor[HTML]{ 0620f8 }{ 1.02 } & \textcolor[HTML]{ 0820f6 }{ 1.03 } & \textcolor[HTML]{ 1320eb }{ 1.11 } & \textcolor[HTML]{ 0920f5 }{ 1.04 } & \textcolor[HTML]{ 1620e8 }{ 1.13 } & \textcolor[HTML]{ 0f20ef }{ 1.08 } & \space & \textcolor[HTML]{ 0020ff }{ 1.00 } & \textcolor[HTML]{ 0520f9 }{ 1.06 } & \textcolor[HTML]{ 0c20f2 }{ 1.12 } & \textcolor[HTML]{ 0d20f1 }{ 1.12 } & \textcolor[HTML]{ 1220ec }{ 1.17 } & \textcolor[HTML]{ 0e20f0 }{ 1.13 } & \textcolor[HTML]{ 1420ea }{ 1.19 } & \textcolor[HTML]{ 1120ed }{ 1.16 } \\
        2 &                                    & \textcolor[HTML]{ 3d20c1 }{ 1.43 } & \textcolor[HTML]{ 1b20e3 }{ 1.17 } & \textcolor[HTML]{ 2420da }{ 1.24 } & \textcolor[HTML]{ 3120cd }{ 1.33 } & \textcolor[HTML]{ 2120dd }{ 1.21 } & \textcolor[HTML]{ 3320cb }{ 1.35 } & \textcolor[HTML]{ 3220cc }{ 1.34 } & \space &                                    & \textcolor[HTML]{ 3c20c2 }{ 1.58 } & \textcolor[HTML]{ 3620c8 }{ 1.51 } & \textcolor[HTML]{ 3720c7 }{ 1.53 } & \textcolor[HTML]{ 4520b9 }{ 1.66 } & \textcolor[HTML]{ 3920c5 }{ 1.55 } & \textcolor[HTML]{ 4d20b1 }{ 1.73 } & \textcolor[HTML]{ 4320bb }{ 1.64 } \\
        3 &                                    &                                    & \textcolor[HTML]{ 73208b }{ 1.82 } & \textcolor[HTML]{ 2c20d2 }{ 1.30 } & \textcolor[HTML]{ 4c20b2 }{ 1.53 } & \textcolor[HTML]{ 3d20c1 }{ 1.43 } & \textcolor[HTML]{ 5220ac }{ 1.58 } & \textcolor[HTML]{ 4120bd }{ 1.45 } & \space &                                    &                                    & \textcolor[HTML]{ 752089 }{ 2.11 } & \textcolor[HTML]{ 5220ac }{ 1.78 } & \textcolor[HTML]{ 6a2094 }{ 2.01 } & \textcolor[HTML]{ 5620a8 }{ 1.81 } & \textcolor[HTML]{ 772087 }{ 2.14 } & \textcolor[HTML]{ 662098 }{ 1.97 } \\
        4 &                                    &                                    &                                    & \textcolor[HTML]{ 83207b }{ 1.94 } & \textcolor[HTML]{ 4c20b2 }{ 1.53 } & \textcolor[HTML]{ 3920c5 }{ 1.39 } & \textcolor[HTML]{ 4f20af }{ 1.56 } & \textcolor[HTML]{ 5220ac }{ 1.58 } & \space &                                    &                                    &                                    & \textcolor[HTML]{ 80207e }{ 2.22 } & \textcolor[HTML]{ 6f208f }{ 2.05 } & \textcolor[HTML]{ 5920a5 }{ 1.85 } & \textcolor[HTML]{ 7d2081 }{ 2.19 } & \textcolor[HTML]{ 6b2093 }{ 2.02 } \\
        5 &                                    &                                    &                                    &                                    & \textcolor[HTML]{ e92015 }{ 2.69 } & \textcolor[HTML]{ 4f20af }{ 1.55 } & \textcolor[HTML]{ 862078 }{ 1.96 } & \textcolor[HTML]{ 6a2094 }{ 1.76 } & \space &                                    &                                    &                                    &                                    & \textcolor[HTML]{ c62038 }{ 2.88 } & \textcolor[HTML]{ 74208a }{ 2.10 } & \textcolor[HTML]{ a92055 }{ 2.61 } & \textcolor[HTML]{ 8e2070 }{ 2.35 } \\
        6 &                                    &                                    &                                    &                                    &                                    & \textcolor[HTML]{ 8c2072 }{ 2.00 } & \textcolor[HTML]{ 5b20a3 }{ 1.64 } & \textcolor[HTML]{ 4920b5 }{ 1.51 } & \space &                                    &                                    &                                    &                                    &                                    & \textcolor[HTML]{ 8c2072 }{ 2.33 } & \textcolor[HTML]{ 83207b }{ 2.25 } & \textcolor[HTML]{ 70208e }{ 2.06 } \\
        7 &                                    &                                    &                                    &                                    &                                    &                                    & \textcolor[HTML]{ ff2000 }{ 2.85 } & \textcolor[HTML]{ 74208a }{ 1.83 } & \space &                                    &                                    &                                    &                                    &                                    &                                    & \textcolor[HTML]{ ff2000 }{ 3.41 } & \textcolor[HTML]{ a4205a }{ 2.56 } \\
        8 &                                    &                                    &                                    &                                    &                                    &                                    &                                    & \textcolor[HTML]{ b3204b }{ 2.30 } & \space &                                    &                                    &                                    &                                    &                                    &                                    &                                    & \textcolor[HTML]{ c82036 }{ 2.90 } \\
    \end{tabular}

    \caption{Die durchschnittliche Laufzeit von Pollards Rho-Algorithmus (links) und berechnete Werte für die erwartete Laufzeit (rechts), für $M = 2$ und $1 \le k_1 \le k_2 \le 8$. $k_1$ ist vertikal aufgetragen, $k_2$ horizontal. Zur Berechnung der Werte rechts wurde Formel (\ref{lkkp}) verwendet, wenn $k_1 = k_2$ und Formel (\ref{expectation-tmin}), wenn $k_1 \ne k_2$. Die Werte wurden jeweils normalisiert, sodass bei $k_1 = k_2 = 1$ Wert 1 ist.}
    \label{fig:measurements-m2}
\end{figure}

\section{Fazit}

Zur Beantwortung der Frage,

\printbibliography

\end{document}