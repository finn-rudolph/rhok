\documentclass[a4paper, 10pt, ngerman]{article}

\usepackage{amsmath}
\usepackage{amssymb}
\usepackage[ngerman]{babel}
\usepackage[backend=biber,style=alphabetic,]{biblatex}
\usepackage[left = 2.5cm, right = 2.5cm, top = 2.5cm, bottom = 2.cm]{geometry}
\usepackage[hidelinks]{hyperref}
\usepackage[onehalfspacing]{setspace}
\usepackage{sectsty}

\allsectionsfont{\sffamily}

\title{\sffamily{\textbf{Parametrisierung von Pollards Rho-Methode}}}
\author{Finn Rudolph}
\date{03.01.2024}

\addbibresource{rhok.bib}
\nocite{*}

\newcommand{\N}{\mathbb{N}}
\newcommand{\Z}{\mathbb{Z}}

\newtheorem{definition}{Definition}

\begin{document}

\maketitle

\section*{Projektüberblick}

\tableofcontents

\section{Zusammenfassung}

\section{Motivation und Fragestellung}

Während für Rekordfaktorisierungen mittlerweile ausschließlich das Zahlkörpersieb verwendet wird, bleibt Pollards Rho-Algorithmus einer schnellsten Algorithmen zur Faktorisierung kleiner Zahlen bis ca. $2^{64}$. Da Leistungssteigerungen bei modernen Computern häufig durch verbesserete Nebenläufigkeit (z. B. durch mehr Prozessorkerne) erzielt werden, ist es eine interessante Frage, wie Pollards Rho-Methode am besten parallel ausgeführt wird. Bevor die Fragestellung präzise gestellt werden kann, soll jedoch Pollards Rho-Methode erklärt werden.

\subsection{Pollards Rho-Methode}

Sei $n$ die zu faktorisierende Zahl und $f : \Z/n\Z \to \Z/n\Z$ mit $f : x \mapsto x^{2k} + 1$ für einen Parameter $1 \le k \in \N$. Man wähle einen Anfangswert $x_0 \in \Z/n\Z$ und betrachte die Folge $(x_n)_{n \in \N}$ definiert durch $x_n = f(x_{n - 1})$ für einen Parameter $1 \le k \in \N$. Da $(x_n)_{n \in \N}$ über der endlichen Menge $\Z/n\Z$ definiert ist, ist die Folge ab einem bestimmten Punkt periodisch. Sei $p$ ein Primfaktor von $n$ und $\pi : \Z/n\Z \to \Z/p\Z$ die natürliche Projektion. Die Idee von Pollards Rho-Algorithmus ist, zwei Folgenglieder $x_i, x_j \in \Z/n\Z$ zu finden, sodass $x_i \ne x_j$ aber $\pi(x_i) = \pi(x_j)$. Dann ist nämlich $\gcd(n, x_i - x_j)$ ein echter Faktor von $n$. Wir nehmen heuristisch an, dass die Periodenlänge von $(x_n)_{n \in \N}$ in $\Z/n\Z$ deutlich länger als die Periodenlänge in $\Z/p\Z$ ist, sodass es ausreicht, $x_i, x_j$ mit $i \ne j$ zu finden, die kongruent modulo $p$ sind. Diese Annahme begründet sich darin, dass für den kleinsten Primfaktor $p \le \sqrt n$ gilt. Zum Finden solcher $x_i, x_j$ betrachten wir den funktionalen Graphen von $\pi(f)$, wobei $\pi(f)$ die Abbildung $f$ projeziert auf $\Z/p\Z$ ist.

\begin{definition}[Funktionaler Graph]
    Sei $X$ eine endliche Menge und $f: X \to X$ eine Abbildung. Der funktionale Graph von $f$, geschrieben $\gamma(f)$, ist der gerichtete Graph mit Knotenmenge $X$ und Kantenmenge $E$, wobei die Kante $(x, y) \in X \times X$ genau dann in $E$ liegt, wenn $f(x) = y$.
\end{definition}

\noindent Es ist leicht zu zeigen, dass jede Zusammenhangskomponente eines funktionalen Graphen aus einem Zyklus und an den Zyklusknoten gewurzelten Bäumen besteht. In $\gamma(f)$ betrachtet startet $(x_n)_{n \in \N}$ mit $x_0$ in einem Baum und ,,läuft`` durch den Graphen, wobei immer die eindeutige von einem Knoten ausgehende Kante entlanggegangen wird. An einem Punkt wird der Zyklus in der Zusammenhangskomponente von $x_0$ betreten, und ab genau diesem Punkt ist $(x_n)_{n \in \N}$ periodisch. Um $x_i, x_j$ mit $i \ne j$ und $\pi(x_i) = \pi(x_j)$ zu finden, wird Floyds Algorithmus zum Finden des Zyklus in der Zusammenhangskomponente von $\pi(x_0)$ in $\gamma(\pi(f))$ verwendet. Floyds Algorithmus macht sich zunutze, dass es ein $r \in \N, 1 \le r \ge 1$ mit $\pi(x)_r = \pi(x)_{2r}$ geben muss. Für das minimale solcher $r$ gilt außerdem $r \le \mu + \lambda$, wobei $\mu$ die Höhe von $x_0$ in seinem Baum und $\lambda$ die Länge des Zyklus ist (Knuth, 2014, S. 7). Indem man die Folgen $(x_n)_{n \in \N}$ und $(x_{2n})_{n \in \N}$ gleichzeitig berechnet, stößt man in maximal $\mu + \lambda$ Schritten auf gewünschte $x_i, x_j$. Das Überprüfen in jedem Schritt, ob $\pi(x_i) = \pi(x_{2i})$ geschieht natürlich nicht explizit, da $p$ unbekannt ist, aber indirekt durch Berechnung von $\gcd(n, x_i - x_{2i})$.

\subsection{Parallelisierung der Rho-Methode}

\section{Die erwartete Rho-Länge in zufälligen Funktionen}

\subsection{Der Fall zweier Startpunkte}

\subsection{Mögliche Methoden für den allgemeinen Fall}

\section{Bestimmung optimaler Exponenten für die Rho-Methode}

\subsection{Der Fall einer Maschine}

\subsection{Der Fall zweier Maschinen}

\section{Experimentelle Ergebnisse}

\section{Fazit}

\printbibliography

\end{document}