\documentclass[a3paper, extrafontsizes, ngerman, 25pt]{memoir}
\usepackage[left=1cm, right=7cm, top=1cm, bottom=1cm]{geometry}


\renewcommand{\familydefault}{\sfdefault}
\newcommand{\Z}{\mathbb{Z}}
\newcommand{\N}{\mathbb{N}}


\begin{document}

\subsection{Pollard's Rho-Methode}

Die zu faktorisierende Zahl wird $n$ genannt. In der Rho-Methode wird zunächst ein Anfangswert $0 \le x_0 < n$ zufällig gewählt. Anschließend wird die Funktion $f(x) = x^{2k} + 1$ wiederholt auf $x_0$ angewandt (hier wird der genannte Parameter $k$ verwendet). In der daraus entstehende Folge an Zahlen $x_0, f(x_0), f(f(x_0)), \dots$ wird jeweils nur der Rest bei Division durch $n$ gespeichert (die Zahlen werden \emph{modulo} $n$ betrachtet).  Wenn nun $p$ ein Primfaktor von $n$ ist und die Folge $x_0, f(x_0),$ $f(f(x_0)), \dots$ modulo $n$ berechnet wird, wird sie auch implizit modulo $p$ berechnet. Die Folge ist natürlich nicht wirklich modulo $p$ bekannt, da $p$ als Primfaktor von $n$ unbekannt ist. Mit der impliziten Berechnung modulo $p$ ist gemeint, dass wenn man ein Folgenglied, das modulo $n$ gespeichert wird, modulo $p$ betrachtet, man den gleichen Rest erhält, wie wenn man die Folge von Beginn an modulo $p$ berechnet hätte.

Die Idee der Rho-Methode ist es, dass die Folge modulo $p$ wesentlich früher einen Wert zum zweiten Mal annehmen wird, da $p$ deutlich kleiner als $n$ ist. Das $i$-te Folgenglied von $x_0, f(x_0), \dots$ modulo $n$ wird mit $x_i$ bezeichnet. Haben beispielsweise $x_i$ und $x_j$ ($i \ne j$) den gleichen Rest modulo $p$ aber nicht modulo $n$, ist $\gcd(x_i - x_j, n)$ ein echter Faktor von $n$, wobei $\gcd$ der größte gemeinsame Teiler ist. Das Problem reduziert sich also darauf, $x_i$ und $x_j$ mit gleichem Rest modulo $p$ zu finden.

Zum Finden solcher $x_i, x_j$ ist es hilfreich, den funktionalen Graphen von $f$ zu betrachten. Der funktionale Graph von $f$ enthält alle Reste $0, 1, \dots p - 1$ modulo $p$ als Knoten und eine Kante von $a$ nach $b$, wenn $f(a) = b$ modulo $p$. Die Folgenglieder von $x_0, f(x_0), f(f(x_0)), \dots$ sind also genau die Knoten, die man besucht, wenn man in dem funktionalen Graphen bei $x_0$ startet und immer die ausgehende Kante von jedem Knoten nimmt. Dann reduziert sich das Finden von $x_i, x_j$ darauf, einen Zyklus in dem Graphen zu finden. Denn ein Knoten im Zyklus wird bei der Berechnung der Folge wieder besucht, wird er also beispielsweise im $i$-ten und $j$-ten Schritt besucht, haben $x_i$ und $x_j$ den gleichen Rest modulo $p$. Der Algorithmus zum Finden eines Zyklus (Floyd's Algorithmus) ist in der Animation veranschaulicht.

\end{document}