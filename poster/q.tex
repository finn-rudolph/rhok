\documentclass[a1paper, portrait, 25pt, innermargin=2cm,blockverticalspace=0cm]{tikzposter}

\usepackage{o}
\usepackage{pgfplots}
\usepackage{svg}
\usetikzlibrary{arrows, decorations.markings}

\pgfplotsset{
  every axis plot/.append style={line width=\lw},
  every axis plot post/.append style={
    every mark/.append style={line width=6pt}
  }
}

\begin{document}

\maketitle

% \draw[fill = c1, draw = none] (-50, 17) rectangle ++(100, 100);
\draw[fill = c1, draw = none] (-50, -30) rectangle ++(100, -100);

\useblockstyle{framedblock}

\begin{columns}
  \column{0.5}
  \block{}{
    \color{white}
    \begin{center}
      \LARGE Eine Maschine \\
      \vspace{1.64em}
      \textcolor{c4}{\textbf{\Huge{$\boldsymbol{k = 1}$ ist optimal}}}
      \vspace{1.2em}
    \end{center}
    In dem Beweis dafür wurde die Formel für die Laufzeit zunächst bei $M = k = 1$ ausgewertet. Anschließend wurde eine untere Schranke für $M = 1$ und $k > 1$ bestimmt und damit die Optimalität von $k = 1$ bewiesen. Die unten dargestellten Laufzeitmessungen unterstützen die Aussage.
  }
  \column{0.5}
  \block{}{
    \color{white}
    \begin{center}
      \vspace{-0.3em}
      \LARGE Zwei Maschinen \\
      \vspace{1em}
      \textcolor{c5}{\huge{\textbf{$\boldsymbol{k_1 = k_2 = 1}$ ist besser als $\boldsymbol{k_1 = k_2 > 1}$ und $\boldsymbol{k_1, k_2}$ prim}}} \\
      \vspace{1em}
    \end{center}
    Im Fall zweier Primzahlen und zweier gleicher Zahlen für $k$ bei beiden Maschinen war eine Abschätzung der Formel nach unten möglich. Laufzeitmessungen bestätigen das Ergebnis und deuten darauf hin, dass $k_1 = k_2 = 1$ optimal ist.
  }
\end{columns}

\useblockstyle{bstyle}

\block{
  \vspace{1em}
  \huge{Laufzeitmessungen bestätigen die Korrektheit der Formel.}
}{
  \color{c1}
  \vspace{-0.6em}
  \centering
  Hier wird die Formel mit der tatsächlichen Laufzeit verglichen für...
}
\begin{columns}
  \column{0.33}
  \block{}{
    \vspace{-2.2em}
    \centering
    \includesvg[width=50pt]{cpu-icon} \\
    \color{c1}
    \bf
    1 Maschine
  }
  \column{0.33}
  \block{}{
    \vspace{-2.2em}
    \centering
    \LARGE
    \vspace{0.3cm}
    $1 \le k \le 48$ \\
    \vspace{0.4cm}
    \large
    \color{c1}
    \bf
    Werte von $\boldsymbol{k}$ zwischen 1 und 48
  }
  \column{0.33}
  \block{}{
    \vspace{-2.2em}
    \centering
    \color{c1}
    \includesvg[width=50pt]{experiment-icon} \\
    \bf
    16384 Testfälle
  }
\end{columns}

\block{}
{
  \vspace{-2em}
  \def\firstcolor{c3}
  \def\secondcolor{c5}
  \color{c1}

  \begin{tikzpicture}
    \begin{axis}[
        axis lines=left,
        axis line style = {line width = \lw},
        xmin = 1,
        xmax = 48,
        xlabel = { $k$},
        x label style = {at={(axis description cs:0.97,0.12)},anchor=north},
        xtick = {1, 8, 16, 24, 32, 40, 48},
        ytick = {1, 2, 3, 4, 5, 6},
        width = 1450pt,
        height = 700pt,
      ]

      \addplot[color=\firstcolor,mark=*]
      coordinates{
          (1, 1.0)
          (2, 1.5773502691896257)
          (3, 2.1086517918741277)
          (4, 2.21648605664914)
          (5, 2.8790043489023804)
          (6, 2.332272327437964)
          (7, 3.414299970503978)
          (8, 2.89543195056782)
          (9, 3.3067332978837425)
          (10, 2.970093877947532)
          (11, 4.142494904167608)
          (12, 2.7949482208102387)
          (13, 4.411181889332409)
          (14, 3.4111279805567123)
          (15, 3.481018974295858)
          (16, 3.594729442047154)
          (17, 4.840295239186588)
          (18, 3.2665648302249286)
          (19, 5.017128905823505)
          (20, 3.426648342312123)
          (21, 3.952865384732847)
          (22, 4.006284568144385)
          (23, 5.319207262349245)
          (24, 3.3342189545431045)
          (25, 4.848341222073144)
          (26, 4.224515025049569)
          (27, 4.541184581320172)
          (28, 3.8606214340939586)
          (29, 5.682737117278753)
          (30, 3.3580839014528188)
          (31, 5.786717613861166)
          (32, 4.303622115889641)
          (33, 4.585058767874234)
          (34, 4.57165798930078)
          (35, 4.768544753994481)
          (36, 3.65286546982577)
          (37, 6.061272505622063)
          (38, 4.714254318317245)
          (39, 4.815669489895146)
          (40, 3.9872807230736402)
          (41, 6.219718898826906)
          (42, 3.7480426730792065)
          (43, 6.293026650584624)
          (44, 4.440755450349039)
          (45, 4.475806923989725)
          (46, 4.957283888509379)
          (47, 6.429590774020685)
          (48, 3.903718627468377)
        };

      \addplot[color=\secondcolor,mark=*]
      coordinates {
          (1 ,   1)
          (2 ,   1.2688871624076672)
          (3 ,   1.5448360772795062)
          (4 ,   1.6215294878078745)
          (5 ,   2.1260552548319587)
          (6 ,   1.7210271616473245)
          (7 ,   2.5083228373472504)
          (8 ,   2.006075780278468)
          (9 ,   2.2840987997737323)
          (10,   2.227057271909877)
          (11,   3.0667905710118877)
          (12,   2.02973927462346)
          (13,   3.1096795366509973)
          (14,   2.554240766520122)
          (15,   2.605128864950923)
          (16,   2.4065268921363367)
          (17,   3.3463078632578274)
          (18,   2.3995424079135566)
          (19,   3.636437145493917)
          (20,   2.540642287388499)
          (21,   2.798891347254288)
          (22,   3.0196492908359134)
          (23,   3.962603150398965)
          (24,   2.3707754814488204)
          (25,   3.263084687726768)
          (26,   3.055858996661355)
          (27,   3.1536357812541125)
          (28,   2.8452919940049712)
          (29,   4.002758086162577)
          (30,   2.7051029645392757)
          (31,   4.281180949475698)
          (32,   2.798678843877791)
          (33,   3.0338163972786867)
          (34,   3.236647402168213)
          (35,   3.456162662926002)
          (36,   2.6958680662121215)
          (37,   4.206046640838136)
          (38,   3.480612316774627)
          (39,   3.530674663584416)
          (40,   2.894119424906177)
          (41,   4.195187693016042)
          (42,   2.8835503555481226)
          (43,   4.500107134645032)
          (44,   3.324245578471752)
          (45,   3.2505677313318966)
          (46,   3.7547741175937004)
          (47,   4.7967391294287784)
          (48,   2.7134203084992854)};

      \node[align=left] (F) at (1450, 470) {\color{\firstcolor} theoretische Werte \\ \color{\firstcolor} der Formel};
      \node[align=left] (M) at (3800, 70) {\color{\secondcolor} gemessene Laufzeit \\ \color{\secondcolor} der Rho-Methode};
      \node[align=left] (L) at (300, 470) {\color{c1} Laufzeit \\ \color{c1}  relativ \\ \color{c1} zu $k = 1$};

      \draw[dashed, draw = \firstcolor, line width = \lw] (1450, 433) to (1600, 380);
      \draw[dashed, draw = \secondcolor, line width = \lw] (3700, 110) to (3550, 230);
    \end{axis}
  \end{tikzpicture}
}

\block{}{
  \vspace{-1.8cm}
  \centering
  \LARGE{\textbf{\textcolor{c3}{Ähnlicher Kurvenverlauf $\Longrightarrow$ Die Formel beschreibt die Laufzeit gut.}}}
  \vspace{0.35cm}
}

\begin{columns}
  \column{0.7}
  \block{}{
    \raggedleft
    \vspace{80pt}
    \Huge
    \color{white}
    \textbf{Link zur schriftlichen Arbeit $\boldsymbol{\to}$}
  }
  \column{0.3}
  \block{}{
    \hfill
    \includesvg[width=220pt]{rhok-qr}
  }
\end{columns}


\end{document}