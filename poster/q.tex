\documentclass[a1paper, portrait, 25pt, innermargin=2cm]{tikzposter}

\usepackage{o}
\usepackage{pgfplots}
\usepackage{svg}
\usetikzlibrary{arrows, decorations.markings}

\pgfplotsset{
  every axis plot/.append style={line width=\lw},
  every axis plot post/.append style={
    every mark/.append style={line width=6pt}
  }
}

\begin{document}

\usetitlestyle{Zero}
\maketitle

\draw[fill = darkweakcolor, draw = none] (-50, 21.5) rectangle ++(100, 100);
\draw[fill = darkweakcolor, draw = none] (-50, -30) rectangle ++(100, -100);

\begin{columns}
  \column{0.5}
  \block{\huge{\textcolor{white}{\textbf{Das optimale $\boldsymbol{k}$ ist...}}}}{
    \begin{center}
      \textcolor{white}{\Huge{$\boldsymbol{k = 1}$}}
    \end{center}
    \vspace{2em}
    \color{white}
    bei einer Maschine. In dem Beweis dafür wurde die Formel für die Laufzeit bei $M = k = 1$ ausgewertet und anschließend die Formel für $M = 1$ und $k > 1$ nach unten abgeschätzt.
  }

  \column{0.5}
  \block{}{
    \vspace{2.97em}
    \begin{center}
      \textcolor{white}{\Huge{{$\boldsymbol{k_1 = k_2 = 1}$}}}
    \end{center}
    \vspace{2em}
    \color{white}
    (vermutlich) bei zwei Maschinen. Mit der Formel wurde bewiesen, dass $k_1 = k_2 = 1$ besser ist als zwei Primzahlen oder zwei gleiche Zahlen größer als 1. Auswertungen der Formel und Laufzeitmessungen suggerieren, das $k_1 = k_2 = 1$ tatsächlich optimal ist.
    \vspace{1.7em}
  }
\end{columns}

\block{\huge{Laufzeitmessungen bestätigen die Korrektheit der Formel.}}{
  Hier wird die Formel mit der tatsächlichen Laufzeit verglichen für...
}
\begin{columns}
  \column{0.33}
  \block{}{
    \centering
    \includesvg[width=50pt]{cpu-icon} \\
    1 Maschine
  }
  \column{0.33}
  \block{}{
    \centering
    \LARGE
    \vspace{0.3cm}
    $1 \le k \le 48$ \\
    \vspace{0.4cm}
    \large
    Werte von $k$ zwischen 1 und 48
  }
  \column{0.33}
  \block{}{
    \centering
    \includesvg[width=50pt]{experiment-icon} \\
    16384 Testfälle
  }
\end{columns}

\block{}
{
  \begin{tikzpicture}
    \begin{axis}[
        axis lines=left,
        axis line style = {line width = \lw},
        xmin = 1,
        xmax = 48,
        xlabel = { $k$},
        x label style = {at={(axis description cs:0.97,0.12)},anchor=north},
        xtick = {1, 8, 16, 24, 32, 40, 48},
        ytick = {1, 2, 3, 4, 5, 6},
        width = 1450pt,
        height = 700pt,
      ]

      \addplot[color=darkstrongcolor,mark=*]
      coordinates{
          (1, 1.0)
          (2, 1.5773502691896257)
          (3, 2.1086517918741277)
          (4, 2.21648605664914)
          (5, 2.8790043489023804)
          (6, 2.332272327437964)
          (7, 3.414299970503978)
          (8, 2.89543195056782)
          (9, 3.3067332978837425)
          (10, 2.970093877947532)
          (11, 4.142494904167608)
          (12, 2.7949482208102387)
          (13, 4.411181889332409)
          (14, 3.4111279805567123)
          (15, 3.481018974295858)
          (16, 3.594729442047154)
          (17, 4.840295239186588)
          (18, 3.2665648302249286)
          (19, 5.017128905823505)
          (20, 3.426648342312123)
          (21, 3.952865384732847)
          (22, 4.006284568144385)
          (23, 5.319207262349245)
          (24, 3.3342189545431045)
          (25, 4.848341222073144)
          (26, 4.224515025049569)
          (27, 4.541184581320172)
          (28, 3.8606214340939586)
          (29, 5.682737117278753)
          (30, 3.3580839014528188)
          (31, 5.786717613861166)
          (32, 4.303622115889641)
          (33, 4.585058767874234)
          (34, 4.57165798930078)
          (35, 4.768544753994481)
          (36, 3.65286546982577)
          (37, 6.061272505622063)
          (38, 4.714254318317245)
          (39, 4.815669489895146)
          (40, 3.9872807230736402)
          (41, 6.219718898826906)
          (42, 3.7480426730792065)
          (43, 6.293026650584624)
          (44, 4.440755450349039)
          (45, 4.475806923989725)
          (46, 4.957283888509379)
          (47, 6.429590774020685)
          (48, 3.903718627468377)
        };

      \addplot[color=lightstrongcolor,mark=*]
      coordinates {
          (1 ,   1)
          (2 ,   1.2688871624076672)
          (3 ,   1.5448360772795062)
          (4 ,   1.6215294878078745)
          (5 ,   2.1260552548319587)
          (6 ,   1.7210271616473245)
          (7 ,   2.5083228373472504)
          (8 ,   2.006075780278468)
          (9 ,   2.2840987997737323)
          (10,   2.227057271909877)
          (11,   3.0667905710118877)
          (12,   2.02973927462346)
          (13,   3.1096795366509973)
          (14,   2.554240766520122)
          (15,   2.605128864950923)
          (16,   2.4065268921363367)
          (17,   3.3463078632578274)
          (18,   2.3995424079135566)
          (19,   3.636437145493917)
          (20,   2.540642287388499)
          (21,   2.798891347254288)
          (22,   3.0196492908359134)
          (23,   3.962603150398965)
          (24,   2.3707754814488204)
          (25,   3.263084687726768)
          (26,   3.055858996661355)
          (27,   3.1536357812541125)
          (28,   2.8452919940049712)
          (29,   4.002758086162577)
          (30,   2.7051029645392757)
          (31,   4.281180949475698)
          (32,   2.798678843877791)
          (33,   3.0338163972786867)
          (34,   3.236647402168213)
          (35,   3.456162662926002)
          (36,   2.6958680662121215)
          (37,   4.206046640838136)
          (38,   3.480612316774627)
          (39,   3.530674663584416)
          (40,   2.894119424906177)
          (41,   4.195187693016042)
          (42,   2.8835503555481226)
          (43,   4.500107134645032)
          (44,   3.324245578471752)
          (45,   3.2505677313318966)
          (46,   3.7547741175937004)
          (47,   4.7967391294287784)
          (48,   2.7134203084992854)};

      \node[align=left] (F) at (1450, 470) {\color{darkstrongcolor} theoretische Werte \\ \color{darkstrongcolor} der Formel};
      \node[align=left] (M) at (3800, 70) {\color{lightstrongcolor} gemessene Laufzeit \\ \color{lightstrongcolor} der Rho-Methode};
      \node[align=left] (L) at (300, 470) {Laufzeit \\ relativ \\ zu $k = 1$};

      \draw[dashed, draw = darkstrongcolor, line width = \lw] (1450, 433) to (1600, 380);
      \draw[dashed, draw = lightstrongcolor, line width = \lw] (3700, 110) to (3550, 230);
    \end{axis}
  \end{tikzpicture}
}

\block{}{
  \vspace{-1.2cm}
  \LARGE{\textbf{Ähnlicher Kurvenverlauf $\Longrightarrow$ Die Formel beschreibt die Laufzeit gut.}}
  \vspace{3.7cm}
}

\begin{columns}
  \column{0.8}
  \block{}{
    \raggedleft
    \vspace{160pt}
    \LARGE
    \color{white}
    \textbf{Link zur schriftlichen Arbeit}
  }
  \column{0.2}
  \block{}{
    \includesvg[width=220pt]{rhok-qr}
  }
\end{columns}

\end{document}