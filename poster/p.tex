\documentclass[a2paper, landscape, 25pt]{tikzposter}

\usepackage{o}

\begin{document}

\maketitle

\draw[fill = c1, draw = none] (-50, 10) rectangle ++(100, 100);

\block{\vspace{-0.7em}\Huge\color{white} Wie faktorisiert man \textcolor{c4}{Zahlen} am schnellsten mit Pollards \textcolor{c5}{Rho-Methode}? \vspace{9em}}

\tikzset{every node/.style={text = white, draw = none, inner sep = 0.8cm, rounded corners = 0.5cm}}

\node[align = center, fill = c2] (PRM) at (-10, 7) {\bf \Large  Pollards Rho-Methode};
\node[align = center, fill = c3] (ZVZ) at (-16, 1) {\bf ...ist ein Algorithmus \\ \bf zur Zerlegung von Zahlen \\ \bf in ihre Primfaktoren};
\node[align = center, fill = c2] (AC) at (-20, -6) {\bf hat Anwendung in \\ \bf der Kryptographie};
\node[align = center, fill = c3] (MMG) at (-4, 0) {\bf kann auf mehreren Computern \\ \bf gleichzeitig ausgeführt werden};


\node[align = center, fill = c3] (PK) at (10, 7) {\bf \Large Der Parameter $\boldsymbol{k}$};
\node[align = center, fill = c2] (SG) at (4, 2) {\bf kann für jeden Computer \\ \bf separat gewählt werden};
\node[align = center, fill = c3] (EXP) at (10, 5) {\bf bestimmt den Exponenten \\ \bf der Funktion $\boldsymbol{f(x) = x^{2k} + 1}$.};
\node[align = center, fill = c2] (VIS) at (14, -2) {};

\tikzset{every node/.style={draw=none, fill=none, text = c1}}

\def\mcolor{c5}
\def\kicolor{c3}
\def\pcolor{c2}

\begin{columns}
    \column{0.382}
    \block{}{
        \color{c3}
        \LARGE \bf
        Um das herauszufinden, habe ich folgende Formel für die Laufzeit bestimmt:
    }
    \column{0.618}
    \block{}{
        \color{c1}
        \Large
        \begin{align*}
            \sqrt{\pi \textcolor{\pcolor}{p} / 2} \ \Bigg ( \sum_{i = 1}^{\textcolor{\mcolor}{M}}
            \frac {\gcd(\textcolor{\pcolor}{p} - 1, 2\textcolor{\kicolor}{k_i}) - 1} {\lg^2 2\textcolor{\kicolor}{k_i}} \Bigg )^{-1/2}
        \end{align*}
    }
\end{columns}

\node[align = center, text = \mcolor] (MLABEL) at (3, -10.8) {\bf $\boldsymbol{M =}$ Anzahl Maschinen};
\node[align = center, text = \kicolor] (KILABEL) at (17, -11) {\bf $\boldsymbol{k_i =}$ Wert von $\boldsymbol{k}$ für die $\boldsymbol{i}$-te Maschine};
\node[align = center, text = \pcolor] (PLABEL) at (7, -19) {\bf $\boldsymbol{p =}$ Primfaktor der zu faktorisierenden Zahl};

\end{document}
