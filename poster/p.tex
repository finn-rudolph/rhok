\documentclass[a2paper, landscape, 25pt]{tikzposter}

\usepackage{o}

\begin{document}

\maketitle

\draw[fill = c1, draw = none] (-50, 9.5) rectangle ++(100, 100);

\block{\vspace{-0.7em}\Huge\color{white} Wie faktorisiert man Zahlen am schnellsten mit Pollards Rho-Methode?\vspace{8em}}

\node[align = center] (PRM) at (-10, 8) {Pollards Rho-Methode};
\node[align = center] (ZVZ) at (-14, 3) {ist ein Algorithmus zur Zerlegung von \\ Zahlen in ihre Primfaktoren};
\node[align = center] (AC) at (-16, -4) {hat Anwendung in der Kryptographie};
\node[align = center] (MMG) at (-4, 0) {kann auf mehreren Computern gleichzeitig ausgeführt werden};


\node[align = center] (PK) at (10, 8) {Der Parameter $k$};
\node[align = center] (SG) at (4, 2) {kann für jeden Computer separat gewählt werden};
\node[align = center] (EXP) at (10, 5) {bestimmt den Exponenten der Funktion $\boldsymbol{f(x) = x^{2k} + 1}$.};
\node[align = center] (VIS) at (14, -2) {};

\begin{columns}
    \column{0.382}
    \block{}{
        \color{c3}
        \LARGE \bf
        Um das herauszufinden, habe ich folgende Formel für die Laufzeit bestimmt:
    }
    \column{0.618}
    \block{}{
        \color{c1}
        \LARGE
        \begin{align*}
            \sqrt{\pi p / 2} \ \Bigg ( \sum_{i = 1}^M
            \frac {\gcd(p - 1, 2k_i) - 1} {\lg^2 2k_i} \Bigg )^{-1/2}
        \end{align*}
    }
\end{columns}

\end{document}
