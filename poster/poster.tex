\documentclass[a4paper, extrafontsizes, ngerman, 25pt]{memoir}
\usepackage[left=1cm, right=2cm, top=1cm, bottom=1cm]{geometry}
\usepackage{amssymb}
\usepackage{amsmath}
\usepackage{svg}
\usepackage{float}

\renewcommand{\familydefault}{\sfdefault}
\newcommand{\Z}{\mathbb{Z}}
\newcommand{\N}{\mathbb{N}}

\begin{document}
\subsection{Einführung}

Pollards Rho-Methode ist einer der schnellsten Algorithmen zur Primfaktorzerlegung kleiner Zahlen. Bei der Implementierung des Algorithmus kann ein Parameter $k$ gewählt werden, der unter Umständen großen Einfluss auf die Laufzeit hat, sowohl im positiven als auch im negativen Sinn. In dieser Arbeit wurde untersucht, wie $k$ bestmöglich gewählt wird, wenn der Algorithmus auf mehreren Maschinen gleichzeitig ausgeführt wird.

\newpage

\subsection{Vorgehen}

Das Ziel war es, durch eine geeignete mathematische Beschreibung des Rho-Algorithmus eine Formel für die Laufzeit in Abhängigkeit der Werte von $k$ für jede Maschine herzuleiten. Es wurde wie üblich für den Rho-Algorithmus angenommen, dass sich $f(x) = x^{2k} + 1$ wie eine zufällige Funktion modulo $p$ verhält, sodass der Algorithmus wahrscheinlichkeitsbasiert analysiert werden konnte. Damit sollten die optimalen Werte für $k$ bestimmt werden.

\vspace{0.5cm}
\noindent \textbf{Abhängige und unabhängige Maschinen.} Eine Schwierigkeit bei der Analyse war, dass zwei Maschinen mit gleichem $k$ stochastisch abhängig sind. Der Fall stochastisch unabhängiger Maschinen, in dem nur paarweise verschiedene Werte von $k$ möglich sind, wurde daher vom Fall abhängiger Maschinen unterschieden. Im Fall unabhängiger Maschinen wurde folgende Formel als asymptotische Näherung für die Laufzeit gefunden:

\begin{figure}[H]
    \hspace{3.3cm} \includesvg[height=70pt]{formula-indep}
\end{figure}

\noindent $M$ bezeichnet hier die Anzahl an Maschinen und $k_i$ den Wert von $k$ für die $i$-te Maschine.

Der Fall abhängiger Maschinen konnte für $M = 2$ gelöst werden. Mithilfe erzeugender Funktionen und einiger Fallunterscheidungen wurde folgende Formel bestimmt.

\begin{figure}[H]
    \hspace{3.7cm}
    \includesvg[height=60pt]{formula-dep2}
\end{figure}

\noindent \textbf{Laufzeitmessungen.} Es wurden Laufzeitmessungen für eine und zwei Maschinen durchgeführt, um zu zeigen, dass die hergeleiteten Formeln das Laufzeitverhalten des Algorithmus gut beschreiben. Dafür wurden für verschiedene Werte von $k$ jeweils $2^{20}$ zufällige 62-Bit Zahlen mit einem 21-Bit und einem 41-Bit Faktor gewählt und die Zeit gemessen, bis ein Faktor gefunden wurde. Die Ergebnisse sind in Abbildung 1 und 2 dargestellt.

\subsection{Ergebnisse}

Für den Fall einer Maschine konnte unter vereinfachenden Annahmen mithilfe obiger Formeln gezeigt werden, dass $k = 1$ optimal ist. Im Fall zweier Maschinen wurde gezeigt, dass für beide Maschinen $k = 1$ zu wählen besser ist, als wenn $k$ bei beiden Maschinen gleich ist oder jeweils eine Primzahl ist. Es wird vermutet und von Experimenten bestätigt, dass $k = 1$ für beide Maschinen optimal ist.

\end{document}