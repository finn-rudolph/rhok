\documentclass[a4paper, extrafontsizes, ngerman, 25pt]{memoir}
\usepackage[left=1cm, right=1cm, top=1cm, bottom=1cm]{geometry}
\usepackage{amssymb}
\usepackage{amsmath}


\renewcommand{\familydefault}{\sfdefault}
\newcommand{\Z}{\mathbb{Z}}
\newcommand{\N}{\mathbb{N}}



\begin{document}
\subsection{Einführung}

Pollards Rho-Methode ist einer der schnellsten Algorithmen zur Primfaktorzerlegung kleiner Zahlen. Bei der Implementierung des Algorithmus kann ein Parameter $k$ gewählt werden, der unter Umständen großen Einfluss auf die Laufzeit hat, sowohl im positiven als auch im negativen Sinn. In dieser Arbeit wurde untersucht, wie $k$ bestmöglich gewählt wird. Es wurde insbesondere der Fall betrachtet, wenn der Rho-Algorithmus auf mehreren Maschinen gleichzeitig ausgeführt wird. Hier kann $k$ für jede Maschine separat gewählt werden.

\newpage

\subsection{Pollard's Rho-Methode}

Die zu faktorisierende Zahl wird $n$ genannt. In der Rho-Methode wird zunächst ein Anfangswert $0 \le x_0 < n$ zufällig gewählt. Der Paramter $k$ kommt ins Spiel, da nun die Funktion $f(x) = x^{2k} + 1$ wiederholt auf $x_0$ angewandt wird. Das erzeugt eine Folge an Zahlen $x_0, f(x_0), f(f(x_0)), \dots$. Es wird jeweils nur der Rest bei Division durch $n$ betrachtet (die Zahlen werden \emph{modulo} $n$ betrachtet). Wenn nun $p$ ein Primfaktor von $n$ ist und die Folge $x_0, f(x_0),$ $f(f(x_0)), \dots$ modulo $n$ berechnet wird, wird sie auch implizit modulo $p$ berechnet. Die Folge ist natürlich nicht wirklich modulo $p$ bekannt, da $p$ als Primfaktor von $n$ unbekannt ist. Aber nimmt man den Rest der Folgenglieder modulo $n$ und betrachtet diesen modulo $p$, erhält man das gleiche Ergebnis, wie wenn man Folge von Beginn an modulo $p$ berechnet hätte.

Die Idee der Rho-Methode ist es, dass die Folge modulo $p$ wesentlich früher einen Wert zweimal annehmen wird, da $p$ deutlich kleiner als $n$ ist. dass  zwei Folgenglieder $x_i, x_j \in \Z/n\Z$ zu finden, sodass $x_i \ne x_j$ aber $\pi(x_i) = \pi(x_j)$. Dann ist nämlich $\gcd(n, x_i - x_j)$ ein echter Faktor von $n$. Das Ereignis, dass eine Folge einen Wert zweimal annimmt, wird eine \emph{Kollision} in dieser Folge genannt. Nimmt man an, dass die Periodenlänge von $(x_i)_{i \in \N}$ in $\Z/n\Z$ deutlich länger als die Periodenlänge in $\Z/p\Z$ ist, reicht es aus, $x_i, x_j$ mit $i \ne j$ zu finden, die kongruent modulo $p$ sind. Die Annahme ist plausibel, weil für den kleinsten Primfaktor $p \le \sqrt n$ gilt, es also deutlich weniger mögliche Werte für $\pi(x_i)$ als $x_i$ gibt. Im Folgenden ist $p$ immer der kleinste Primfaktor von $n$. Außerdem wird angenommen, dass die anderen Primfaktoren von $n$ so viel größer als $p$ sind, dass die Wahrscheinlichkeit einer Kollision modulo eines anderen Primfaktors vernachlässigbar gering ist. Zum Finden solcher $x_i, x_j$ ist es hilfreich, den funktionalen Graphen von $h$ zu betrachten.


\subsection{Vorgehen}

Ziel: mathematische Beschreibung der Laufzeit in Abh. von k, Ableiten optimaler Werte daraus + Experimente zur Bestätigung
Beschreibung der betrachteten Fälle, Probleme, etc.

\newpage

\subsection{Ergebnisse}

\end{document}