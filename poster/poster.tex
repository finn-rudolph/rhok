\documentclass[a4paper, extrafontsizes, ngerman, 25pt]{memoir}
\usepackage[left=1cm, right=2cm, top=1cm, bottom=1cm]{geometry}
\usepackage{amssymb}
\usepackage{amsmath}
\usepackage{svg}
\usepackage{float}

\renewcommand{\familydefault}{\sfdefault}
\newcommand{\Z}{\mathbb{Z}}
\newcommand{\N}{\mathbb{N}}

\begin{document}
\subsection{Einführung}

Pollards Rho-Methode ist einer der schnellsten Algorithmen zur Primfaktorzerlegung kleiner Zahlen. Bei der Implementierung des Algorithmus kann ein Parameter $k$ gewählt werden, der unter Umständen gro-ßen Einfluss auf die Laufzeit hat, sowohl im positiven als auch im negativen Sinn. In dieser Arbeit wurde untersucht, wie $k$ bestmöglich gewählt wird. Es wurde insbesondere der Fall betrachtet, wenn der Algorithmus auf mehreren Maschinen (Computern) gleichzeitig ausgeführt wird. Hier kann $k$ für jede Maschine separat gewählt werden.

\newpage

\subsection{Vorgehen}

Das Ziel war es, durch eine geeignete mathematische Beschreibung des Rho-Algorithmus eine Formel für die mittlere Laufzeit in Abhängigkeit der Werte von $k$ für jede Maschine herzuleiten. Es wurde wie üblich für den Rho-Algorithmus angenommen, dass sich $f(x) = x^{2k} + c$ für $c \ne 0, -2$ wie eine zufällige Funktion modulo $p$ verhält, sodass der Algorithmus wahrscheinlichkeitsbasiert analysiert werden konnte. Die Laufzeit des Rho-Algorithmus wird vor allem von der Anzahl an Iterationen bestimmt, bis ein Faktor gefunden wird. Diese ist von der Struktur des funktionalen Graphen von $f$ und damit von $k$ abhängig, im Allgemeinen hat der Graph bei größerem $k$ eine bessere Struktur. Die Dauer einer Iteration steigt aber um $\log_2 2k$ im Verhältnis zu $k = 1$ wegen der Berechnung von $x^{2k}$.

\vspace{0.5cm}
\noindent \textbf{Eine Formel für die Laufzeit.} Der Wert von $c$ in $f$ wird für jede Maschine zufällig gewählt, um sicherzustellen, dass die Maschinen unabhängig von-einander sind. Denn verwenden zwei Maschinen den gleichen Wert von $k$ und $c$, bewegen sie sich im gleichen funktionalen Graphen. Die dadurch entstehen-de Abhängigkeit sorgt für eine schlechtere Laufzeit und erschwert die Analyse. Wenn $c$ zufällig und unabhängig für jede Maschine gewählt wird, kann jedoch angenommen werden, dass die Maschinen alle unabhängig voneinander sind. Unter dieser Annahme konnte folgende Formel für die Laufzeit der Rho-Methode bestimmt werden.

\begin{figure}[H]
    \centering
    \includesvg[height=70pt]{running-time-formula}
\end{figure}

\noindent $M$ bezeichnet hier die Anzahl an Maschinen und $k_i$ den Wert von $k$ für die $i$-te Maschine.

\vspace{0.5cm}
\noindent \textbf{Bestimmung optimaler Werte von $\boldsymbol{k}$.} Im Fall $M = 1$ wurde obige Formel für $k = 1$ ausgewertet und eine untere Schranke für $k > 1$ bestimmt. So konnte gezeigt werden, dass bei $k = 1$ der minimale Wert angenommen wird. Im Fall $M = 2$ wurde ähnlich vorgegangen, um eine untere Schranke zu erhalten, wenn $k_1$ und $k_2$ prim oder gleich sind.

\vspace{2.3cm}
\subsection{Laufzeitmessungen}

Für $M = 1$ und $M = 2$ wurden die berechneten Werte für die Laufzeit mit Laufzeitmessungen verglichen. Es wurden für verschiedene Werte von $k$ jeweils $2^{14}$ zufällige 192-Bit Zahlen gewählt, bei denen jeder Primfaktor größer als $2^{22}$ ist und mindestens ein Primfaktor kleiner $2^{23}$. Die Größe der Primfaktoren nach unten zu beschränken ist sinnvoll, da die Laufzeit der Rho-Methode sonst sehr kurz wäre und die Messergebnisse durch Dinge wie den Aufwand der Laufzeitmessung selbst verfälscht werden würden. Es ist auch sinnvoll, dass die Testzahlen einen kleinen Primfaktor enthalten, da die Laufzeit sonst sehr groß wäre. Die Ergebnisse sind in den Abbildungen 1 und 2 dargestellt.

\vspace{2.3cm}
\subsection{Ergebnisse}

Im Fall einer Maschine konnte unter vereinfachenden Annahmen gezeigt werden, dass $k = 1$ optimal ist. Im Fall zweier Maschinen wurde gezeigt, dass für beide Maschinen $k = 1$ zu wählen besser ist, als wenn $k$ bei beiden Maschinen gleich ist oder jeweils eine Primzahl ist. Es wird vermutet und von Experimenten bestätigt, dass $k = 1$ für beide Maschinen optimal ist. In weiteren Untersuchungen könnte der Fall $M = 2$ vollständig geklärt werden oder $M \ge 3$ behandelt werden.

\end{document}