\documentclass[a4paper, extrafontsizes, ngerman, 25pt]{memoir}
\usepackage[left=1cm, right=1cm, top=1cm, bottom=1cm]{geometry}
\usepackage{amssymb}
\usepackage{amsmath}
\usepackage{svg}
\usepackage{float}

\renewcommand{\familydefault}{\sfdefault}
\newcommand{\Z}{\mathbb{Z}}
\newcommand{\N}{\mathbb{N}}

\begin{document}
\subsection{Einführung}

Pollards Rho-Methode ist einer der schnellsten Algorithmen zur Primfaktorzerlegung kleiner Zahlen. Bei der Implementierung des Algorithmus kann ein Parameter $k$ gewählt werden, der unter Umständen großen Einfluss auf die Laufzeit hat, sowohl im positiven als auch im negativen Sinn. In dieser Arbeit wurde untersucht, wie $k$ bestmöglich gewählt wird, wenn der Algorithmus auf mehreren Maschinen gleichzeitig ausgeführt wird.

\subsection{Pollard's Rho-Methode}

Die zu faktorisierende Zahl wird $n$ genannt. In der Rho-Methode wird zunächst ein Anfangswert $0 \le x_0 < n$ zufällig gewählt. Anschließend wird die Funktion $f(x) = x^{2k} + 1$ wiederholt auf $x_0$ angewandt (hier wird der genannte Parameter $k$ verwendet). In der daraus entstehende Folge an Zahlen $x_0, f(x_0), f(f(x_0)), \dots$ wird jeweils nur der Rest bei Division durch $n$ gespeichert (die Zahlen werden \emph{modulo} $n$ betrachtet). Der Rest modulo $n$ nach $i$-maliger Anwendung von $f$ wird mit $x_i$ bezeichnet. Wenn nun $p$ ein Primfaktor von $n$ ist und die Folge $x_0, f(x_0),$ $f(f(x_0)), \dots$ modulo $n$ berechnet wird, wird sie auch implizit modulo $p$ berechnet. Die Folge ist natürlich nicht wirklich modulo $p$ bekannt, da $p$ als Primfaktor von $n$ unbekannt ist. Aber nimmt man den Rest von $x_i$ modulo $p$, erhält man das gleiche Ergebnis, wie nach $i$-maliger Anwendung von $f$ und Reduktion modulo $p$.

Die Idee der Rho-Methode ist es, dass die Folge modulo $p$ wesentlich früher einen Wert zum zweiten Mal annehmen wird, da $p$ deutlich kleiner als $n$ ist. Haben beispielsweise $x_i$ und $x_j$ ($i \ne j$) den gleichen Rest modulo $p$ aber nicht modulo $n$, ist $\gcd(x_i - x_j, n)$ ein Faktor von $n$, wobei $\gcd$ der größte gemeinsame Teiler ist. Das Problem reduziert sich also darauf, $x_i$ und $x_j$ mit gleichem Rest modulo $p$ zu finden.

Zum Finden solcher $x_i, x_j$ ist es hilfreich, den funktionalen Graphen von $f$ zu betrachten. Der funktionale Graph von $f$ enthält alle möglichen Reste modulo $p$, also $0, 1, \dots p - 1$, als Knoten und eine Kante von $a$ nach $b$, wenn $f(a) = b$ modulo $p$. Die Folgenglieder von $x_0, f(x_0), f(f(x_0)), \dots$ sind also genau die Knoten, die man besucht, wenn man in dem funktionalen Graphen bei $x_0$ startet und immer die ausgehende Kante von jedem Knoten nimmt. Dann reduziert sich das Finden von $x_i, x_j$ darauf, einen Zyklus in dem Graphen zu finden. Denn ein Knoten im Zyklus wird bei der Berechnung der Folge wieder besucht, wird er also beispielsweise im $i$-ten und $j$-ten Schritt besucht, haben $x_i$ und $x_j$ den gleichen Rest modulo $p$.

\newpage

\subsection{Vorgehen}

Das Ziel war es, durch eine geeignete mathematische Beschreibung des Rho-Algorithmus eine Formel für die Laufzeit in Abhängigkeit der Werte von $k$ für jede Maschine herzuleiten. Es wurde wie üblich für den Rho-Algorithmus angenommen, dass sich $f(x) = x^{2k} + 1$ wie eine zufällige Funktion modulo $p$ verhält, sodass der Algorithmus wahrscheinlichkeitsbasiert analysiert werden konnte. Damit sollten die optimalen Werte für $k$ bestimmt werden.

\vspace{0.5cm}
\noindent \textbf{Abhängige und unabhängige Maschinen.} Eine Schwierigkeit bei der Analyse war, dass zwei Maschinen mit gleichem $k$ stochastisch abhängig sind. Der Fall stochastisch unabhängiger Maschinen, in dem nur paarweise verschiedene Werte von $k$ möglich sind, wurde daher vom Fall abhängiger Maschinen unterschieden. Im Fall unabhängiger Maschinen wurde folgende Formel als asymptotische Näherung für die Laufzeit gefunden:

\begin{figure}[H]
    \hspace{3.5cm} \includesvg[height=70pt]{formula-indep}
\end{figure}

\noindent $M$ bezeichnet hier die Anzahl an Maschinen und $k_i$ den Wert von $k$ für die $i$-te Maschine.

Der Fall abhängiger Maschinen konnte für $M = 2$ gelöst werden. Mithilfe erzeugender Funktionen und einiger Fallunterscheidungen wurde folgende Formel bestimmt.

\begin{figure}[H]
    \centering
    \includesvg[height=60pt]{formula-dep2}
\end{figure}

\noindent \textbf{Laufzeitmessungen.} Es wurden Laufzeitmessungen für eine und zwei Maschinen durchgeführt, um zu zeigen, dass die hergeleiteten Formeln das Laufzeitverhalten des Algorithmus gut beschreiben. Dafür wurden für verschiedene Werte von $k$ jeweils $2^{20}$ zufällige 62-Bit Zahlen mit einem 21-Bit und einem 41-Bit Faktor gewählt und die Zeit gemessen, bis ein Faktor gefunden wurde. Die Ergebnisse sind in Abbildung 1 und 2 dargestellt.

\subsection{Ergebnisse}

Für den Fall einer Maschine konnte unter vereinfachenden Annahmen mithilfe obiger Formeln gezeigt werden, dass $k = 1$ optimal ist. Im Fall zweier Maschinen wurde gezeigt, dass für beide Maschinen $k = 1$ zu wählen besser ist, als wenn $k$ bei beiden Maschinen gleich ist oder jeweils eine Primzahl ist. Es wird vermutet und von Experimenten bestätigt, dass $k = 1$ für beide Maschinen optimal ist.

\end{document}